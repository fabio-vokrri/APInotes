\documentclass{book}

% document style
\usepackage[italian]{babel}
\usepackage[
  a4paper,
  bindingoffset=0in,
  left=1in,
  right=1in,
  top=1in,
  bottom=1in,
  footskip=.25in
]{geometry}

% references
\usepackage{hyperref}
\hypersetup{
  pdftitle={Appunti di Algoritmi e Prencipi dell'Informatica},
  pdfauthor={Fabio Vokrri},
  pdfsubject={Corso di Algoritmi e Prencipi dell'Informatica del professor Davide Martinenghi, Politecnico di Milano 2021-2022},
  pdfkeywords={Informatica Teorica, Strutture Dati, Algoritmi},  colorlinks=false,
  linkcolor=black,
  pdfpagemode=FullScreen,
}

% graphics and images
\usepackage{tikz}
\usetikzlibrary{automata, arrows.meta, positioning}

\usepackage{graphicx}
\graphicspath{ {../images/} }

% math
\usepackage{amsmath}
\usepackage{amsfonts}

\newtheorem{theorem}{Teorema}[section]
\newtheorem{definition}{Definizione}[section]
\newtheorem{thesis}{Tesi}[section]

% coding style
\newcommand{\code}[1]{\texttt{#1}}
\usepackage{listings}

\lstdefinestyle{mystyle}{
  basicstyle=\ttfamily,
  numberstyle=\tiny,
  breakatwhitespace=false,
  breaklines=true,
  captionpos=b,
  keepspaces=true,
  numbers=left,
  numbersep=7.5pt,
  showspaces=false,
  showstringspaces=false,
  showtabs=false,
  tabsize=2,
  xleftmargin=20pt,
  frame=l,
  aboveskip=10pt,
  belowskip=10pt,
  keywordstyle=\bf\itshape,
  classoffset=0,
  morekeywords={for, down, to, by, while, and, or, repeat, until, if, else, return},
}

\lstset{style=mystyle}

\renewcommand\lstlistingname{codice}
\renewcommand\lstlistlistingname{codice}

% MAIN
\begin{document}

  \title{\Huge \textbf{Algoritmi e Principi dell'Informatica}}
\author{\huge Fabio Vokrri}

\begin{titlepage}
  \maketitle
\end{titlepage}

  \tableofcontents
  \clearpage

  \part{Informatica Teorica}

  \chapter{Linguaggi Formali}

  \section{Alfabeti e Stringhe}
  Di seguito sono riportate alcune definizioni importanti riguardanti i linguaggi formali:
  \begin{definition}[Alfabeto]
    Un alfabeto (o vocabolario) V è un insieme finito di oggetti elementari detti simboli (o caratteri)
  \end{definition} 

  \begin{definition} [Stringa]
    Una stringa x appartenente ad un alfabeto V è una sequenza finita e ordinata di caratteri appartenenti a tale alfabeto.
  \end{definition}

  Data una stringa \(x\), la notazione \(|\,x\,|\) indica la lunghezza di tale stringa, ovvero il numero di caratteri di cui è composta (la cardinalità del suo dominio \(V\)). Per convenzione la stringa vuota, che non contiene nessun carattere, è indicata con la lettera \(\varepsilon\). La stringa vuota \(\varepsilon\) ha lunghezza zero. 

  Per poter confrontare due stringhe è necessario verificare che:
  \begin{itemize}
    \item \(|x| = |y|\)
    \item \(x_i = y_i\; \forall i=0,1,2,...,|x|\)
  \end{itemize}

  La concatenazione (o prodotto) \(x.y\) di due stringhe \(x\) e \(y\) è una stringa composta da tutti i caratteri di \(x\) seguiti da tutti i caratteri di \(y\). Inoltre, la concatenazione \(x.\varepsilon\) produce come risultato la stringa \(x\).

  Data una stringa \(s\), una stringa \(x\) è sottostringa (o fattore) di \(s\) se esistono due stringhe \(y\) e \(z\), tali che \(s=yxz\). Inoltre:
  \begin{itemize}
    \item se \(y=\varepsilon\), \(x\) è detta prefisso.
    \item se \(z=\varepsilon\), \(x\) è detta suffisso.
    \item se \(z=y=\varepsilon\), \(x=s\)
  \end{itemize}

  Data una stringa \(x\), l'espressione \(x^i\) indica la concatenazione della stringa \(x\) con sè stessa per \(i-1\) volte, ovvero \(x^0 = \varepsilon\) e \(x^{i+1} = x.x^{i}\).

  \section{Operatori di Kleene}
  Per poter proseguire con il trattato, è prima necessario introdurre alcuni concetti matematici fondamentali. Si danno quindi le seguenti definizioni:
  \begin{definition}[Semigruppo] \label{Semigruppo}
    Un semigruppo è una coppia \(<S, \circ >\), dove:
    \begin{itemize}
      \item \(S\) è un insieme chiuso rispetto a \(\circ\); per cui, se si prendono due qualsiasi elementi \(A\) e \(B\) di tale insieme, l'operazione \(A\circ B\) produce come risultato un elemento appartenente ad \(S\);
      \item \(\circ\) è un'operazione associativa su \(S\).
    \end{itemize}
  \end{definition}

  Nel contesto dei linguaggi, l'operatore \(\circ\) rappresenta la concatenazione di stringhe.
  
  \begin{definition}[Monoide] \label{Monoide}
    Un monoide è un semigruppo in cui è definito un elemento unitario \(u\in S\), tale che \(\forall x, \exists u(x\circ u = x)\).
  \end{definition}

  Nel contesto dei linguaggi, l'elemento unitario \(u\) rappresenta la stringa vuota \(\varepsilon\): infatti, la concatenazione di una generica stringa \(x\) con la stringa vuota \(\varepsilon\) produce come risultato nuovamente la stringa originaria \(x\). 

  \begin{definition}[Gruppo] \label{Gruppo}
    Un gruppo è un monoide in cui è definito un elemento inverso \(x^{-1}\) unico per ogni elemento \(x\) dell'insieme \(S\), tale che \(\forall x(x\circ x^{-1} = u)\).
  \end{definition}

  Date le definizioni \ref{Semigruppo}, \ref{Monoide} e \ref{Gruppo}, si possono definire gli operatori di Kleene:

  \begin{definition} [Più di Kleene]
    Sia \(<S, \circ>\) un semigruppo. Per ogni \(X\subseteq S\), \(S^+\) indica il sottoinsieme di S generato da X, ovvero l'insieme di tutti gli elementi \(s\in S\) tale per cui \(s=x_1\circ x_2\circ...\circ x_n\) per qualche \(n\geq 1\), con \(x_i\in X\) (stringhe non vuote).
  \end{definition}

  Nel contesto dei linguaggi, l'operatore + di Kleene rappresenta l'insieme infinito di stringhe non vuote, che si possono generare a partire dall'insieme \(S\) di simboli, a cui si applica tale operatore.

  \begin{definition} [Stella di Kleene]
    Se \(<S, \circ, u>\) è un monoide, allora \(X^*=X^+\cup\{u\}\) è un monoide (più precisamente un sottomonoide) di S ed è detto monoide libero generato da X.
  \end{definition} 

  Nel contesto dei linguaggi, la stella di Kleene rappresenta l'insieme infinito di stringhe, inclusa la stringa vuota \(\varepsilon\), che si possono generare a partire dall'insieme \(S\) di simboli, a cui si applica tale operatore. Quindi, \(X^*=X^+\cup \{\varepsilon\}\).

  Ad esempio, dato l'insieme di simboli \(S=\{a,b,c\}\), \(S^+=\{a,b,c,aa,ab,ac,ba,bb,bc,ca,cb,cc,aaa...\}\) e \(S^*=\{\varepsilon,a,b,c,aa,ab,ac,ba,bb,bc,ca,cb,cc,aaa...\}\). 

  \section{Linguaggi} 
  Data la definizione della stella di Kleene, si può ora dare la definizione di linguaggio:
  \begin{definition} [Linguaggio]
    Un linguaggio L su un alfabeto V è un sottoinsieme di \(V^*\).
  \end{definition}
  Dato un insieme di simboli, si possono generare infiniti linguaggi.
  
  \vspace{10pt}

  \noindent 
  Poichè i linguaggi sono un inisieme di stringhe, valgono tutte le operazioni insiemistiche come:
  \begin{itemize}
    \item Unione (\(L_1\cup L_2\)): l'inisieme di tutte le stringhe che appartengono o ad \(L_1\) o ad \(L_2\) o ad entrambi i linguaggi;
    \item Intersezione (\(L_1 \cap L_2\)): l'insieme di tutte le stringhe che appartengono sia ad \(L_1\) che ad \(L_2\);
    \item Differenza (\(L_1\backslash L_2\)): l'insieme di tutte le stringhe che appartengono ad \(L_1\) ma non ad \(L_2\);
    \item Complementare (\(L^C = A^* \backslash L\)): l'insieme di tutte le stringhe che non appartengono al linguaggio \(L\);
    \item Concatenazione (\(L_1.L_2\)): l'insieme di tutte le stringhe che si ottengono concatenando ad ogni stringa di \(L_1\) ogni stringa di \(L_2\); formalmente \(L_1.L_2=\{xy:x\in L_1, \, y\in L_2\}\);
    \item Potenza \(n\)-esima (\(L^n\)): l'insieme di tutte le stringhe che si ottengono concatenando \(L\) con sè stesso \(n\) volte, utilizzando le regole della concatenazione precedentemente definite;
    \item Chiusura di Kleene \(\biggl(\displaystyle L^*=\bigcup_{n=0}^\infty L^n\) e \(\displaystyle L^+=\bigcup_{n=1}^\infty L^n\biggr)\)
  \end{itemize}
  Le operazioni su di un determinato linguaggio crea nuove classi di linguaggi con caratteristiche proprie, talvolta interessanti. Un linguaggio diventa di interesse nel momento in cui le stringhe di cui è composto possono essere utilizzate per veicolare informazioni, problemi, soluzioni o per rappresentare programmi, documenti, elementi multimediali o, nel caso più rilevante, per rappresentare computazioni.
  \chapter{Algoritmi}
In questo capitolo si analizzano a fondo i principali algoritmi di ordinamento e i relativi tempi di esecuzione. Nello specifico, si utilizzerà come modello di riferimento la macchina RAM con un criterio di costo costante, come analizzato nei capitoli precedenti. Prima di proseguire nella trattazione è necessario dare una definizione generale di algoritmo:

\begin{definition}
  Un algoritmo è una procedura di calcolo ben definita che prende un certo valore, o un insieme di valori, in input e genera un valore, o un insieme di valori, in output. Dunque, un algoritmo è una serie di passi computazionali che trasformano l'input in output.
\end{definition}

Un algoritmo può anche essere visto come uno strumento per la risoluzione di un problema computazionale ben definito: sotto questo sguardo, un algoritmo si definisce corretto se, per ogni istanza di input, termina con l'output corretto. Se un algoritmo è corretto, allora risolve quel determinato problema computazionale. Esistono molti modi per poter specificare un determinato algoritmo: si può utilizzare la lingua italiana o inglese, ma anche un linguaggio di programmazione come C, C++, JAVA e Pascal, o ancora tramite uno pseudocodice.

\section{Pseudocodifica}
La pseudocodifica può avvenire in molti modi, ma nel seguito si utilizzeranno le convenzioni qui riportate:
\begin{itemize}
  \item L'indentazione serve ad indicare la struttura a blocchi dello pesudocodice, in modo da comprendere quali istruzioni appartengono, per esempio, ad un ciclo \code{for}, a un ciclo \code{while} o ad un \code{if}-\code{else} statement. Non sono utilizzate le parentesi graffe o parole chiave come begin ed end in quanto appesantiscono la sintassi;
  \item I costrutti iterativi \code{while}, \code{for}, \code{repeat-until} e il costrutto condizionale \code{if-else} hanno interpretazioni simili a quelle dei comuni linguaggi di programmazione. Il contatore del ciclo mantiene il suo valore dopo la fine del ciclo, quindi il valore che ha provocato la terminazione del ciclo stesso. Inoltre, si utilizza la parola chiave \code{to} quando il ciclo \code{for} incrementa il valore del suo contatore ad ogni iterazione, mentre si utilizza la parola chiave \code{down to} nel caso la variabile venga decrementata;
  \item Le assegnazioni di un valore ad una certa variabile avviene con il simbolo \code{:=}, differente dall'operatore \code{=}, che invece indica l'eguaglianza di due valori all'interno di un costrutto \code{if};
  \item Per identificare un elemento appartenente ad un array, si utilizza la notazione con le parentesi quadre, al cui interno si indica l'indice dell'elemento a cui si vuole accedere: \code{array[i]}; Per indicare un intervallo di valori all'interno dell'array si utilizza la seguente sintassi: \code{array[i..j]}, con cui si indica la sottomatrice composta dagli elementi compresi fra \(i\) e \(j\); 
  \item I dati utilizzati sono tipicamente organizzati in oggetti, formati da attributi, a cui si accede tramite la notazione punto: \code{oggetto.prop}. Le variabili che rappresentano un determinato oggetto sono trattate come puntatori a tale oggetto. Un puntatore che non fa riferimento ad alcun oggetto è inizializzato con il valore \code{NIL};
  \item I parametri vengono passati ad una procedura per valore: la procedura chiamata riceve una sua copia dei parametri e, quindi, se a una di queste variabili è assegnato un nuovo valore, la modifica non è visibile dalla procedura chiamante. Nel caso venga passato come argomento un oggetto, viene copiato il puntatore a tale oggetto e quindi le modifiche sono visibili anche dalla procedura chiamante;
  \item L'istruzione \code{return} restituisce immediatamente il controllo al punto in cui la procedura chiamante ha effettuato la chiamata. Le istruzioni \code{return} possono anche ritornare un valore al chiamante;
  \item Gli operatori booleani \code{and} e \code{or} sono cortocircuitati. Ciò significa che nella valutazione dell'espressione \code{x and y}, si valuta prima se il valore di \code{x} sia falso, in quanto, se lo fosse, l'intera espressione sarebbe falsa e non avrebbe quindi alcun senso valutare il valore della variabile \code{y}. Al contrario, nella valutazione dell'espressione \code{x or y}, si verifica innanzitutto se il valore di \code{x} sia vero, in quanto, se lo fosse, l'intera espressione sarebbe vera e non avrebbe quindi alcun senso valutare il valore della variabile \code{y}.
\end{itemize}

Tramite queste regole è possibile definire un generico algoritmo.

\section{Insertion Sort}
Una classe di algoritmi molto studiati è quella riguardante l'ordinamento di un vettore, che consiste nella disposizione dei suoi elementi in ordine crescente.

\vspace{10pt}

Il primo algoritmo analizzato è l'\textbf{insertion sort}, che prende in input una sequenza di \(n\) numeri \([a_1, a_2, ...,a_n]\) e restituisce in output una permutazione \([a_1', a_2',...,a_n']\) tale che \(a_1'\le a_2' \le ... \le a_n'\). Questo algoritmo ordina sul posto \footnote{L'algoritmo risistema gli elementi della sequenza all'interno dell'array avendo, in ogni istante, al più un numero finito di elementi memorizzati all'esterno dell'array: ciò permette di risparmiare memoria nel calcolatore.} gli elementi assumendo che la sequenza da ordinare sia inizialmente partizionata in una sottosequenza già ordinata, all'inizio composta da un unico elemento (il primo dell'array), e una sottosequenza ancora da ordinare. Ad ogni iterazione viene rimosso un elemento dalla sottosequenza non ordinata e inserita nella posizione corretta all'interno della sottosequenza già ordinata. 

In pseudocodice:

\lstinputlisting{../docs/algorithms/insertion_sort.txt}

All'inizio di ogni iterazione del ciclo \code{for}, il cui indice è \(j\), la sottosequenza di elementi \code{A[1..j-1]} è la parte ordinata dell'array, mentre la sottosequenza \code{A[j+1..n]} è costituita da elementi ancora da ordinare.

\vspace{10pt}

Si analizza ora il tempo di esecuzione della procedura \code{insertion sort}: per ogni \(j=2,3,...,n\) in cui \(n\) = \code{A.length}, si indica con \(t_j\) il numero di volte che il test del ciclo \code{while} nella riga 5 viene eseguito per quel determinato valore di \(j\).

\begin{table}[!h]
  \centering
  \begin{tabular}{l l l}
    Codice & Costo & Numero di volte \\
    \hline
    \code{for j:= 2 to A.length} & \(c_1\) & \(n\) \\
    \code{key := A[j]} & \(c_2\) & \(n-1\) \\
    \code{i := j - 1} & \(c_3\) & \(n-1\) \\
    \code{while i > 0 and A[i] > key} & \(c_4\) & \(\sum_{j=2}^n{t_j}\) \\
    \code{A[i + 1] := A[i]} & \(c_5\) & \(\sum_{j=2}^n{(t_j-1)}\) \\
    \code{i := i - 1} & \(c_6\) & \(\sum_{j=2}^n{(t_j-1)}\) \\
    \code{A[i + 1] := key} & \(c_7\) & \(n-1\) \\
  \end{tabular}
  
\end{table}


Ad ogni riga di codice viene associato un costo \(c_i\) che va moltiplicato per il numero di volte che tale riga viene eseguita. Il tempo totale di esecuzione si calcola, dunque, sommando i vari contributi di tempo di ogni riga, ottenendo così l'espressione di \(T(n)\):

\begin{equation*}
  \displaystyle T(n) = c_1n+c_2(n-1)+c_3(n-1)+c_4\sum_{j=2}^n{t_j}+c_5\sum_{j=2}^n(t_j-1)+c_6\sum_{j=2}^n(t_j-1)+c_7(n-1)
\end{equation*}

Ovviamente, il caso migliore si verifica quando l'array in input è già ordinato. In questo caso, \(t_j = 1 \;\; \forall j=2,3...,n\) e l'espressione di \(T(n)\) assume la forma:

\begin{equation*}
  T(n) = (c_1+c_2+c_3+c_4+c_7)n - (c_2+c_3+c_4+c_7)
\end{equation*}

che è funzione lineare di \(n\). Dunque, \(T(n)=\Theta(n)\).

Al contrario, il caso pessimo si verifica quando l'array in input è ordinato, ma in ordine decrescente. In questo caso \(t_j = j \;\; \forall j=2,3...,n\) e l'espressione di \(T(n)\) assume la forma:

\begin{equation*}
  T(n) = \frac{1}{2}(c_4+c_5+c_6)n^2+(c_1+c_2+c_3)n+\frac{1}{2}(c_4-c_5-c_6+c_8)n-(c_2+c_3+c_4+c_7)
\end{equation*}

che è funzione quadratica di \(n\). Dunque, \(T(n)=\Theta(n^2)\).

\section{Merge Sort}
L'algoritmo appena analizzato utilizza un approccio di tipo incrementale: dopo aver ordinato il sottoarray \code{A[1..j-1]} inserisce l'elemento \code{A[j]} nella posizione corretta, ottenendo il sottoarray ordinato \code{A[1..j]}. Nel seguito, invece, si analizza un secondo approccio, più efficiente del primo, soprattutto per array di molti elementi: Divide et Impera. Questo criterio si basa sulla suddivisione ricorsiva del problema in sottoproblemi più piccoli, simili a quello originario, ma di dimensione ridotta, per poi risolvere i sottoproblemi di dimensione minima e fondere i risultati ottenuti, per costruire una soluzione generale del problema originario. 

Il paradigma Divide et Impera, si basa in realtà su tre passaggi:
\begin{enumerate}
  \item Divide: il problema viene suddiviso in un certo numero di sottoproblemi, che sono istanze più piccole del problema originario, fino ad ottenere sottoproblemi minimi, non più divisibili;
  \item Impera: i sottoproblemi di dimensione minima vengono risolti in maniera ricorsiva; se i problemi hanno dimensione sufficientemente piccola vengono risolti direttamente;
  \item Combina: le soluzioni dei sottoproblemi vengono combinate per generare la soluzione del problema generale.
\end{enumerate}

Un tipico algoritmo che segue questo metodo di risoluzione è il \textbf{merge sort}, che suddivide l'array originario a metà e ordina ricorsivamente i due sottoarray ottenuti, chiamando sè stesso fino ad ottenere sequenze di dimensione uno, di per sè già ordinate. A questo punto, le sottosequenze vengono fuse in modo da ottenere un array ordinato. 

Quest'ultimo passaggio viene effettuato tramite una procedura ausiliaria \code{merge(A,p,q,r)}, dove \(A\) è un array, e \(p,q,r\) sono tre indici dell'array tali che \(p\le q < r\).
La procedura assume che le sottosequenze \code{A[p..q]} e \code{A[q+1..r]} siano ordinate e, quindi, le fonde per formare un unico sottoarray ordinato che sostituisce il sottoarray corrente \code{A[p..r]}. La procedura \code{merge(A,p,q,r)} impiega un tempo \(\Theta(n)\) con \(n=r-p+1\) il numero di elementi da fondere. Ad ogni iterazione, la procedura \code{merge} confronta gli elementi più piccoli dei due sottoarray, inserendoli nel sottoarray "successivo" fino a quando uno dei due sottoarray è vuoto: a quel punto, i restanti elementi del sottoarray rimanente vengono copiati per completare l'array "successivo". Da un punto di vista computazionale, ogni iterazione della procedura impiega un tempo costante, in quanto deve semplicemente confrontare i due elementi dei due sottoarray. Poichè tale procedura viene effettuata per un massimo di \(n\) volte, la fusione impiega un tempo \(\Theta(n)\).

In pseudocodice:

\lstinputlisting[mathescape=true]{../docs/algorithms/merge.txt}

\noindent
In altri termini, le righe 2 e 3 inizializzano i valori di \(n_1\) ed \(n_2\), che rappresentano la lunghezza dei due sottoarray \code{A[p..q]} e \code{A[q+1..r]}. Nelle due righe successive vengono creati i due sottoarray ausiliari \code{L} (per Left) ed \code{R} (per Right), che contano \(n+1\) elementi (per motivi che verranno chiariti a breve). Le righe dalla 6 alla 9, inizializzano gli array appena creati con i valori contenuti rispettivamente nella prima e nella seconda metà dell'array \code{A}. Le righe 10 e 11 inizializzano l'ultimo (\(n+1\) -esimo) elemento dei due sottoarray \code{L} ed \code{R}, con un valore sentinella. Impostando tale valore ad \(\infty\), si è certi che non possa essere il valore più piccolo fra i due confrontati: in questo modo, una volta arrivati alla fine di uno dei due sottoarray, gli elementi dell'altro vengono ricopiati nell'array "successivo" in quanto necessariamente più piccoli di \(\infty\). Le ultime righe (dalla 12 alla 20) implementano la logica del confronto e dell'inserimento dell'elemento correntemente più piccolo nell'array \code{A}.

\vspace{10pt}

Una volta analizzata la procedura \code{merge}, si può introdurre l'algoritmo di ordinamento \code{mergeSort}. In pseudocodice:

\lstinputlisting[mathescape=true]{../docs/algorithms/merge_sort.txt}

L'algoritmo calcola, in riga 2, un indice \code{q}, che serve a suddividere l'array \code{A} in due sottoarray che contengono rispettivamente \(\lceil n/2 \rceil\) elementi ed \(\lfloor n/2 \rfloor\) elementi, su cui richiama ricorsivamente sè stessa. Una volta suddiviso l'array \code{A} in sottoarray di dimensione minima, viene chiamata la procedura \code{merge}, precedentemente analizzata. 

Come si può facilmente osservare, la procedura \code{mergeSort} è definita in maniera ricorsiva, quindi l'analisi delle prestazioni temporali diventa leggermente più complessa: infatti, si deve necessariamente far uso di un'equazione di ricorrenza, che esprime il tempo di esecuzione totale di un problema di dimensione \(n\), in funzione del tempo di esecuzione per input più piccoli. Se la dimensione del problema diventa sufficientemente piccola, per esempio \(n\le c\) per qualche costante \(c\), la soluzione del problema è diretta e richiede un tempo di esecuzione costante, indicata con \(\Theta(1)\). Si suppone, inoltre, che il problema originario venga suddiviso in \(a\) sottoproblemi, tutti di dimensione \(1/b\) volte la dimensione del problema originario. Dunque, è necessario un tempo \(T(n/b)\) per risolvere un sottoproblema di dimensione \(n/b\) e un tempo \(aT(n/b)\) per risolverli tutti. Infine, se si impiega un tempo \(D(n)\) per suddividere il problema in \(a\) sottoproblemi e un tempo \(C(n)\) per fonderne le soluzioni, si ottiene la ricorrenza:

\begin{equation*}
  T(n) = \begin{cases}
    \Theta(1) & if\;n\le c\\
    aT(n/b)+D(n)+C(n) & else
  \end{cases}
\end{equation*}

Per trovare ora il tempo di esecuzione \(T(n)\) nel caso peggiore si può ragionare come segue. Nel caso in cui i sottoarray abbiano cardinalità uno, la soluzione è diretta, quindi viene impiegato un tempo costante per risolvere il problema, mentre se i sottoarray hanno \(n > 1\) elementi, si suddivide il tempo di esecuzione impostando \(D(n) = \Theta(1)\), in quanto si impiega un tempo costante per calcolare il centro di un array, \(C(n)=\Theta(n)\), in quanto si è già precedentemente dimostrato che la procedura \code{merge} impieghi un tempo lineare per la fusione delle soluzioni, e infine si pone \(a=b=2\), in quanto si suddivide ricorsivamente il problema in due sottoproblemi di uguale dimensione \footnote{In realtà, sarebbe più accurato scrivere \(T(\lfloor n/2 \rfloor) + T(\lceil n/2 \rceil)\) in quanto non sempre la dimensione dell'array \code{A} è potenza di 2 e, dunque, divisibile ricorsivamente in due metà. Tale approssimazione, comunque, non influisce sulla complessità finale del calcolo.}. Con questo ragionamento, la ricorrenza assume l'espressione:

\begin{equation*}
  T(n)=\begin{cases}
    \Theta(1) & if\; n=1\\
    2T(n/2)+\Theta(n)+\Theta(1) & if\; n>1
  \end{cases}
\end{equation*}

Si può facilmente dimostrare (analiticamente oppure tramite il teorema dell'espreto, di cui si discuterà successivamente) che tale equazione ha soluzione \(T(n)=\Theta(n\,log_2\,n)\), che rappresenta il tempo di esecuzione dell'algoritmo \code{mergeSort} nel caso pessimo. Si può osservare come tale algoritmo sia decisamente migliore rispetto all'\code{insertionSort}, il cui tempo di esecuzione nel caso pessimo è \(\Theta(n^2)\).

Un modo per comprendere meglio come mai la complessità temporale del \code{mergeSort} sia proprio \(\Theta(n\,log_2\,n)\), si riscrive la ricorrenza nel seguente modo:

\begin{equation*}
  T(n)=\begin{cases}
    c & if\; n=1\\
    2T(n/2)+cn+c & if\; n>1
  \end{cases}
\end{equation*}

in cui la costante \(c\) rappresenta sia il tempo richiesto per risolvere i problemi di dimensione 1, sia il tempo per elemento dell'array dei passi divide e combina. Si può costruire un albero di ricorsione, in cui ogni ramo rappresenta una metà dell'array precedente e ogni foglia sia un array di dimensione unitaria. Il primo livello (in alto) ha un costo totale di \(cn\), il secondo livello ha un costo totale di \(cn/2 + cn/2 = cn\) e così via fino all'ultimo livello, con costo totale di \(n + n +...+ n\) (\(c\) volte), quindi di \(cn\). In generale, il livello \(i\) ha \(2^i\) nodi, ciascuno dei quali ha un costo di \(c(n/2^i)\), quindi, il numero totale di livelli dell'albero di ricorsione è \(log_2\,n+1\), con \(n\) la dimensione dell'input. Dunque, per calcolare il costo totale, basta sommare i costi di tutti i livelli, ottenendo \(cn(log_2\,n+1) = cn(log_2\,n)+cn\), ovvero \(\Theta(n\,log_2\,n)\).

\section{Risoluzione Ricorrenze}
Come già detto in precedenza, quando i problemi sono abbastanza grandi da essere risolti ricorsivamente, si ha il cosiddetto caso ricorsivo, tramite cui si divide il problema in problemi più piccoli di uguale natura. Una volta che i sottoproblemi diventano sufficientemente piccoli da non richiedere più il passo ricorsivo, si è raggiunto il cosiddetto caso base, da cui inizia la soluzione del problema. 

Questo modello di risoluzione del problema viene anche detto Divide et Impera e richiede l'utilizzo di equazioni di ricorrenza, tramite cui si caratterizzano i tempi di esecuzione degli algoritmi in termini dei loro valori con input più piccoli. Per risolvere tali equazioni, ovvero per trovare i limiti asintotici \(\Theta\) oppure \(O\), esistono tre metodi:
\begin{enumerate}
  \item Metodo di Sostituzione: si fa un'ipotesi di soluzione e si utilizza l'induzione matematica per dimostrare che l'ipotesi sia corretta;
  \item Metodo dell'Albero di ricorsione: si converte la ricorrenza in una struttura ad albero, i cui nodi rappresentano i costi ai vari livelli della ricorsione;
  \item Metodo dell'Esperto (Master theorem): fornisce i limiti per ricorrenze nella forma \\ \(T(n)=aT(n/b)+f(n)\) con \(a\ge 1, b>1\) e \(f(n)\) data. Una ricorrenza in questa forma caratterizza un algoritmo divide et impera che crea \(a\) sottoproblemi di dimensione \(1/b\), i cui passi divide e combina richiedono un tempo \(f(n)\).
\end{enumerate}

A volte, le ricorrenze non saranno delle uguaglianze, ma delle disuguaglianze nella forma \(T(n) \le ...\), che stabilisce un limite superiore su \(T(n)\) (quindi si utilizza la notazione \(O\) anzichè \(\Theta\)), oppure nella forma \(T(n) \ge ...\), che stabilisce invece un limite inferiore su \(T(n)\) (quindi si utilizza la notazione \(\Omega\) anzichè \(\Theta\)). Inoltre, ci sono casi in cui si trascurano dei dettagli tecnici di poca importanza, come le condizioni al contorno: infatti, poichè il tempo di esecuzione di un algoritmo con un input di dimensione costante è costante, le ricorrenze che ne derivano hanno \(T(n)=\Theta(1)\), per valori sufficientemente piccoli di \(n\). Questa decisione risiede nel fatto che, sebbene le condizioni al contorno cambino la soluzione esatta della ricorrenza, tuttavia la soluzione non cambia per più di un fattore costante e quindi asintoticamente rimane immutata.

\subsection{Metodo di Sostituzione}
Uno dei metodi per la risoluzione delle occorrenze e, quindi, per il calcolo del tempo di esecuzione degli algoritmi, è il metodo della sostituzione, che richiede due passaggi:
\begin{enumerate}
  \item Ipotizzare la forma della soluzione;
  \item Utilizzare l'induzione matematica per dimostrare che la soluzione ipotizzata sia corretta.
\end{enumerate}
Questo metodo può essere applicato solamente se si ha un'idea della forma generale della soluzione e si vuole calcolare il limite superiore o inferiore della ricorrenza che si analizza.

\textit{ESEMPIO:} Si determini il limite superiore della ricorrenza \(T(n)=2T(\lfloor n/2 \rfloor)+n\).

Si suppone che la soluzione sia \(O(n\,log_2\,n)\). Il metodo di sostituzione consiste nel dimostrare che \(T(n)\le c\,n\,log_2\,n\) per un generico \(c>0\). Si verifica, innanzitutto, che questo limite sia valido anche per \(\lfloor n/2 \rfloor\), ovvero che \(T(\lfloor n/2 \rfloor)\le c\,\lfloor n/2 \rfloor\,log_2(\lfloor n/2 \rfloor)\). Facendo le opportune sostituzioni si ha:  
\begin{flalign*}
  T(n)\;\;\; &\le \;\;\; 2(c\lfloor n/2 \rfloor log_2(\lfloor n/2 \rfloor)) + n &&\\
  &\le \;\;\; c\,n\,log_2(n/2)+n &&\\
  &= \;\;\; c\,n\,log_2\,n-c\,n\,log_2\,2+n &&\\
  &= \;\;\; c\,n\,log_2\,n-c\,n+n &&\\
  &\le \;\;\; c\,n\,log_2\,n &&
\end{flalign*}
L'ultimo passaggio è vero solo per \(c \ge 1\).
A questo punto, l'induzione matematica richiede di dimostrare che la soluzione vale per le condizioni al contorno. Si suppone, per esempio, che l'unica condizione al contorno sia \(T(1)=1\): si deve dimostrare che è possibile scegliere una costante \(c\) sufficientemente grande in modo che il limite \(T(n)\le c\,n\,log_2\,n\) sia valido anche per le condizioni al contorno. Quindi per \(n=1\) (condizione al contorno), il limite \(T(n)\le c\,n\,log_2\,n\) diventa \(T(1)\le c\, log_2\;1 = 0\), che però è in contrasto con \(T(1)=1\): il caso base della dimostrazione induttiva non è valido!

Questo ostacolo nella dimostrazione può essere facilmente superato sfruttando la notazione asintotica, che richiede di provare che \(T(n)\le c\,n\,log_2\,n\) sia valida solamente dopo un certo \(n_0\) in poi, scelto arbitrariamente: l'idea è quella di escludere la condizione al contorno dalla dimostrazione induttiva. Si osservi che, per \(n\ge 3\), la ricorrenza non dipende direttamente da \(T(1)\), quindi si può sostituire con \(T(2)\) e \(T(3)\), impiegati come casi base della dimostrazione induttiva. Inoltre, ponendo \(n_0=2\), se \(T(1)=1\) allora \(T(2)=4\) e \(T(3)=5\). Basta quindi determinare una costante \(c\) tale per cui \(T(2)= 4 \le 2c\,log_2(2)\) e \(T(3)= 5 \le 3c\,log_2(3)\): le precedenti condizioni sono soddisfatte solo per \(c\ge 2\).

\vspace*{10pt}

Non esiste un metodo unico e generale per indovinare la soluzione corretta di una ricorrenza, ma è possibile formulare delle buone ipotesi tramite il metodo dell'albero di ricorsione. Inoltre, se una ricorrenza è simile ad una   già risolta in precedenza, allora è possibile che anche la soluzione sia analoga. Un altro metodo per formulare un'ipotesi di soluzione consiste nel dimostrare dei limiti superiori e inferiori molto generali e larghi, per poi ridurre gradualmente il grado di incertezza, aumentando il limite inferiore e diminuendo il limite superiore.

Ci sono poi casi in cui la soluzione ipotizzata sembra essere corretta, ma i calcoli matematici non soddisfano il passo induttivo: solitamente, il problema risiede nel fatto che l'ipotesi induttiva non è abbastanza forte per dimostrare il limite esatto. In un caso del genere, spesso è necessario semplicemente correggere l'ipotesi sottraendo un termine di ordine inferiore per fare in modo che i calcoli soddisfino i requisiti. 

\textit{ESEMPIO:} Si calcoli la ricorrenza \(T(n)=T(\lfloor n/2 \rfloor)+T(\lceil n/2 \rceil)+1\) supponendo che la soluzione sia \(T(n)=O(n)\). Si deve quindi dimostrare che \(T(n)\le cn\) per qualche \(c\) arbitraria. Sostituendo l'ipotesi all'interno della ricorrenza si ottiene:
\begin{flalign*}
  T(n) \;\;\; &\le \;\;\; c\lfloor n/2\rfloor + c\,\lceil n/2 \rceil + 1 &&\\
  &=\;\;\; c\,n+1 &&
\end{flalign*}
che non implica che \(T(n)\le c\,n\) per qualunque valore di \(c\). Sembrerebbe quindi che l'ipotesi fatta sia sbagliata, ma al contrario si può dimostrare che è corretta, formulando un'ipotesi induttiva più forte. Per affrontare tale problema, si sottrae un termine di ordine inferiore dalla precedente ipotesi, ad esempio, un termine costante \(d\ge 0\), ottenendo come nuova ipotesi \(T(n)\le c\,n-d\), che sostituita alla ricorrenza:
\begin{flalign*}
  T(n)\;\;\; &\le \;\;\; (c\lfloor n/2 \rfloor -d)+ (c\lceil n/2 \rceil-d)+1 &&\\
  &= \;\;\; cn -2d +1 &&\\
  &\le\;\;\; cn-d
\end{flalign*}
che diventa valida per ogni \(d\ge 1\). Come prima, la costante \(c\) deve essere scelta arbitrariamente grande affinchè siano soddisfatte le condizioni al contorno.

\vspace*{10pt}

Infine, ci sono casi in cui tramite una piccola manipolazione algebrica è possibile rendere una ricorrenza ignota simile ad una più familiare.

\textit{ESEMPIO:} Si calcoli la ricorrenza \(T(n)=2T(\lfloor \sqrt{n} \rfloor)+ log_2(n)\). Tale ricorrenza sembra molto complessa da risolvere, ma è possibile semplificarla ponendo \(m=log_2n\), ottenendo così \\ \(T(2^m)=2T(2^{m/2})+m \). Chiamando \(S(m)\) la ricorrenza appena ottenuta, è possibile scrivere \(S(m)=2S(m/2)+m\), simile alla precedente ricorrenza analizzata \(T(n)=2T(\lfloor n/2 \rfloor)+n\); in effetti, la soluzione della ricorrenza \(S(m)\) è la stessa ottenuta in precedenza. Dunque, la soluzione è \(S(m)=m\,log_2\,m\) e, ripristinando i termini con la sostituzione \(m=log_2n\), si ottiene che \(T(n)=O(log_2n \cdot log_2(log_2n))\).

\subsection{Metodo dell'Albero di Ricorsione}
Dato che spesso è complesso formulare un'ipotesi di soluzione per una data ricorrenza, è possibile utilizzare il metodo dell'albero di ricorsione, in cui ogni nodo rappresenta il costo di un singolo sottoproblema. Sommando i costi dei nodi di ogni livello, si ottengono i costi relativi a quel livello e, sommando tali costi, si ottiene il costo generale della ricorrenza, che rappresenta l'ipotesi da verificare con il metodo della sostituzione. Utilizzando questo metodo, si tollera un certo livello di approssimazione, in quanto è interessante analizzare solamente il comportamento asintotico della ricorrenza: si possono quindi eliminare gli operatori 'ceil' e 'floor' e fare delle ipotesi blande per semplificare i calcoli.

\textit{ESEMPIO:} Si calcoli la ricorrenza \(T(n)=3T(\lfloor n/4 \rfloor)+\Theta(n^2)\). Come detto, si può approssimare la ricorrenza eliminando l'operatore floor, ottenendo \(T(n)=3T(n/4)+cn^2\), per una data costante \(c>0\). Per comodità, si suppone anche che \(n\) sia una potenza di 4, in modo tale che ogni livello dell'albero abbia dimensione intera. Si ottiene così il seguente albero delle ricorrenze:

\begin{figure}[!h]
  \centering
  \includegraphics[width=10cm]{albero ricorrenze.jpg}
  \caption{Albero della ricorrenza \(T(n)=3T(n/4)+cn^2\)}
\end{figure}

La parte \((a)\) della figura mostra \(T(n)\), che viene espanso nella parte \((b)\) in un albero equivalente che rappresenta la ricorrenza. Il termine \(cn^2\) nella radice di quest'albero rappresenta il costo al livello più alto della ricorsione, mentre i tre sottoalberi rappresentano i costi richiesti dai tre sottoproblemi di dimensione \(n/4\). La parte \((c)\) mostra l'espansione dei nodi di costo \(T(n/4)\) dalla parte \((b)\), in cui ogni nodo figlio ha costo \(c(n/4)^2\). Tale processo viene ripetuto più e più volte fino ad ottenere i casi base, rappresentati nella parte \((d)\) con \(T(1)\). 

La dimensione dei sottoproblemi per i nodi alla profondità \(i\) è di \(n/4^i\), quindi la dimensione del sottoproblema diventa 1 (dimensione delle foglie) quando \((n/4)^i=1\), ovvero quando \(i=log_4(n)\): dunque, l'albero della ricorrenza ha esattamente \(log_4n+1\) livelli. Ora, per determinare il costo di ogni livello, basti pensare che ogni nodo dell'albero genera tre sottonodi e che, dunque, il numero di nodi alla profondità \(i\) è \(3^i\). Moltiplicando il risultato appena ottenuto con il costo di un singolo nodo, si ottiene che ogni livello ha un costo di \(3^ic(n/4^i)^2 = (3/16)^icn^2\). L'ultimo livello dell'albero conta \(n^{log_43}\) nodi, ognuno di costo \(T(1)\), per un costo totale di \(n^{log_43}T(1)\), ovvero \(\Theta(n^{log_43})\).

A questo punto, si sommano i contributi di ogni livello, ottenendo:
\begin{flalign*}
  T(n)\;\;\; &= \;\;\; cn^2 + \frac{3}{16}cn^2 + \bigg(\frac{3}{16} \bigg)^2cn^2+...+\bigg(\frac{3}{16} \bigg)^{log_4n-1}cn^2+\Theta(n^{log_43}) &&\\
  &=\;\;\; \sum_{i=0}^{log_4n-1}\bigg(\frac{3}{16}\bigg)^icn^2+\Theta(n^{log_43}) &&\\
  &\overset{*}{<} \;\;\; \sum_{i=0}^{\infty}\bigg(\frac{3}{16}\bigg)^icn^2+\Theta(n^{log_43}) &&\\
  &= \;\;\; \frac{1}{1-(3/16)}cn^2 + \Theta(n^{log_43}) &&\\
  &= \;\;\; \frac{16}{13}cn^2 + \Theta(n^{log_43}) &&\\
  &= \;\;\; O(n^2).
\end{flalign*}

Il passaggio segnato con * rappresenta una piccola approssimazione: la \(\sum_{i=0}^{log_4n-1}\big(\frac{3}{16}\big)^icn^2\) ammette come limite superiore \(\sum_{i=0}^{\infty}\big(\frac{3}{16}\big)^icn^2\), che rappresenta una serie geometrica decrescente infinita. In questo modo è possibile proseguire con agilità i calcoli, ottenendo come ipotesi \(T(n)=O(n^2)\), che dovrà essere verificata con il metodo della sostituzione. 

\subsection{Metodo dell'Esperto}
Il metodo dell'esperto è impiegato per la risoluzione di ricorrenze del tipo \(T(n)=aT(n/b)+f(n)\), con \(a\ge 1, b>1\) costanti ed \(f(n)\) una funzione asintoticamente positiva. Una ricorrenza di questo tipo rappresenta il tempo di esecuzione di un algoritmo che divide il problema di dimensione \(n\) in \(a\) sottoproblemi di dimensione \(n/b\), mentre la funzione \(f(n)\) rappresenta il costo di divisione del problema e di combinazione delle soluzioni.
Il metodo dell'esperto dipende dal seguente teorema:

\begin{theorem}[Master Theorem]
  Date le costanti \(a\ge 1\), \(b>1\) e la funzione \(f(n)\), se la ricorsione \(T(n)\) si presenta nella forma \(T(n)=aT(n/b)+f(n)\), allora può essere limitata asintoticamente nei seguenti modi:
  \begin{enumerate}
    \item Se \(f(n)=O(n^{log_b a-\varepsilon})\) per qualche \(\varepsilon>0\), allora \(T(n)=\Theta(n^{log_b a})\);
    \item Se \(f(n)=\Theta(n^{log_b a})\), allora \(T(n)=\Theta(n^{log_b a}log_2(n))\);
    \item Se \(f(n)=\Omega(n^{log_b a +\varepsilon})\) per qualche \(\varepsilon>0\) e se \(af(n/b)\le cf(n)\) per qualche \(c<1\) e per ogni \(n\) sufficientemente grande, allora \(T(n)=\Theta(f(n))\).
  \end{enumerate}
\end{theorem}

Si osservi che in ciascuno dei tre casi, si confronta la funzione \(f(n)\) con la funzione \(n^{log_b a}\): intuitivamente, la soluzione della ricorrenza è determinata dalla funzione polinomialmente \footnote{Una funzione è polinomialmente più grande rispetto ad un'altra funzione se la prima è asintoticamente più grande della seconda di un fattore \(n^\varepsilon\) per qualche \(\varepsilon >0\).} più grande. Se la funzione \(n^{log_b a}\) è più grande polinomialmente, come nel caso uno, allora sarà soluzione della ricorrenza, altrimenti la soluzione sarà \(f(n)\), come enunciato nel caso tre, in cui si deve anche verificare la condizione di regolarità della funzione. Nel caso due, in cui le due funzioni sono asintoticamente uguali, si moltiplicano entrambi i membri per un fattore logaritmico e la soluzione sarà \(T(n)=\Theta(n^{log_b a}log_2(n)) = \Theta(f(n)log_2(n))\).

I tre casi, sfortunatamente, non coprono tutte le funzioni \(f(n)\) possibili, in quanto ci sarà un intervallo fra i casi 1 e 2, in cui la funzione \(f(n)\) è minore di \(n^{log_b a}\), ma non polinomialmente, mentre ci sarà anche un intervallo fra i casi 2 e 3, in cui la funzione \(f(n)\) è maggiore di \(n^{log_b a}\), ma non polinomialmente. In questi casi, il teorema dell'esperto non può essere applicato. 

Per utilizzare il teorema enunciato, bisogna semplicemente determinare in quali dei tre casi rientra la funzione \(f(n)\) e confrontarla con la funzione \(n^{log_b a}\).

\textit{ESEMPIO:} Si determini la soluzione della ricorrenza \(T(n)=9T(n/3)+n\). In questo caso, si ha che \(a=9, b=3\) ed \(f(n)=n\) e quindi \(n^{log_b a} = n^{log_3(9)}=\Theta(n^2)\). Dato che \(f(n)=O(n^{log_3(9)-\varepsilon})\), con \(\varepsilon = 1\) (in quanto \(f(n)=n\)), si può applicare il caso 1 del teorema dell'esperto e concludere immediatamente che la soluzione della ricorrenza è \(T(n)=\Theta(n^2)\), in quanto \(n^2\) è polinomialmente più grande di \(n\). 

\vspace*{10pt}

\textit{ESEMPIO:} Si determini la soluzione della ricorrenza \(T(n)=2T(n/2)+n\,log_2\,n\). In questo caso, si ha che \(a=2,b=2\) ed \(f(n)=n\,log_2\,n\) e quindi \(n^{log_b a}=n^{log_2(2)} = n\). Si potrebbe erroneamente pensare di essere nel terzo caso del teorema dell'esperto, ma le due funzioni non sono polinomialmente comparabili quindi non si può applicare il teorema. La ricorrenza, dunque, deve necessariamente essere risolta con l'utilizzo dei metodi precedentemente analizzati.

\section{Heap Sort}
Analizzando l'algoritmo Merge Sort si è constatato che è efficiente dal punto di vista temporale, ma non dal punto di vista spaziale, in quanto occupa una grande quantità di memoria. A questo proposito, si analizza ora l'algoritmo \textbf{Heap Sort}, che effettua un ordinamento sul posto degli elementi utilizzando una struttura dati detta Heap ('mucchio'), per la gestione delle informazioni.

Un Heap (binario) è una struttura dati ad albero binario quasi completo \footnote{Un albero binario quasi completo è una struttura dati ad albero in cui ogni livello è completo, eccetto per al più l'ultimo livello, che potrebbe essere completo solo fino ad un certo punto da sinistra}, in cui ogni nodo rappresenta un elemento dell'array da ordinare. Nello specifico, \code{A[1]} è la radice dell'albero, e per ogni elemento \code{A[i]}, \code{A[2i]} e \code{A[2i+1]} rappresentano i figli del nodo, mentre \lstinline[mathescape]{A[$\lfloor n/2 \rfloor$]} rappresenta il nodo padre. Si possono quindi definire le seguenti procedure:

\begin{lstlisting}[mathescape=true]
parent(i):
  return $\lfloor i/2 \rfloor$
\end{lstlisting}
\begin{lstlisting}[mathescape=true]
left(i):
  return $2i$
\end{lstlisting}
\begin{lstlisting}[mathescape]
right(i):
  return $2i+1$
\end{lstlisting}

Oltre all'attributo \code{A.length}, che ne ritorna la lunghezza, l'array \code{A} possiede in questo caso anche l'attributo \code{A.heapSize}, che indica il numero degli elementi dell'heap che sono registrati nell'array \code{A}. In altre parole, anche se l'array contiene \(n\) elementi, con \(n\)=\code{A.length}, soltanto gli elementi in \code{A[1..A.heapSize]}, con \(0\le\) \code{A.heapSize} \(\le\) \code{A.length}, sono elementi validi dell'heap.

Esistono, inoltre, due tipologie di heap binari: max-heap e min-heap. Il primo, più importante, è costruito in modo tale che ogni nodo rispetti la condizione per cui \lstinline[mathescape]{A[parent(i)] $\ge$ A[i]}; dunque, il valore di un nodo è al massimo il valore del nodo padre e, di conseguenza, l'elemento più grande di un max-heap è memorizzato alla sua radice. Il secondo, meno utilizzato, è costruito in modo tale che ogni nodo rispetti la condizione per cui \lstinline[mathescape]{A[parent(i)] $\le$ A[i]}. 

Per implementare l'algoritmo heapsort si fa utilizzo del max-heap; per poterne mantenere le proprietà si utilizza la procedura di supporto \code{maxHeapify}, un algoritmo che prende in input un array \code{A} e un suo indice \code{i}, e restituisce l'array ordinato in modo tale da rappresentare il max-heap. Quando tale procedura viene invocata, essa assume che gli alberi binari di radici \code{left(i)} e \code{right(i)} siano dei max-heap, ma assume anche che \code{A[i]} possa essere più piccolo dei suoi figli, violando la proprietà fondamentale. La procedura, quindi, ha il compito di far 'scendere' il valore \code{A[i]} in modo tale che il sottoalbero con radice di indice \(i\) diventi un max-heap.

\vspace{1in}

In pseudocodice:
\lstinputlisting[mathescape]{../docs/algorithms/max_heapify.txt}
 
A ogni passo viene determinato il più grande degli elementi \code{A[i]}, \code{A[left(i)]} e \code{A[right(i)]} e il suo indice viene memorizzato nella variabile \code{max}. Se \code{A[i]} è l'elemento più grande, allora il sottoalbero è già un max-heap e la procedura termina la propria esecuzione, altrimenti uno dei due figli contiene l'elemento più grande e \code{A[i]} viene scambiato con \code{A[max]} (\code{swap with}); in questo modo, il nodo di indice \(i\) e i suoi figli soddisfano la proprietà di max-heap. Il nodo con indice massimo, però, presenta il valore originale di \code{A[i]} e, quindi, il sottoalbero di radice \code{max} potrebbe violare la proprietà fondamentale: quindi, la procedura viene chiamata ricorsivamente sul sottoalbero, fino a raggiungere le foglie.

Questa procedura viene eseguita in un tempo \(O(h)\), con \(h\) l'altezza dell'albero. Essendo l'albero quasi completo, \(h=O(log\,n)\), quindi \(T(n)=O(log\,n)\).

\vspace*{10pt}

Ora, tramite la procedura \code{maxHeapify} è possibile convertire un array \code{A[1..n]} (con \(n\)=\code{A.length}) in un max-heap. Prima di procedere, è importante osservare come tutti gli elementi \lstinline[mathescape]{A[$\lfloor n/2 \rfloor + 1$ .. n]} siano foglie dell'albero e, quindi, ciascuno di essi è un heap di un solo elemento, che si può utilizzare come punto di partenza per la costruzione dell'heap. Si introduce, dunque, la procedura \code{buildMaxHeap}, che attraversa i nodi restanti dell'albero ed esegue la procedura \code{maxHeapify} in ciascuno di essi. 

In pseudocodice:
\lstinputlisting[mathescape]{../docs/algorithms/build_max_heap.txt}

Si può dimostrare che tale procedura impiega un tempo di esecuzione \(T(n)=O(n)\).

\vspace*{10pt}

A questo punto, è possibile scrivere l'algoritmo \code{heapSort}. In pseudocodice:

\lstinputlisting[mathescape]{../docs/algorithms/heap_sort.txt}

Questo algoritmo si basa sul fatto che, una volta riordinato l'array in maniera che rappresenti un max-heap, l'elemento più grande dell'array si trova in \code{A[1]}: questo elemento può quindi essere inserito nella posizione finale corretta scambiandolo con \code{A[n]}. Se ora si toglie il nodo \(n\) dall'heap, diminuendo \code{A.heapSize}, si nota che i figli della radice restano max-heap, ma la nuova radice potrebbe violare la proprietà del max-heap. Questo problema può essere rimosso chiamando la procedura \code{maxHeapify(A, 1)}, che lascia un max-heap in \code{A[1..n-1]}. Questa operazione viene ripetuta fino ad un heap di dimensione 2, già ordinato per definizione. 

Come detto in precedenza, la procedura \code{buildMaxHeap} impiega un tempo di esecuzione lineare (\(O(n)\)), mentre le \(n-1\) chiamate alla procedura \code{maxHeapify} impiegano ciascuna un tempo \(O(log\,n)\). Pertanto il tempo di esecuzione dell'\code{heapSort} impiega un tempo di esecuzione \(T(n)=O(n\,log\,n)\).

\section{Quick Sort}
\textbf{Quick sort} è un algoritmo di ordinamento divide et impera, il cui tempo di esecuzione nel caso peggiore è \(O(n^2)\). Nonostante un tempo di esecuzione molto lento nel caso peggiore, quick sort è uno degli algoritmi più utilizzati perchè ha un tempo medio atteso \(\Theta(n\,log\,)\) e i fattori costanti nascosti dalla notazione asintotica sono pressochè nulli. Inoltre, è un algoritmo di ordinamento sul posto, che lo rende utilizzabile anche in calcolatori con memoria limitata. 

L'idea generale di questo algoritmo si basa sui tre tipici passi del metodo divide et impera, per un sottoarray \code{A[p..r]}:
\begin{enumerate}
  \item Divide: partiziona l'array \code{A[p..r]} in due sottoarray \code{A[p..q-1]} e \code{A[q+1..r]} (eventualmente vuoti) tali che ogni elemento di \code{A[p..q-1]} sia minore o uguale di \code{A[q]} che, a sua volta, è minore o uguale di ogni elemento di \code{A[q+1..r]}.
  \item Impera: ordina i due sottoarray \code{A[p..q-1]} e \code{A[q+1..r]} chiamando ricorsivamente sè stesso.
  \item Combina: nessuna operazione di ricombinazione necessaria in quanto l'array \code{A[p..r]} è già ordinato.
\end{enumerate}

La procedura \code{quickSort} è implementata tramite il seguente pseudocodice:

\lstinputlisting{../docs/algorithms/quick_sort.txt}

Per poter ordinare un intero array \code{A}, la chiamata iniziale a tale algoritmo è \code{quickSort(A, 1, A.length)}. Si noti che all'interno di tale algoritmo viene chiamata la sottoprocedura \code{partition}, definita come segue in pseudocodifica:

\lstinputlisting{../docs/algorithms/partition.txt}

Questo algoritmo riarrangia il sottoarray \code{A[p..r]} sul posto selezionando un elemento \code{x = A[r]} come pivot, intorno a cui partizionare l'array. All'inizio di ogni iterazione del ciclo \code{for} di riga 4, per qualsiasi indice \(k\) si individuano quattro regioni:
\begin{enumerate}
  \item Se \(p\le k\le i\), allora \lstinline[mathescape]{A[k] $\le$ x};
  \item Se \(i+1\le k \le j-1\), allora \lstinline[mathescape]{A[k] $\ge$ x};
  \item Se \(k=r\), allora \code{A[k]=x};
  \item Gli indici \(k\) tali che \(j \le k \le r-1\) non hanno una particolare relazione con il pivot \(x\).
\end{enumerate}

Le ultime due righe della procedura, invece, inseriscono il pivot al suo posto nel mezzo dell'array, scambiandolo con l'elemento più a sinistra, che è maggiore di \(x\), e restituisce un nuovo indice di pivot. Il tempo di esecuzione dell'algoritmo \code{partition} con input il sottoarray \code{A[p..r]} è \(\Theta(n)\), con \(n=r-p+1\).

Il tempo di esecuzione dell'algoritmo \code{quickSort} dipende solamente da come viene partizionato l'array (in maniera bilanciata o meno) che, a sua volta, dipende da quali elementi vengono utilizzati per il partizionamento. Se il partizionamento è bilanciato, l'algoritmo ha un tempo di esecuzione \(\Theta(n\,log\,n)\), mentre nel caso peggiore, quando il partizionamento è sbilanciato, l'algoritmo converge ad una solzuione in un tempo \(\Theta(n^2)\).

\vspace*{10pt}

Il comportamento nel caso peggiore si verifica quando la subroutine \code{partition} produce un sottoproblema con \(n-1\) elementi e uno vuoto. Per calcolare il tempo di esecuzione, si suppone che questo sbilanciamento si verifichi per ogni chiamata ricorsiva. Il partizionamento costa un tempo di esecuzione \(\Theta(n)\) e, dato che uno dei due array è vuoto e l'altro conta \(n-1\) elementi, si ha un tempo totale:
\begin{equation*}
  T(n) = T(n-1) + T(0) + \Theta(n) = T(n-1) + \Theta(1) + \Theta(n)
\end{equation*}
Intuitivamente, e si sommano i costi ad ogni livello della ricorsione si ottiene una serie aritmetica, il cui valore è \(\Theta(n^2)\). Questa situazione si verifica quando l'array di partenza è già completamente ordinato. 

\vspace*{10pt}

Il comportamento nel caso ottimo si verifica quando la subroutine \code{partition} produce due sottoproblemi di dimensione non maggiore di \(n/2\): in questo caso il tempo di esecuzione dell'algoritmo è molto più rapido e avviene in un tempo totale \(T(n) \le 2T(n/2) + \Theta(n)\), che per il secondo caso del teorema dell'esperto, ha soluzione \(T(n)=\Theta(n\,log\,n)\). Si noti, inoltre, che nel caso in cui la partizione non fosse perfettamente bilanciata, l'algoritmo riuscirebbe comunque a riordinare l'array in un tempo \(T(n)=\Theta(n\,log\,n)\): questo dimostra che il \code{quickSort} è un algoritmo molto più vicino al caso ottimo che al caso pessimo, caso che si verifica in una sola istanza del problema (quando, appunto, è ordinato). 

\vspace*{10pt}

Il comportamento nel caso medio si verifica quando la subroutine \code{partition} produce una combinazione di partizioni 'buone' e 'cattive'. Si suppone, per semplicità, che le partizioni buone e cattive si alternino all'interno dell'albero di ricorsione e che quelle buone siano tutte nel caso migliore, mentre le quelle cattive siano nel caso pessimo. Si ipotizzi che nella radice dell'albero il costo di ripartizione è \(n\) e i sottoarray prodotti hanno dimensione \(n-1\) e 0 (caso pessimo), mentre nel livello successivo il partizionamento del sottoarray \(n-1\) produca due array di dimensione \((n-1)/2\) e \((n-1)/2 - 1\) (caso migliore). Il costo di una divisione cattiva, seguito da una divisione buona è comunque \(\Theta(n)\), ovvero lo stesso costo di una divisione buona: intuitivamente, quindi, la coppia divisione buona/cattiva impiega lo stesso tempo totale di esecuzione \(\Theta(n\,log\,n)\), con l'unica differenza che cambiano le costanti moltiplicative, eclissante nella notazione asintotica.

\section{Counting Sort}
Gli algoritmi analizzati fino ad ora, seppur differenti fra loro, condividono un'importante proprietà: l'ordinamento che effettuano è basato soltanto su confronti fra gli elementi di input. Questi algoritmi sono detti di ordinamento per confronti e, dati due elementi \(a_i\) e \(a_j\), eseguono uno dei test \(a_i < a_j, a_i\le a_j, a_i=a_j, a_i\ge a_j\) o \(a_i > a_j\) per determinare il loro ordine relativo. 

Gli algoritmi di ordinamento per confronti possono essere visti in termini di alberi di decisione, ovvero alberi binari completi che rappresentano i confronti fra gli elementi effettuati da un determinato algoritmo di ordinamento. Ora, l'esecuzione di un algoritmo di ordinamento corrisponde ad indicare su tale albero un cammino semplice che collega la radice dell'albero con una foglia (nello specifico, una delle foglie che rappresentano le permutazioni ordinate dell'array di ingresso). Ogni nodo interno di tale albero rappresenta un confronto: il sottoalbero sinistro corrisponde a confronti del tipo \(a_i \le a_j\), mentre quello destro corrisponde a confronti del tipo \(a_i > a_j\). Si noti che sulle foglie sono presenti tutte le possibili permutazioni della sequenza di input: dunque un albero di decisione può presentare più di \(n!\) foglie, in quanto alcune permutazioni potrebbero comparire più volte, ma meno di \(2^h\) (con h, l'altezza dell'albero). La lunghezza del cammino semplice più lungo dalla radice di un albero di decisione ad una delle sue foglie rappresenta il numero di confronti che un determinato algoritmo di ordinamento deve svolgere nel caso peggiore: questo numero è equivalente all'altezza dell'albero stesso. Si introduce quindi il seguente teorema, che determina un limite inferiore sul tempo di esecuzione degli algoritmi di ordinamento per confronti:

\begin{theorem}
  Qualsiasi algoritmo di ordinamento per confronti richiede \(\Omega(n\,log\,n)\) confronti nel caso peggiore.
\end{theorem}

\noindent
da cui deriva anche:

\begin{theorem}
  Ogni albero di decisione di un algoritmo di ordinamento di n elementi ha altezza \(\Omega(n\,log\,n)\).
\end{theorem}

\vspace*{10pt}

Una volta determinato il limite inferiore del tempo di esecuzione degli algoritmi di ordinamento, si introduce qui l'algoritmo \textbf{counting sort}, che riordina un array di dimensione \(n\) in un tempo lineare \(\Theta(n)\). 
Tale algoritmo suppone che ciascuno degli elementi di input sia un numero intero compreso nell'intervallo da 0 a \(k\in \mathbb{N}\) e determina per ciascun di essi il numero di elementi minori; utilizza poi questa informazione per inserire l'elemento corrente direttamente nella giusta posizione nell'array di output. Ad esempio, se l'elemento \(x\) è più grande di 5 altri elementi, allora verrà inserito nella posizione 6 dell'array di output.

Nel codice di counting sort, si suppone che l'input sia un array \code{A[1..n]}, con \(n\)=\code{A.length}. Occorrono altri due array: l'array \code{B[1..n]}, che contiene l'output ordinato, e l'array \code{C[0..k]} fornisce la memoria temporanea di lavoro. In pseudocodice:

\lstinputlisting{../docs/algorithms/counting_sort.txt}

Dopo che il ciclo \code{for} in riga 3 inizializza a zero tutti gli elementi dell'array \code{C}, ogni elemento dell'input viene esaminato con il secondo ciclo \code{for} (in riga 5): se il valore di un elemento è \(i\), viene incrementato il valore di \code{C[i]}. Dunque, dopo la riga 6, l'array \code{C} contiene il numero di elementi uguagli ad \(i\), per ogni \(i=0,1,...,k\). Le righe 7 e 8 determinano, per ogni \(i=0,1,...,k\), quanti elementi di input sono minori o uguali a \(i\). Infine, il ciclo \code{for} di riga 9 inserisce l'elemento corrente nella posizione corretta all'interno dell'array di output \code{B}. 

Il primo ciclo \code{for} impiega un tempo di esecuzione \(\Theta(k)\), il secondo un tempo \(\Theta(n)\), il terzo un tempo \(\Theta(k)\) e, infine, il ciclo \code{for} di riga 9 impiega un tempo di esecuzione lineare \(\Theta(n)\). Dunque, il tempo di esecuzione totale dell'algoritmo \code{countingSort} è \(T(n)=\Theta(n+k)\). Di solito, questa procedura viene utilizzata quando \(h=\Theta(n)\), facendo si che il tempo totale di esecuzione sia un \(\Theta(n)\). 

L'algoritmo appena analizzato batte il limite inferiore di tempo \(\Omega(n\,log\,n)\) per gli algoritmi di ordinamento per confronti perchè non confronta nessun elemento dell'input, ma utilizza il valore di ogni elemento come indice di un array \code{C} di appoggio. 





  \chapter{Strutture Dati}
Gli insiemi manipolati dagli algoritmi, a differenza di quelli matematici, possono essere modificati inserendo o rimuovendo elementi. Questi insiemi sono detti dinamici e giocano un ruolo importante in informatica, perchè modellano le strutture utilizzate per memorizzare in modo ordinato i dati.

In una tipica implementazione di un insieme dinamico, ogni elemento è rappresentato da un oggetto, i cui attributi possono essere esaminati e manipolati a piacimento dagli algoritmi. In molte strutture dati, l'oggetto dispone di una chiave identificativa (spesso univoca, ma non necessariamente) e ovviamente di dati satelliti che si vogliono memorizzare ordinatamente in memoria. Oltre a questi due attributi, l'oggetto può anche contenere altri dati specifici per una determinata struttura dati, in modo da rendere più semplice e veloce la loro manipolazione.

Le tipiche operazioni che si possono svolgere sulle strutture dati sono suddivise in due categorie: le query (interrogazioni), che hanno il solo scopo di estrapolare informazioni dall'insieme dinamico, e le operazioni di modifica, che hanno il compito di modificare l'insieme. Di seguito sono elencate le istruzioni più comuni:

\begin{itemize}
  \item \code{search(S, k)}: è un'operazione di query che, dato un insieme \(S\) e un valore chiave \(k\), restituisce \code{NIL} se tale elemento non appartiene all'insieme.
  \item \code{insert(S, x)}: è un'operazione di modifica che inserisce all'interno dell'insieme \(S\) l'elemento puntato da \(x\).
  \item \code{delete(S, x)}: è un'operazione di modifica che, dato un puntatore \(x\) ad un elemento dell'insieme \(S\), rimuove \(x\) da \(S\).
  \item \code{minimum(S)}: è un'operazione di query che ritorna l'elemento dell'insieme \(S\) con la chiave più piccola.
  \item \code{maximum(S)}: è un'operazione di query che ritorna l'elemento dell'insieme \(S\) con la chiave più grande.
  \item \code{successor(S, x)}: è un'operazione di query che, dato un elemento \(x\) la cui chiave appartiene ad un insieme totalmente ordinato \(S\), restituisce un puntatore all'elemento successivo più grande di \(S\) oppure \code{NIL} se \(x\) è il più grande degli elementi.
  \item \code{predecessor(S, x)}: è un'operazione di query che, dato un elemento \(x\) la cui chiave appartiene ad un insieme totalmente ordinato \(S\), restituisce un puntatore all'elemento precedente più piccolo di \(S\) oppure \code{NIL} se \(x\) è il più piccolo degli elementi.
\end{itemize}

\section{Stack}
Gli \textbf{stack} sono insiemi dinamici dove l'elemento da rimuovere tramite l'operazione \code{delete} è predeterminato. In questa struttura dati, l'elemento cancellato è quello inserito per ultimo, secondo la politica LIFO (Last In, First Out). Nello specifico, le operazioni di \code{insert} e \code{delete} prendono rispettivamente il nome di \code{push} e \code{pop} \footnote{Questa operazione non prende nessun argomento, in quanto l'elemento da eliminare è predeterminato}: la prima inserisce in cima alla pila l'elemento passato come argomento, mentre la seconda operazione elimina l'unico elemento accessibile dalla pila, ovvero la cima. 

Questa struttura dati può essere implementata tramite un array con un massimo di \(n\) elementi \code{S[1..n]}, che presenta lo specifico atrtibuto \code{S.top}, ovvero l'indice tramite cui accedere all'ultimo elemento inserito. L'array è dunque composto dagli elementi \code{S[1..S.top]}, dove \code{S[1]} rappresenta l'elemento in fondo alla pila, mentre \code{S[S.top]} rappresenta l'elemento in cima. Ovviamente, se \code{S.top} = 0, si dice che la pila è vuota, in quanto non contiene nessun elemento. In questo caso, se si tenta di estrarre un elemento dallo stack, si ottiene un errore di underflow dello stack, mentre se si cerca di inserire un elemento sulla pila piena (che conta quindi di \(n\) elementi), si ottiene un errore di overflow dello stack.

Le operazioni dello stack possono essere implementate molto semplicemente in pseudocodifica come segue:

\begin{lstlisting}
push(S, x):
  if S.top = S.length:
    error "overflow"
  else:
    S.top := S.top + 1
    S[S.top] := x
\end{lstlisting}

\begin{lstlisting}
pop(S):
  if iS.top = 0:
    error "underflow"
  else:
    S.top := S.top - 1
    return S[S.top + 1]
\end{lstlisting}

Si noti come, in questo caso, l'operazione di \code{pop} ritorna l'elemento appena eliminato dallo stack. Entrambe le procedure vengono eseguite in tempo costante \(\Theta(1)\).

\section{Queue}
Le \textbf{code} sono insiemi dinamici dove l'elemento da rimuovere tramite l'operazione \code{delete} è predeterminato. In questa struttura dati, l'elemento cancellato è quello inserito per primo, secondo la politica FIFO (First In First Out). Nello specifico, la coda presenta un inizio detto \code{head} e una fine detta \code{tail}, e le operazioni di \code{insert} e \code{delete} prendono rispettivamente il nome di \code{enqueue} e \code{dequeue} \footnote{Anche in questo caso, questa operazione non prende nessun argomento, in quanto l'elemento da eliminare è predeterminato}: la prima inserisce in fondo alla fila l'elemento passato come argomento, mentre la seconda operazione elimina il primo elemento della fila. 

Questa struttura dati può essere implementata tramite un array di \(n\) elementi \code{Q[1..n]}, che contiene un massimo di \(n-1\) elementi, per ragioni che verranno chiarite in seguito. L'attributo \code{Q.head} punta all'inizio della coda, mentre l'attributo \code{Q.tail} punta alla posizione in cui l'ultimo elemento che dovrà essere inserito prenderà posto (ovvero alla posiziove vuota successiva all'ultimo elemento della coda). Gli elementi della coda, quindi, occupano le posizioni \code{Q.head}, \code{Q.head + 1}, ..., \code{Q.tail - 1}. Alla fine dell'array la posizione 1 della queue segue immediatamete la posizione \(n\) secondo un ordine circolare. Se \code{Q.head = Q.tail} allora la coda è vuota. All'inizio le posizioni \code{Q.head} e \code{Q.tail} combaciano e sono entrambe inizializzate al valore 1.

Come per gli stack, se la coda è vuota, il tentativo di rimuovere un elemento provoca un errore di underflow, mentre se \code{Q.head = Q.tail + 1} la coda è piena e il tentativo di inserire un nuovo elemento provoca un errore di overflow. 

Le operazioni della queue possono essere implementate molto semplicemente in pseudocodifica come segue:

\begin{lstlisting}
enqueue(Q, x):
  Q[Q.tail] := x
  if Q.tail = Q.length:
    Q.tail := 1
  else:
    Q.tail := Q.tail + 1
\end{lstlisting}

\begin{lstlisting}
dequeue(Q, x):
  x := Q[Q.head]
  if Q.head = Q.length:
    Q.head := 1
  else:
    Q.head := Q.head + 1
  return x
\end{lstlisting}

Entrambe le procedure vengono eseguite in un tempo costante \(\Theta(1)\).

\section{Linked List}
Una \textbf{lista concatenata} è una struttura dati i cui oggetti sono disposti in ordine lineare, determinato da un puntatore in ogni oggetto. Una lista doppiamente concatenata è una lista in cui ogni oggetto presenta, oltre ad una chiave \code{key}, anche un puntatore all'elemento successivo \code{next} e un puntatore a quello precedente \code{prev}. Se \code{x.prev = NIL}, allora l'elemento \(x\) è il primo elemento della lista e si dice essere la testa (o head) della lista. Se, invece, \code{x.next = NIL}, allora \(x\) è l'ultimo elemento della lista e si dice essere la coda (o tail) della lista. Intuitivamente, l'attributo \code{L.head} punta alla testa della lista, che sarà vuota se \code{L.head = NIL}.

Questa struttura dati può presentare varie forme: può essere doppiamente concatenata o singolarmente concatenata, oppure può essere circolare o non. Una lista si dice singolarmente concatenata se i suoi oggetti non sono dotati di puntatore all'elemento precedente, mentre si dicono circolari se l'ultimo elemento possiede un puntatore alla testa della lista, che a sua volta possiede un puntatore alla coda se la lista è circolare. Una lista concatenata può anche essere ordinata o non: si dice ordinata quando la disposizione lineare degli elementi corrisponde con la disposizione crescente delle chiavi degli elementi e, in tal caso, la testa della lista conterrà l'elemento minimo, mentre la coda l'elemento massimo. 

Nel seguito si fa riferimento a liste non ordinate doppiamente concatenate per lo sviluppo degli algoritmi che le manipolano. 

\vspace{10pt}

La prima procedura che sia analizza è \code{listSearch(L, k)}, che trova il primo elemento con la chiave \(k\) nella lista \(L\), restituendo un puntatore a tale oggetto. Se nessun oggetto con chiave \(k\) è presente nella lista, allora viene restituito il valore \code{NIL}. In pseudocodifica:

\begin{lstlisting}
listSearch(L, k):
  x := L.head
  while x != NIL and x.key != key:
    x := x.next
  return x  
\end{lstlisting}

Si noti quindi che l'algoritmo \code{listSearch} cerca l'elemento di chiave \(k\) tramite una ricerca lineare sulla list \(L\) di \(n\) elementi. Dunque, l'algoritmo impiega un tempo \(\Theta(n)\) nel caso peggiore, in quanto potrebbe essere necessario scorrere l'intera lista.

\vspace{10pt}

La seconda procedura analizzata è \code{listInsert(L, x)} che inserisce l'elemento x di attributo key (già inizializzato) in testa alla lista. In pseudocodifica:

\begin{lstlisting}
listInsert(L, x):
  x.next := L.head
  if L.head != NIL:
    L.head.prev := x
  L.head := x
  x.prev := NIL 
\end{lstlisting}

Questa procedura impiega un tempo costante \(\Theta(1)\) per la sua esecuzione.

\vspace{10pt}

L'ultima procedura analizzata per le linked list è \code{listDelete(L, x)}, che rimuove l'elemento \(x\) dalla lista \(L\). Per poter eliminare tale elemento è prima necessario chiamare la funzione \code{listSearch} per ottenere il puntatore all'elemento desiderato. In pseudocodifica:

\begin{lstlisting}
listDelete(L, x):
  if x.prev != NIL:
    x.prev.next := x.next
  else:
    L.head := x.next
  if x.next != NIL:
    x.next.prev := x.prev
\end{lstlisting}

Anche questa procedura impiega un tempo di esecuzione \(\Theta(1)\).

\section{Hash Table}
\subsection{Indirizzamento diretto}
Prima di procedere con l'introduzione alle tavole di hash, è prima necessario introdurre il concetto di indirizzamento diretto, una tecnica molto efficiente nel caso in cui l'insieme da cui vengono acquisite le chiavi, detto insieme universo \(U = \{0,1,2...,m-1\}\), è un insieme ragionevolmente piccolo. Si suppone, inoltre, che due elementi distinti non possano avere chiavi coincidenti. Per rappresentare un tale insieme dinamico si utilizza un array, oppure una tavola ad indirizzamento diretto, indicata con \(T[0..m-1]\), dove ogni cella \(k\) di tale tabella punta all'elemento dell'insieme di chiave \(k\). Se la \(k\)-esima cella non contiene nessun elemento, viene inizializzata con il valore \code{NIL}.

Le operazioni di dizionario sono semplici da implementare in pseudocodifica e impiegano tutte un tempo costante \(O(1)\) (nel caso peggiore):

\begin{lstlisting}
directAddressSearch(T, k):
  return T[k]
\end{lstlisting}

\begin{lstlisting}
directAddressInsert(T, x):
  T[x.key] := x 
\end{lstlisting}

\begin{lstlisting}
directAddressDelete(T, x):
  T[x.key] := NIL
\end{lstlisting}

In alcune implementazioni è possibile memorizzare l'elemento di chaive \(k\) direttamente all'interno della tabella, anzichè in un oggetto esterno, risparmiando spazio in memoria. 

\subsection{Introduzione alle Tavole di Hash}

La difficoltà nell'implementazione di una tale struttura dati è evidente: l'insieme universo \(U\), nella maggior parte dei casi, è troppo grande per essere memorizzato in una tavola \(T\) di dimensione \(|U|\). Inoltre, l'insieme \(K\) delle chiavi effettivamente memorizzate è molto più piccolo dell'inisieme \(U\) delle chiavi disponibili e, dunque, la maggior parte dello spazio allocato per la tavola \(T\) non verrebbe mai utilizzato. A questo proposito si introduce una nuova struttura dati, detta \textbf{tavola di hash}, che utilizza una memoria proporzionale al numero delle chiavi effettivamente memorizzate nel dizionario, riducendo lo spreco di memoria. 

Quando l'insieme \(K\) delle chiavi memorizzate in un dizionario è molto più piccolo dell'universo \(U\) di tutte le chiavi possibili, utilizzando una tavola di hash si può ridurre lo spazio richiesto fino a \(\Theta(|K|)\), rdiucendo però l'efficienza temporale a \(O(1)\) nel caso medio (anzichè pessimo). Nella tabella di hash, l'elemento di chiave \(k\) non viene memorizzato direttamente nella cella \(k\), ma viene utilizzata una cosiddetta funzione di hash \(h(k)\) che calcola l'indice \(k\) della cella. La funzione di hash \(h(k)\) associa ad ogni chiave dell'universo \(U\) una specifica chiave della tavola di hash \(T[0..m-1]\). Formalmente:

\(h(k):U\to \{0,1,..., m-1\},\;\;\; m<<|U|\)

Si dice che \(h(k)\) è il valore hash della chiave \(k\).

\subsection{Hashing Concatenato}

Il problema principale con la tecnica di indirizzamento appena analizzata è che, riducendo l'intervallo degli indici da \(|U|\) ad \(m<<|U|\), è molto probabile che due chiavi vengano mappate nella stessa cella: in tal caso si dice che avviene una collisione. Per evitare un evento simile è possibile, in prima analisi, implementare una funzione di hash totalmente deterministica il più randomica possibile, in modo da minimizzare le collisioni. Si dimostra però che un evento di collisione è impossibile da evitare in quanto \(|U| > m\) e quindi, dopo l'\(m\)-esima chiamata alla funzione di hash, avverà sicuramente una collisione. 

Si rende necessario, dunque, implementare un meccanismo che gestisca tali eventi. Nello specifico, la tecnica più utilizzata è il concatenamento (o chaining), tramite cui, tutti gli elementi associati ad una stessa cella \(k\) sono posti in una lista concatenata. La cella \(k\), in questo caso, punta al nodo di testa della lista che contiene gli elementi mappati in tale cella, oppure ha valore \code{NIL}, nel caso in cui la cella non contenga nessun elemento. 

Le operazioni di dizionario su una tavola di hash \(T\) sono facili da implementare in pseudocodifica nel caso di gestione delle collisioni tramite concatenamento:

\begin{lstlisting}
chainedHashInsert(T, x):
  listInsert(T[h(x.key)], x)
\end{lstlisting}

\begin{lstlisting}
chainedHashSearch(T, x):
  listSearch(T[h(x.key)], x)
\end{lstlisting}

\begin{lstlisting}
chainedHashDelete(T, x):
  listDelete(T[h(x.key)])
\end{lstlisting}

In cui le procedure \code{listInsert}, \code{listSearch} e \code{listDelete}, sono le stesse analizzate nella sezione delle liste concatenate e hanno il compito, rispettivamente, di inserire in testa alla lista un nodo, cercare un nodo nella lista ed eliminare un nodo dalla lista. 

Si passa ora all'analisi del tempo di esecuzione di tali procedure: nel caso peggiore, l'inserimento in lista di un nodo è \(O(1)\), la ricerca avviene in tempo proporzionale alla lunghezza della lista, quindi \(O(n)\), mentre l'eliminazione di un nodo dalla lista avviene, sempre nel caso peggiore, in tempo \(O(1)\) se la lista è doppiamente concatenata. 

Si noti che la funzione \code{chainedHashDelete} prende come input un elemento \(x\), non la sua chiave \(k\), quindi non occorre cercare prima l'elemento \(x\). Se la tavola di hash supporta la cancellazione, allora le sue liste dovrebbero essere doppiamente concatenate in modo che la cancellazione di un elemento sia più rapida. Se le liste fossero singolarmente concatenate, per cancellare l'elemento \(x\), si dovrebbe prima trovare \(x\) nella lista \(T[h(x.key)]\) in modo da poter aggiornare l'attributo \code{next} dell'elemento precedente in lista, assegnandogli il valore \code{NIL}.

\subsection{Analisi della Funzione di Hash}
Data una tavola di hash \(T\), che conta \(m\) celle in cui sono memorizzati \(n\) elementi, si definisce il fattore di carico \(\alpha\) della tavola \(T\) come il rapporto \(n/m\), ossia il numero medio di elementi memorizzati in una lista. Il caso peggiore nell'hashing si verifica quando tutte le \(n\) chiavi sono associate alla stessa cella, creando una lista di lunghezza \(n\). Il tempo di esecuzione della ricerca diventa quindi \(\Theta(n)\) a cui si aggiunge il tempo di esecuzione della funzione di hashing. Ovviamente, un caso del genere è molto improbabile nel caso in cui la funzione di hash sia ben progettata. Per il momento si suppone che ogni elemento ha uguale probabilità di essere mappato in una qualsiasi delle \(m\) celle, indipendentemente dalle celle in cui sono stati mappati gli altri elementi. Tale ipotesi viene definita hashing uniforme semplice. Per ogni \(j=0,1,..., m-1\), si indica con \(n_j\) la lunghezza della lista \(T[j]\), ottenendo quindi il numero di elementi totali memorizzati in tabella è \(n=n_0+n_1+...+n_{m-1}\). Il valore atteso di ogni \(n_j\) sarà \(E[n_j] = \alpha = n/m\), quindi il tempo medio per la ricerca di un elemento di chiave \(k\) non presente nella lista (caso pessimo) è \(\Theta(1+\alpha)\), che si dimostra essere anche il tempo di ricerca dello stesso elemento, questa volta presente tabella. Nella pratica, se il numero di celle nella tavola di hash è almeno proporzionale al numero di elementi della tavola, si ottiene che \(n=O(m)\) e quindi \(\alpha = n/m = O(m)/m = O(1)\). Pertanto, la ricerca di un elemento della tavola richiede un tempo costante.Ogni operazione di dizionario può essere svolta, in media, in un tempo \(O(1)\).

\subsection{Funzione di Hash}
Per progettare una buona tabella di hash è necessario implementare una funzione di hash che sia altamente efficiente. A questo proposito si introducono tre possibili schemi di implementazione: hashing per divisione (uristico), hashing per moltiplicazione (euristico) e hashing universale (aleatorio, non analizzato in questa sezione). In generale, una buona funzione di hashing deve soddisfare approssimativamente la condizione di hashing uniforme semplice: ogni chiave deve avere la stessa probabilità di essere mappata in una qualsiasi cella della tabella. Di solito non è possibile verificare questa condizione, in quanto non è nota la distribuzione delle probabilità secondo cui vengono estratte le chiavi. Quindi, nella pratica, spesso si utilizzano delle tecniche euristiche per la realizzazione di tali funzioni.

La maggior parte delle funzioni di hashing suppone che l'universo delle chiavi sia l'insieme dei numeri naturali \(\mathbb{N}\) e quindi, se nella struttura dati progettata, la chiave non è un numero naturale ma, ad esempio, una stringa, è necessario studiare un metodo di conversione: nei calcolatori questo metodo è tipicamente già implementato, in quanto ogni informazione analogica viene convertita in una stringa binaria di bit (appartenente ad \(\mathbb{N}\)). Nello studio dei tre metodi di hashing si suppone che le chiavi siano numeri naturali.

\vspace{10pt}

\textbf{Metodo della Divisione}. Quando si applica il metodo della divisione per creare una funzione di hash, una chiave \(k\) viene associata a una delle \(m\) celle prendeno il resto della divisione fra \(k\) ed \(m\). Formalmente, la funzione di hash è così definita:

\(h(k)=k\; mod\; m\)

\noindent Il vantaggio principale di questo metodo è che si può implementare molto rapidamente e richiede un tempo di esecuzione costante. 

Quando si utilizza il metodo della divisione, si cerca di evitare alcuni valori di \(m\). Nello specifico si evitano le potenze di 2, in quanto se \(m=2^p\), allora \(h(k)\) rappresenta proprio i \(p\) bit meno significativi di \(k\): infatti, sarebbe più corretto far dipendere la funzione di hash da tutti i bit della chiave. Inoltre, si evita di scegliere \(m+2^p-1\) quando \(k\) è una stringa di caratteri interpretata in base \(2^p\), in quanto la permutazione dei caratteri di \(k\) non cambia il suo valore di hash. Una buona scelta di \(m\), invece, potrebbe essere un numero primo non troppo vicino a una potenza esatta di 2.

\vspace{10pt}

\textbf{Metodo della Moltiplicazione}. Il metodo della moltiplicazione per la creazione di funzioni di hash consiste in due passi. Nel primo passaggio si moltiplica la chiave \(k\) per una costante \(A\), tale che \(0<A<1\), per poi estrarre la parte frazionaria del numero appena ottenuto. Nel secondo passaggio si moltiplica questo valore per \(m\) e si prende la parte intera inferiore del risultato. Formalmente, la funzione di hash è così definita:

\(h(k) = \lfloor m\,(k\cdot A\;mod\;1)\rfloor\)

\noindent dove \(k\cdot A \; mod \;1 \) rappresenta la parte frazionaria di \(kA\), ovvero \(kA-\lfloor kA\rfloor\). 

Il vantaggio principale di questo metodo è che il valore di \(m\) non è critico. Tipicamente si sceglie un valore di \(m\) tale per cui sia una potenza di 2, in modo da rendere più semplice implementare tale funzione in un calcolatore reale. Si upponga, infatti, che la dimensione di una parola nel calcolatore sia \(w\) bit e che il numero \(k\) sia contenuto in una sola parola. Si scelga poi un valore di \(A\) che sia una frazione nella forma \(s/2^w\), con \(s\) intero nell'intervallo \(0<s<2^w\). A questo punto, si moltiplica \(k\) per \(s=A\cdot 2^w\): il risultato sarà un numero di \(2w\) bit \(r_1 2^w+r_0\), in cui \(r_1\) rappresenta la parte più significativa del prodotto ed \(r_0\) la parte meno significativa. Il valore hash desiderato di \(p\) bit è formato dai \(p\) bit più significativi di \(r_0\). Sebbene queste operazioni funzionino con qualsiasi valore della costante \(A\), la scelta spesso adoperata è \(A \approx (\sqrt{5}-1)/2 \approx 0.6180339887...\)

\subsection{Indirizzamento Aperto}
UN altro metodo per evitare le collisioni è tramite l'indirizzamento aperto, in cui tutti gli elementi sono memorizzati nella tavola hash, ovvero ogni cella contiene un elemento dell'insieme dinamico o la costante \code{NIL} se non contiene nessun elemento. Quando si cerca un elemento, si esamina sistematicamente la tabella fino a quando non si trova l'elemento desiderato, oppure finchè non ci sono più elementi da controllare (l'elemento non è nell'array). Nell'indirizzamento aperto, a differenza degli altri metodi, la tavola hash può riempirsi fino a quando non può più fisicamente contenere altri elementi. Una conseguenza di questo design è che il fattore di carico \(\alpha\) non supera mai il valore 1. Il vantaggio di questo metodo sta nel fatto che elimina completamente l'utilizzo dei puntatori, in quanto calcola la sequenza delle celle da esaminare, e libera quindi una notevole quantità di memoria, utilizzata per incrementare la capacità della tabella, riducendo il rischio di collisioni. 

Per effettuare un insierimento mediante il metodo dell'indirizzamento aperto, si esamina in successione le posizioni della tavola hash fino a che non si trova una cella libera in cui inserire la chiave. L'efficienza di questo metodo consiste nel calcolare una nuova sequenza di accesso alla tabella in base alla chiave dell'oggetto da inserire, anzichè seguire sempre lo stesso cammino di accessi, che impiegherebbe un tempo di esecuzione \(\Theta(n)\). Per determinare quali celle esaminare durante la fase di ispezione, si estende la funzione di hash in modo da includere l'ordine di ispezione (a partire da 0), come secondo input. Formalmente, la funzione di hash modificata è definita come segue:

\(h(k,i):U\times \{0,1,...,m-1\} \to \{0,1,...,m-1\}\)

\noindent Si richiede, inoltre, che per ogni chiave \(k\) la sequenza di ispezione \(<h(k,0), h(k,1),..., h(k,m-1)>\) sia una permutazione della sequenza \(<0,1,...,m-1>\), in modo tale che ogni cella della tavola possa essere considerata come possibile cella in cui inserire una nuova chiave. In pseudocodifica:

\begin{lstlisting}
hashInsert(T, k):
  i := 0
  repeat
    j := h(k, i)
    if T[j] = NIL or T[j] = DELETED:
      T[j] := k
      return j
    else:
      i := i + 1
  until i = m
  error "hash table overflow"
\end{lstlisting}

Questa procedura prende in input una tavola di hash \(T\) e una chiave \(k\) da inserire in tabella, e ritorna l'indice della cella in cui è stato inserito l'elemento, oppure \code{error} di overflow se non è presente nessuna cella libera. Si noti nella condizione in riga 5, che viene verificato se la cella \(T[k]\) contiene il valore \code{DELETED}: questo valore speciale verrà ripreso e analizzato più avanti, con la procedura di eliminazione delle chiavi dalla tabella.

\vspace{10pt}

Per la ricerca di una determinata chiave \(k\) esamina la stessa sequenza di celle che ha esaminato l'algoritmo di inserimento quando ha inserito l'elemento di chiave \(k\). Dunque, la procedura di ricerca potrebbe terminare la propria esecuzione senza successo quando trova una cella vuota, in quanto la chiave \(k\) sarebbe stata inserita in quella posizione (si presuppone che le chiavi non possano essere eliminate dalla tavola). In pseudocodifica:

\begin{lstlisting}
hashSearch(T, k):
  i := 0
  repeat
    j := h(k, i)
    if T[j] = k:
      return j
    i := i + 1
  until T[j] = NIL or i = m
  return NIL
\end{lstlisting}

Questa procedura, come la precedente, prende in input una tavola di hash \(T\) e una chiave \(k\) da ricercare in tabella, e ritorna l'indice della cella in cui è stato trovato l'elemento, oppure \code{NIL} se non è presente nessuna cella che contiene l'elemento oppure se nella sequenza di ricerca si trova una cella libera (per le ragioni precedentemente analizzate).

\vspace{10pt}

La procedura di cancellazione di una chiave dalla tavola di hash ad indirizzamento aperto è un'operazione molto complessa, in quanto non è possibile semplicemente cancellare il contenuto della cella \(i\), assegnandone il valore \code{NIL}, in quanto sarebbe impossibile ricercare qualsiasi elemento in tabella per come è stata definita la procedura di ricerca. Dunque, si rende necessario marcare la cella il cui contenuto è stato eliminato con il valore speciale \code{DELETED}, anzichè \code{NIL}. È per questo motivo che in riga 5 della procedura di inserimento viene anche verificato se la cella ispezionata contanga il valore \code{DELETED}. Si noti inoltre che, utilizzando questa nuova notazione, il tempo di esecuzione della procedura di ricerca non dipende più dal fattore di carico \(\alpha\). Per questo motivo, nella pratica si preferisce spesso utilizzare il metodo della concatenazione quando è necessario che la tabella di hash supporti l'operazione di cancellazione delle chiavi. 

\vspace{10pt}

Nell'analisi delle tabelle hash con indirizzamento aperto si ipotizza hashing uniforme: si suppone, infatti, che ogni chiave abbia la stessa probabilità di avere come sequenza di ispezione una delle \(m!\) permutazioni di \(<0,1,...,m-1>\). L'hashing uniforme estende il concetto di hashing uniforme semplice, impiegato più volte precedentemente, al caso in cui la funzione di hash produce, non un singolo numero, ma un'intera sequenza di ispezione. Nella pratica non è possibile ottenere una funzione di hash uniforme, ma si utilizzano delle approssimazioni accettabili. Si esaminano nel seguito tre tecniche utilizzate per calcolare le sequenze di ispezione richieste nell'indirizzamento aperto: ispezione lineare, ispezione quadratica e doppio hashing. Tali tecniche grantiscono che la sequenza \(<h(k,0), h(k,1),..., h(k,m-1)>\) sia una permutazione di \(<0,1,...,m-1>\) per ogni chiave \(k\), ma nessuna di loro può garantire l'ipotesi di hashing uniforme, in quanto nessuna di esse è in grado di generare più di \(m^2\) sequenze di ispezioni differenti (invece delle \(m!\) sequenze richieste). 

\vspace{10pt}

\textbf{Ispezione Lineare}. Data una funzione di hash ordinaria \(h':U\to \{0,1,...,m-1\}\), detta funzione di hash ausiliaria, il metodo di ispezione lineare utilizza la funzione di hash

\(h(k,i)=(h'(k)+i)\;mod\;m\)

\noindent per \(i=0,1,...,m-1\). Data la chiave \(k\), la prima cella esaminata è \(T[h'(k)]\), che è la cella data dalla funzione di hash ausiliaria, la seconda cella è \(T[h'(k)+1]\) e così via fino alla cella \(T[m-1]\). Poi, l'ispezione riprende dalle celle \(T[0], T[1],...,T[h'(k)-1]\). Poichè la prima cella ispezionata determina l'intera sequenza di ispezioni, ci sono soltanto \(m\) sequenze di ispezione distinte. Questa tecnica è facile da implementare ma presenta un problema noto come addensamento primario: si formano lunghe file di celle occupate che aumentano il tempo medio di ricerca. Tale fenomeno si presenta perchè una cella vuota preceduta da \(i\) celle piene ha probabilità \((i+1)/m\) di essere la prossima ad essere occupata e le lunghe file di celle occupate tendono, dunque, a diventare sempre più lunghe.

\vspace{10pt}

\textbf{Ispezione Quadratica}. Data la funzione di hash ausiliaria \(h':U\to \{0,1,...,m-1\}\), il metodo di ispezione qudratica utilizza la funzione di hash

\(h(k,i)=(h'(k)+c_1i+c_2i^2)\;mod\;m\)

\noindent per \(i=0,1,...,m-1\), con \(c_1, c_2 \neq 0 \) costanti ausiliarie. Data la chiave \(k\), la prima cella esaminata è \(T[h'(k)]\), mentre le successive posizioni esaminate sono distanti dalle precedenti di quantità che dipendono in modo quadratico dal numero d'ordine di ispezione \(i\). Questa tecnica funziona meglio della precedente, ma i valori \(c_1, c_2\) ed \(m\) non possono essere scelti in maniera arbitraria, ma devono essere tali per cui si possa percorrere l'intera tabella. Inolte, se due chiavi hanno la stessa posizione iniziale di ispezione, allora le due sequenze di ispezione saranno identiche portando al cosiddetto addensamento secondario.

\vspace{10pt}

\textbf{Doppio Hashing}. Date la funzioni di hash ausiliarie \(h_1\) ed \(h_2\), il metodo di ispezione lineare utilizza la funzione di hash

\(h(k,1) = (h_1(k)+ih_2(k))\;mod\;m\)

\noindent per \(i=0,1,...,m-1\). Il doppio hashing è il metodo migliore disponibile per l'indirizzamento aperto, in quanto le permutazioni prodotte hanno molte caratteristiche comuni con le permutazioni casuali. Data la chiave \(k\), la prima cella esaminata è \(T[h_1(k)]\), mentre le successive posizioni sono distanziate dalle precedenti di quantità \(h_2(k)\;mod\;m\). Differentemente dai precedenti, il metodo del doppio hashing produce sequenze che dipendono in due modi dalla chiave \(k\), in quanto possono variare sia la posizione iniziale della sequenza di ispezione, sia la distanza fra due posizioni successive. 

Inoltre, il valore \(h_2(k)\) deve essere relativamente primo con la dimensione \(m\) della tavola hash perchè venga ispezionata l'intera tabella. Un modo pratico per garantire tale condizione è scegliere \(m\) potenza di due e definire \(h_2\) in modo che produca sempre un numero dispari. Un altro modo è scegliere \(m\) primo e definire \(h_2\) in modo che generi sempre un numero intero positivo minore di \(m\). In questo contesto, il doppio hashing è migliore delle precedenti tecniche in quanto utilizza \(\Theta(m^2)\) sequenze di ispezione, anzichè \(\Theta(m)\), perchè ogni possibile coppia di \(h_1, h_2\) produce una distinta sequenza di ispezione.





% \section{Alberi Binari}
% \section{Alberi RB}
% \section{Grafi}

  \chapter{Grammatiche}
  Come visto in precedenza, spesso gli automi vengono utilizzati come modelli per il riconoscimento di linguaggi. Gli automi sono quindi uno strumento formale per la descrizione e la definizione di un determinato linguaggio, costituito dall'insieme delle stringhe accettate dall'automa stesso.

  Un altro formalismo utilizzato per la definizione di linguaggio sono le cosiddette grammatiche formali, che definiscono un linguaggio fornendo il procedimento mediante cui si ottengono le stringhe appartenenti al linguaggio stesso. 
  
  \section{Introduzione}

  Una grammatica formale è un insieme di regole per costruire stringhe appartenenti ad un determinato linguaggio, attraverso il meccanismo di riscrittura, che consiste in un insieme di tecniche che determinano come sostituire le parti di una formula con parti più semplificate. In generale, un meccanismo di riscrittura consiste in un insieme di regole linguistiche, di cui una descrive l'oggetto principale come una sequenza di componenti. Ogni componente si può raffinare da elementi via via sempre più dettagliati, fino ad ottenere una sequenza di componenti elementari.

  Una grammatica non è altro che un meccanismo linguistico, composto dall'oggetto principale, detto anche simbolo iniziale, da un insieme di componenti, a loro volta da sostituire durante il processo di derivazione, detti anche simboli non terminaliun insieme di elementi di base, detti anche simboli elementari, e da un insieme di regole di raffinamento o sostituzioni, chiamate produzioni.

  \begin{definition} \label{definizione grammatica}
    Una grammatica G è una tupla di 4 elementi \(G = <V_T, V_N, P, S>\), dove:
    \begin{itemize}
      \item \(V_T\) è un insieme di simboli terminali (solitamente indicati con lettere minuscole), detto anche alfabeto terminale;
      \item \(V_N\) è un insieme di simboli non terminali (solitamente indicati con lettere maiuscole), tali che \(V_T \cap V_N = \emptyset\), detto anche alfabeto non terminale; V indica \(V_T\cup V_N\);
      \item P è un insieme finito di \(V_N^+\times V^*\), detto anche insieme delle produzioni di G. Un elemento \(p=<\alpha, \beta> \in P\) si indica con \(\alpha\to\beta\), in cui \(\alpha\) è la parte sinistra di p, mentre \(\beta\) è la parte destra di p;
      \item S è un elemento particolare di \(V_N\), detto assioma o simbolo iniziale. 
    \end{itemize}
  \end{definition}

  Quindi, un elemento che deve essere ancora raffinato è un simbolo non terminale, un elemento di base è un simbolo terminale, le componenti di un oggetto possono essere sia simboli terminali che non terminali, mentre una produzione corrisponde ad una regola di raffinamento.

  \begin{definition}
    Data una grammatica G, si definisce su \(V^*\) la relazione binaria di derivazione immediata, indicata con il simbolo \(\Rightarrow\) da \(\alpha\) a \(\beta\). Tale relazione sussiste se e solo se \(\alpha=\alpha_1\gamma\alpha_2, \beta=\alpha_1\delta\alpha_2\), con \(\alpha_1, \alpha_2, \delta \in V^*, \gamma \in V_N^+, \gamma\to\delta\in P\).
  \end{definition}

  Data la definizione di derivazione immediata, si può anche definire la chiusura riflessiva e transitiva, indicata con il simbolo \(\Rightarrow^*\), che opera su una serie di stringhe (di simboli elementari o non elementari), anzichè che su una sola stringa.

  Date le precedenti definizioni, si può ora definire il linguaggio generato da una grammatica, tramite la seguente definizione:
  \begin{definition}
    Data una grammatica G, il linguaggio L(G) generato da G è definito come:

    \(L(G)=\{x\;|\;S\Rightarrow^*x, x\in V_T^*\}\)
  \end{definition}

  Quindi il linguaggio generato da una grammatica è costituito da tutte e sole le stringhe di simboli terminali, derivati a partire dall'assioma \(S\), applicando un numero qualsiasi di sostituzioni.

  \section{Classificazione}
  Una volta definite cosa siano le grammatiche, è possibile classificarle in base alle loro proprietà e in base alla forma ammessa per le produzioni. Tale classificazione viene anche detta Gerarchia di Chomsky, tramite cui si dividono le grammatiche in quattro categorie:
  \begin{itemize}
    \item Grammatiche di tipo 0 (non ristrette): sono grammatiche definite come nella \ref{definizione grammatica}, ovvero grammatiche che non possiedono nessuna restrizione nel tipo di produzione;
    \item Grammatiche di tipo 1 (sensibili al contesto): sono grammatiche a cui si introduce il vincolo per cui le produzioni possono essere solo nella forma \(\alpha A\beta\to\alpha\gamma\beta\), dove \(\alpha, \beta, \gamma\in V\) e \(A\in V_N\), con \(\gamma\neq\varepsilon\); inoltre, la derivazione \(S\to\varepsilon\) è consentita solo se \(S\) non appare a destra in nessuna regola di derivazione;
    \item Grammatiche di tipo 2 (non contestuali): sono grammatiche a cui si introduce il vincolo per cui ad ogni produzione \(\alpha\to\beta \in P\) si verifica che \(\;|\;\alpha\;|\; =1\) (quindi \(\alpha \in V_N\)) e \(\beta\in V^*\);
    \item Grammatiche di tipo 3 (regolari): sono grammatiche a cui si introduce il vincolo per cui ad ogni produzione \(\alpha\to\beta \in P\) si verifica che \(\;|\;\alpha\;|\; = 1\) (quindi \(\alpha \in V_N\)) e che \(\beta\) sia in una sola delle seguenti forme: \(aB\), \(Ba\), \(a\) oppure \(\varepsilon\), con \(a\in V_T\) e \(B\in V_N\); inoltre, la derivazione \(S\to\varepsilon\) è consentita solo se \(S\) non appare a destra in nessuna regola di derivazione;
  \end{itemize} 

  \begin{figure}[!h]
    \begin{center}    
      \begin{tikzpicture}
        \foreach \X [count=\Y starting from 0.75] 
        in {tipo 3, tipo 2, tipo 1, tipo 0} {
          \draw (-\Y,-\Y/2) circle ({1.5*\Y} and \Y);
          \node at (1-2*\Y,-1.1*\Y) {\X}; 
        }
      \end{tikzpicture}
    \end{center}
    \caption{Gerarchia di Chomsky}    
  \end{figure}

  \section{Grammatiche e Automi}
  Studiando le grammatiche e i linguaggi da esse generate, si può osservare una certa corrispondenza con gli automi analizzati nei capitoli precedenti. Si introducono qui alcuni teoremi che mettono in luce la correlazione esistente fra automi e grammatiche.

  \begin{theorem}
    Dato un FSA A, è possibile costruire una grammatica regolare (di tipo 3) G ad esso equivalente, ossia in grado di riconoscere lo stesso linguaggio riconosciuto da A, e viceversa. Dunque, le grammatiche regolari e gli automi a stati finiti sono modelli differenti per descrivere la stessa classe di linguaggi. 
  \end{theorem}
  Dato un FSA \(A=<I, \delta, q_0, F>\), si può costruire una grammatica \(G=<V_N,V_T, P, S>\) regolare, tale che:
  \begin{itemize}
    \item \(V_N=Q\);
    \item \(V_T=I\);
    \item \(S=q_0\);
    \item \(\forall B\to bC\iff C\in\delta(B,b)\);
    \item \(\forall B\to\varepsilon, B\in F\)
  \end{itemize}
  Data una grammatica \(G=<V_N, V_T, P, S>\) regolare, si può costruire un FSA \(A=<I, \delta, q_0, F>\), tale che:
  \begin{itemize}
    \item \(Q=V_N\cup\{q_F\}\);
    \item \(I=V_T\);
    \item \(q_0=S\);
    \item \(F=\{q_F\}\)
    \item \(\forall A\to bC, C\in \delta(A,B)\);
    \item \(\forall A\to b, q_F\in\delta(A,b)\)
  \end{itemize}

  In generale, l'automa a stati finiti \(A\), ottenuto a partire dalla grammatica regolare \(G\), è non deterministico.

  \begin{theorem}
    Dato un NPDA A è possibile costruire una grammatica G non contestuale (di tipo 2) ad esso equivalente, ossia in grado di riconoscere lo stesso linguaggio riconosciuto da A, e viceversa. Dunque, le grammatiche non contestuali e gli automi a pila non deterministici sono modelli differenti per descrivere la stessa classe di linguaggi.
  \end{theorem}

  \begin{theorem}
    Data una TM M utilizzata come accettatore di linguaggi è possibile costruire una grammatica generale G (di tipo 0) ad essa equivalente, ossia in grado di riconoscere lo stesso linguaggio riconosciuto da M, e viceversa.Dunque, le grammatiche non ristrette e le macchine di Turing sono modelli differenti per descrivere la stessa classe di linguaggi.
  \end{theorem}

  \section{Espressioni Regolari}
  Un'espressione regolare è un'espressione utilizzabile per denotare un linguaggio attraverso la struttura delle stringhe che lo compongono.

  \begin{definition}
    Dato un alfabeto di simboli terminali denotato con \(V_T\), si definiscono su di esso le espressioni regolari e i corrispondenti linguaggi denotati:
    \begin{itemize}
      \item \(\emptyset\) è un'espressione regolare che denota il linguaggio vuoto;
      \item \(\forall a\in V_T, a\) è un'espressione regolare che denota il linguaggio formato solo dal simbolo \(a\);
      \item Se \(R_1\) ed \(R_2\) sono espressioni regolari, anche la loro unione, indicata con \(R_1+R_2\) o \(R_1\;|\;R_2\), è un'espressione regolare;
      \item Se \(R_1\) ed \(R_2\) sono espressioni regolari, anche la loro concatenazione, indicata con \(R_1\cdot R_2\), è un'espressione regolare;
      \item Se R è un'espressione regolare, anche la stella di Kleene di R, indicata con \(R^*\), è un'espressione regolare.
    \end{itemize}
  \end{definition}

  Nessun'altra stringa è un'espressione regolare.

  Gli operatori \(\;|\;, \cdot, ^*\) definiti per le espressioni regolari, hanno un implicito ordine di applicazione, se non indicato diversamente dall'uso delle parentesi. In particolare, * ha la precedenza rispetto a \(\cdot\), che ha a sua volta la precedenza su \(\;|\;\).
  
  Inoltre, vale anche il seguente teorema:
  \begin{theorem}
    La classe dei linguaggi denotati dalle espressioni regolari coincide con la classe dei linguaggi regolari. 
  \end{theorem}

  \section{Riepilogo}

  \begin{table}[ht]
    \caption{Relazione fra grammatiche, linguaggi e automi}
    \centering
    \vspace{10px}
    \begin{tabular}{c c c c}
      Gerarchia & Grammatiche & Linguaggi & Automa minimo \\ 
      \hline
      tipo 0 & Generali & Ricorsivamente enumerabili & TM\\
      tipo 1 & Dipendenti dal contesto & Dipendenti dal contesto & LBA*\\
      tipo 2 & Non contestuali & Non contestuali & NPDA\\
      tipo 3 & Regolari & Regolari & FSA\\
    \end{tabular}
  \end{table}

  * Gli LBA (Linear Bounded Automata) sono un particolare tipo di macchina di Turing non deterministica in cui la lunghezza del nastro è dunzione lineare della dimensione della stringa in ingresso. Tali automi non sono stati trattati in questo documento. 
  \chapter{Logica}
Come già visto più volte, i linguaggi possono essere rappresentati in modi differenti, tramite modelli e astrazioni via via più potenti, tra cui:
\begin{itemize}
  \item Insiemi;
  \item Pattern;
  \item Espressioni Regolari;
  \item Modelli Operazionali (come gli Automi);
  \item Modelli Generativi (come le Grammatiche);
  \item Modelli Dichiarativi (come la Logica).
\end{itemize}

\noindent
I concetti di logica introdotti di seguito sono utilizzati per definire in maniera differente i linguaggi.

\section{Logica Proposizionale}
Il calcolo proposizionale, detto anche logica proposizionale, è un modello dichiarativo formale della logica matematica che si basa sul concetto di proposizione, ovvero frasi che possono assumere solamente il valore vero o il valore falso. In generale, ogni linguaggio consiste di una sintassi e di una semantica: la sintassi è l'insieme delle regole attraverso cui è possibile costruire le frasi di cui il linguaggio è composto, mentre la semantica spiega il significato delle varie frasi del linguaggio. 

\paragraph*{Sintassi}
In modo formale:
\begin{definition}
  La logica proprosizionale è composta da un linguaggio \(\mathcal L\), il cui alfabeto è costituito dai seguenti elementi:
  \begin{enumerate}
    \item Un insieme numerabile (finito o non) di proposizioni (simboli di relazione nullaria), che possono essere simboli, stringhe o frasi;
    \item Un insieme di connettivi logici: \(\lnot\) (NOT), \(\wedge\) (AND), \(\vee\) (OR), \(\Rightarrow\) (Implicazione) e \(\Leftrightarrow \)(Coimplicazione);
    \item Un insieme di simboli di punteggiatura: ( e )
  \end{enumerate}
  I simboli che compongono l'alfabeto sono privi di significato: assegnarne uno è compito della semantica. 
\end{definition}

\begin{table}[!th] \label{Tabella delle verità}
  \caption{Tabella della verità dei connettivi logici}
  \vspace*{10pt}

  \centering
  \begin{tabular}{c c || c | c | c | c | c }
    \(F\) & \(G\) & \(\lnot F\) & \(F \wedge G\) & \(F \vee G\) & \(F \Rightarrow G\) & \(F\Leftrightarrow G\) \\
    \hline
    T & T & F & T & T & T & T \\
    T & F & F & F & T & F & F \\
    F & T & T & F & T & T & F \\
    F & F & T & F & F & T & T \\
  \end{tabular}  
\end{table}

Una proposizione si dice essere atomica quando non può essere scomposta in parti più piccole. Nel caso contrario, la proposizione è composta da due o più proposizioni più piccole legate fra loro tramite i connettivi logici appena introdotti. Le parentesi sono utilizzare per modificare la precedenza dei connettivi, che di base avrebbero il seguente ordine logico: \(\lnot, \wedge, \vee, \Rightarrow, \Leftrightarrow\).

La sintassi del linguaggio definisce le sequenze ammissibili di simboli sull'alfabeto, le cosiddette formule ben formate (fbf). L'insieme di queste formule ben definite su \(\mathcal L\) è il più piccolo insieme tale che ogni proposizione è una formula e, se \(F\) e \(G\) sono formule, allora anche \(\lnot F\), \(F \wedge G\), \(F \vee G\), \(F \Rightarrow G\) \(F \Leftrightarrow G\) sono formule. In logica proposizionale si ha che se \(A\) è una proposizione, allora \(A\) e \(\lnot A\) sono letterali, in cui \(A\) è letterale positivo, mentre \(\lnot A\) è letterale negativo. Infine, si dice letterale complementare la proposizione \(\bar L\) definito come \(\lnot A\) se \(L=A\), oppure \(A\) se \(L=\lnot A\). 

\break

Una volta introdotte le formule ben formate, è possibile definire le sottoformule, ovvero una parte di una fbf che è a sua volta una fbf. L'insieme  \(Stfm(F)\) delle sottoformule di \(F\) è definito come il più piccolo insieme di formule tale che:
\begin{itemize}
  \item \(F \in Stfm(F)\);
  \item Se \(\lnot G \in Stfm(F)\), allora \(G \in Stfm(F)\);
  \item se \(G \wedge H, G\vee H, G \Rightarrow H, G \Leftrightarrow H \in Stfm(F)\), allora \(H, G \in Stfm(F)\). 
\end{itemize}

\paragraph*{Semantica}
La semantica, invece, ha lo scopo di assegnare un significato alle formule appena definite, tramite una funzione \(I\), detta interpretazione, che mappa ogni proposizione ad un valore di verità (vero o falso): formalmente, \(I: \{fbf\} \to \{0,1\}\). Tale funzione, quindi, non fa altro che assegnare il valore di vero (1) o falso (0) alle lettere proposizionali costanti \footnote{Le lettere proposizionali costanti sono \(T\) (o \(V\)) per rappresentare una proposizione vera, oppure \(\bot\) (o \(F\)) per rappresentare una proposizione falsa} e valuta il valore di verità di \(\lnot F, F \wedge G, F \vee G, F \Rightarrow G, F \Leftrightarrow G\) sulla base dei valori di verità delle proposizioni \(F\) e \(G\). 

Da qui, si introduce il simbolo \(\vDash\) (doppio tornello), che si utilizza per associare formule ed interpretazioni. Dunque, la scrittura \(I \vDash F\), che si legge \(I\) rende vera \(F\), vale nei seguenti casi:
\begin{itemize}
  \item \(I \vDash A\) se e solo se \(I(A) = T\), con \(A\) proposizione;
  \item \(I \vDash \lnot F\) se e solo se \(I \nvDash F\);
  \item \(I \vDash F \wedge G\) se e solo se \(I\vDash F\) e \(I \vDash G\);
  \item \(I \vDash F \vee G\) se e solo se \(I \vDash F\) o \(I \vDash G\);
  \item \(I \vDash F \Rightarrow G\) se e solo se \(I \nvDash G\) o \(I \vDash G\);
  \item \(I \vDash F \Leftrightarrow G\) se e solo se \(I \vDash F\Rightarrow G\) e \(I \vDash G \rightarrow F\).
\end{itemize}

\noindent
Dal concetto di interpretazione, si possono definire le seguenti proprietà della semantica delle formule proposizionali:
\begin{itemize}
  \item Se \(I\vDash F \), allora si dice che \(I\) è un modello di \(F\);
  \item \(F\) si dice valida (o si dice essere una tautologia) se e solo se per ogni interpretazione \(I\) vale che \(I \vDash F\);
  \item \(F\) è soddisfacibile se e solo se esiste un'interpretazione \(I\) tale che \(I \vDash F\);
  \item \(F\) è falsificabile se e solo se esiste un'interpretazione \(I\) tale che \(I \nvDash F\);
  \item \(F\) è insoddisfacibile se e solo se per ogni interpretazione \(I\) vale che \(I \nvDash F\);
  \item \(F\) è contingente se e solo se è sia soddisfacibile che falsificabile;
  \item Ogni formula del tipo \(F \wedge \lnot F\) è una contraddizione, indicata con \(\bot\);
  \item Ogni formula del tipo \(F \vee \lnot F\) è detta principio del terzo escluso, indicata con \(\top \)
\end{itemize}

\paragraph{}
Un insieme di formule \(\mathcal F\) comporta logicamente una formula \(G\) o, equivalentemente, una formula \(G\) è una conseguenza logica di un insieme di formule \(\mathcal F\), se ogni modello di \(\mathcal F\) è anche un modello di \(G\) e si scrive con \(\mathcal F \vDash G\). Dopo aver stabilito una corrispondenza semantica fra due formule logiche, è possibile anche stabilire una relazione di equivalenza fra due formule logiche, relazione che vale solamente se la corrispondenza fra le due formule è biunivoca, ovvero se vale sia \(F \vDash G\) che \(G \vDash F\): una tale relazione si rappresenta con la scrittura \(F \equiv G\).

Esistono innumerevoli equivalenze notevoli:
  \begin{flalign*}
    F \wedge F &\equiv F \\
    F \vee F &\equiv F \\
    F \wedge G &\equiv G\wedge F \\
    F \vee G &\equiv G \vee F \\
    F \wedge (G \wedge H) &\equiv (F \wedge G) \wedge H\\
    F \vee (G \vee H) &\equiv (F \vee G) \vee H \\
    (F \wedge G) \vee F &\equiv F \\
    (F \vee G) \wedge F &\equiv F \\
    F \wedge (G \vee H) &\equiv (F\wedge G) \vee (F\wedge H) \\
    F \vee (G \wedge H) &\equiv (F \vee G) \wedge (F \vee H) \\
    \lnot\lnot F &\equiv F \\
    \lnot (F\wedge G) &\equiv \lnot F \wedge \lnot G\\
    \lnot (F \vee G) &\equiv \lnot F \wedge \lnot G \\
    F \Leftrightarrow G &\equiv (F \Rightarrow G) \wedge (G \Rightarrow F) \\
    F \Rightarrow G &\equiv \lnot F \vee G \\
    F \Rightarrow G &\equiv \lnot G \Rightarrow  \lnot F
  \end{flalign*}  
In logica prposizionale è anche possibile sostituire una sottoformula \(G\), di una formula ben formata \(F\), con una formula \(H\): la formula risultate viene indicata con la scrittura \(F[G \backslash H]\). Tale sostituzione, però, può avvenire solamente se \(G \equiv H\). 

In base a questi concetti si può notare che non tutti i connettivi logici sono strettamente necessari, in quanto possono essere sostituiti con altri. A questo proposito, un insieme di connettivi è detto funzionalmente completo se e solo se qualunque formula proposizionale può essere trasformata in una formula semanticamente equivalente che contiene solamente i connettivi dell'insieme dato. Sfruttando tali insiemi, detti minimali, e le equivalenze semantiche, è possibile definire delle forme normali, che introducono degli schemi sintattici di completa generalità semantica per scrivere formule, permettendo così di formalizzare il significato di qualsiasi formula che si possa scrivere con la completa generalità della logica proposizionale. In altre parole, per ogni formula ben formata esistono una o più formule logicamente equivalenti ad essa scritte in una forma normale. Esistono tre principali forme normali per la logica proposizionale, chiamate forma negativa, forma congiuntiva e forma disgiuntiva. Una formula è in forma normale negativa se e solo se è composta solamente da letterali, congiunzioni e disgiunzioni; una formula è in forma normale congiuntiva (detta anche CNF) se e solo se ha la forma \(C_1 \wedge C_2 \wedge ... \wedge C_n\), dove \(C_i\) è una disgiunzione di letterali; una formula è in forma normale disgiuntiva (detta anche DFN) se e solo se ha la forma \(D_1 \vee D_2 \vee ... \vee D_n\), dove \(D_i\) è una congiunzione di letterali. 

A questo punto è possibile completare la definizione della logica proposizionale attraverso i concetti di assioma e regole di inferenza, che costituiscono un sistema formale assiomatico-deduttivo (in inglese calculus). Questi elementi definiscono una relazione di derivabilità (relazione già analizzata nel contesto delle grammatiche), detta anche dimostrabilità, tra un insieme di formule \(\mathcal F\) e una formula \(G\). Dunque, i sistemi formali della logica hanno un compito molto simile a quello assolto dalle grammatiche, ovvero producono meccanicamente una formula risultante a partire da un insieme iniziale di formule, applicando assiomi e regole di inferenza. Si scrive, quindi, \(\mathcal F \vdash G\) se \(G\) può essere ottenuto da \(\mathcal F\). Idealmente, la relazione di derivabilità dovrebbe essere corretta (cioè se \(\mathcal F \vdash G\) allora \(\mathcal F \vDash G\)) e completa (cioè se \(\mathcal F\vDash G\) allora \(\mathcal F\vdash G\)). Se una formula \(F\) può essere derivata in una teoria \(\mathcal F\) usando solamente assiomi e regole di inferenza di un sistema, allora si dice che \(F\) è un teorema. 

\section{Logica del Primo Ordine}
La logica proposizionale appena analizzata ha molte applicazioni, ma il suo potere espressivo è ristretto. Per questo motivo, nel 1979 è stata sviluppata  la logica del primo ordine, che permette dal punto di vista ontologico di considerare non solo fatti (come avveniva nella logica proposizionale), ma anche proprietà, relazioni e funzioni. 

\paragraph*{Sintassi}
In modo formale:
\begin{definition}
  La logica del primo ordine è composta da un linguaggio \(\mathcal L\), il cui alfabeto è costituito dai seguenti elementi:
  \begin{enumerate}
    \item Un insieme numerabile infinito di variabili;
    \item Un insieme di simboli di funzione;
    \item Un insieme di simboli di predicati (o relazioni);
    \item Un insieme di connettivi logici: \(\lnot\) (NOT), \(\wedge\) (AND), \(\vee\) (OR), \(\Rightarrow\) (Implicazione) e \(\Leftrightarrow \)(Coimplicazione);
    \item Un insieme di quantificatori: \(\exists\) (Esiste) e \(\forall\) (Per ogni);
    \item Un insieme di simboli di punteggiatura: ( , ) e le virgole.
  \end{enumerate}
  Ogni simbolo di funzione e relazione ha una arietà fissata, che indica il numero di argomenti associati a quella determinata funzione. Le funzioni nullarie sono dette costanti, mentre i predicati costanti sono detti proposizioni. I simboli dell'alfabeto sono privi di significato: assegnarne uno è compito della semantica.
\end{definition}

Per poter scrivere formule nella logica del primo ordine c'è la necessità di denotare tutti gli oggetti di cui il linguaggio \(\mathcal L\) può parlare, detti termini. Tale denotazione avviene induttivamente come segue:
\begin{itemize}
  \item ogni variabile è un termine della formula;
  \item se \(f\) è un simbolo di funzione \(n\)-aria e \(t_1, t_2, ..., t_n\) sono termini, allora \(f(t_1, t_2,...,t_n)\) è un termine.
\end{itemize}

Gli oggetti appena denotati attraverso i termini, si possono utilizzare all'interno delle formule della logica del primo ordine, definite anch'esse in maniera induttiva. L'insieme delle formule della logica del primo ordine è definito come il più piccolo insieme tale che:
\begin{itemize}
  \item Se \(p\) è un simbolo di relazione \(n\)-aria e \(t_1, t_2,...,t_n\) sono termini, allora \(p(t_1, t_2,...,t_n)\) è una formula detta atomica o semplicemente atomo;
  \item Se \(F\) e \(G\) sono formule e \(X\) è una variabile, allora \(\lnot F, F\vee G, F\wedge G, F \Rightarrow G, F \Leftrightarrow G, \exists XF, \forall XF\) sono formule.
\end{itemize}

Nella scrittura di formule appartenenti alla logica del primo ordine, c'è un'osservazione da fare: quando si utilizza il quantificatore \(\forall\) il connettivo principale utilizzato è \(\Rightarrow\), mentre nel caso si utilizzi il quantificatore \(\exists\) allora il connettivo principale è \(\wedge\). Inoltre, se \(QX(F)\) rappresenta una formula in cui \(Q\) è un quantificatre, allora \(F\) si dice ambito di \(Q\) e che \(Q\) è applicato ad \(F\). Un'occorrenza di una variabile in una formula è legata se e solo se la sua occorrenza è entro l'ambito di un quantificatore che impiega quella variabile, altrimenti è libera. Una formula è chiusa se e solo se non contiene occorrenze libere di variabili. Le formule chiuse sono quelle per le quali, data un'interpretazione \(I\), si può calcolare la veridicità.

\paragraph*{Semantica}
Come per il caso della logica proposizionale, anche la logica del primo ordine ha una semantica basata sul concetto di interpretazione: un'interpretazione \(I\) di un alfabeto \(A\) è un dominio non vuoto \(D\) (indicato anche con \(|I|\)) e una funzione che che associa ogni costante \(c\in A\) a un oggetto \(c_I \in D\), ogni simbolo \(n\)-ario di funzione \(f\in A\) a una funzione \(f_I:D^n\to D\) e ogni simbolo \(n\)-ario di predicato \(p\in A\) a una relazione \(p_I\subseteq D\times D\times ... \times D\), \(n\) volte. Prima di poter assegnare un significato alle fromule, va definito il significato di ogni termine, indicato con \(\phi_I(t)\) con \(t\) il termine a cui dare singificato nell'interpretazione \(I\). 

\break

\noindent \(\phi\) è induttivamente definito nel seguente modo:
\begin{enumerate}
  \item \(c_I\) se \(t\) è una costante \(c\);
  \item \(\phi(X)\) se \(t\) è una variabile \(X\);
  \item \(f_I(\phi_I(t_1),...,\phi_I(t_n))\) se \(t\) è nella forma \(f(t_1,...,t_n)\).
\end{enumerate}
Ora, sia \(\phi\) una valutazione, \(X\) una variabile, \(I\) un'interpretazione e \(c_I\in |I|\), allora \(\phi[X\to c_I]\) è una valutazione identica a \(\phi\), eccetto per il fatto che mappa \(X\) in \(c_I\). Il significato di una formula, quindi, è un valore di verità che è definito induttivamente. Dunque la scrittura \(I \vDash_\phi F\), che si legge \(F\) è vero rispetto all'interpretazione \(I\) e al significato \(\phi\), vale nei seguenti casi:
\begin{itemize}
  \item \(I \vDash_\phi p(t_1,...,t_n)\) se e solo se \(<\phi_I(t_1),..., \phi_I(t_n)>\in p_I\);
  \item \(I \vDash_\phi \lnot F\) se e solo se \(I\nvDash_\phi F\);
  \item \(I \vDash_\phi (F\wedge G)\) se e solo se \(I \vDash_\phi F\) e \(I \vDash_\phi G\);
  \item \(I \vDash_\phi (F \vee G)\) se e solo se \(I \vDash_\phi F\) o \(I \vDash_\phi G\);
  \item \(I \vDash_\phi (F \Rightarrow G)\) se e solo se \(I\nvDash_\phi F\) o \(I\vDash_\phi G\)
  \item \(I \vDash_\phi (F \Leftrightarrow G)\) se e solo se \(I \vDash_\phi (F \Rightarrow G)\) e \(I\vDash_\phi (G \Rightarrow F)\);
  \item \(I\vDash_\phi\forall X(F)\) se e solo se \(I \vDash_{\phi[X\to c_I]} F\) per ogni \(c_I\in |I|\);
  \item \(I\vDash_\phi\exists X(F)\) se e solo se \(I \vDash_{\phi[X\to c_I]} F\) per qualche \(c_I\in |I|\)
\end{itemize}

Se \(F\) è una formula chiusa, il suo significato dipende solamente dall'interpretazione \(I\), che viene detta modello per \(F\) (\(I\vDash F\)) se e solo se per ogni valutazione \(\phi\) si ha che \(I\vDash_\phi F\). Inoltre, se \(\mathcal F\) è un insieme di formule, un'interpretazione è modello di \(\mathcal F\) se e solo se tale interpretazione è modello per ogni \(F \in \mathcal F\). La relazione di conseguenza logica \(\vDash\) tra insiemi di formule e formule può essere estesa anche per la logica del primo ordine, così come anche i concetti di validità, soddisfacibilità, falsificabilità, contingenza e insoddisfacibilità, analizzati nello studio della logica proposizionale.  

Come nel caso della logica proposizionale, esistono innumerevoli equivalenze notevoli:
\begin{flalign*}
  \forall X(F) &\equiv \lnot(\exists X(\lnot F))\\
  \exists X(F) &\equiv \lnot(\forall X(\lnot F))\\
  \forall X(F)\wedge (\forall X)G &\equiv \forall X(F\wedge G)\\
  \exists X(F) \vee (\exists X)G &\equiv \exists X(F\vee G)\\
  (\forall X)(\forall Y)F &\equiv (\forall Y)(\forall X)F\\
  (\exists X)(\exists Y)F &\equiv (\exists Y)(\exists X)F\\
  (\forall X(F))\wedge G &\equiv \forall X(F \wedge G)^*\\
  (\forall X(F))\vee G &\equiv \forall X(F \vee G)^*\\
  (\exists X(F))\wedge G &\equiv \exists X(F\wedge G)^*\\
  (\exists X(F))\vee G &\equiv \exists X(F\vee G)^*
\end{flalign*}
Le equivalenze segnate con * sono valide solo se \(X\) non è libera in \(G\).

\break

\section{Logica Monadica del Primo Ordine}
In questa sezione si analizza un frammento della logica monadica del primo ordine, che permette di descrivere certe stringhe su un determinato alfabeto \(I\). La logica monadica, come suggerisce il nome, si occupa solamente dei predicati monadici, ovvero di tutti quei predicati che hanno arietà pari ad uno.

\paragraph*{Sintassi}
 Una generica formula \(F\) appartenente alla logica monadica del primo ordine ha una sintassi molto semplice, articolata nei seguenti casi:
\begin{itemize}
  \item \(a(x)\) con \(a\in I\), ovvero un predicato unario per ogni simbolo dell'alfabeto \(I\);
  \item \(x < y\), che costituisce l'unica eccezzione della logica monadica in quanto non è un operatore unario, ma binario;
  \item \(\lnot F\), operatore NOT;
  \item \(F \wedge F\), operatore AND;
  \item \(\forall x(F)\), quantifcatore di una formula  rispetto ad una variabile;
\end{itemize}
Il dominio delle variabili è \(\mathbb N\).

Si può osservare come, rispetto alla logica proposizionale e del primo ordine, la logica monadica non fa uso della maggior parte degli operatori logici, del quantificatore esistenziale, della maggior parte degli operatori matematici, di oggetti o funzioni. Tutti i costrutti non utilizzati da tale logica possono essere drivati dagli operatori che sono stati presentati.

Le formule scritte in questa logica possono confrontare dei numeri rappresentati dalle variabili, che a loro volta rappresentano posizioni all'interno delle stringhe scritte sull'alfabeto di riferimento. Quindi con l'alfabeto \(I\) di riferimento posso scrivere delle stringhe mediante i predicati unari e, con la logica monadica è possibile calcolare la posizione, di determinati caratteri all'interno di tali stringhe per poterli confrontare con altre posizioni.

Con gli operatori logici, matematici e i quantificatori utilizzati nella logica monadica del primo ordine, come già detto in precedenza, è possibile derivare tutti gli altri operatori. Nello specifico:
\begin{flalign*}
  F_1 \vee F_2 &\equiv \lnot(\lnot F_1 \and \lnot F_2)\\
  F_1 \Rightarrow F_2 &\equiv \lnot F_1 \vee F_2\\
  \exists x(F) &\equiv \lnot \forall x(\lnot F)\\
  x = y &\equiv \lnot(x<y) \wedge \lnot(y<x)\\
  x\le y &\equiv \lnot(y<x)
\end{flalign*}

In realtà si possono anche introdurre alcune abbraviazioni di comodo. Le più utilizzate sono:
\begin{flalign*}
  x = 0 &\equiv \forall y(\lnot(y<x))\\
  successor(x,y) &\equiv x < y \wedge \exists z(x<z \wedge z<y)\\
  y = x + 1 &\equiv successor(x,y)\\
  y=x+k\,(k>1)&\equiv\exists z_1,...,z_{k-1}(y=z_{k-1}+1\wedge...\wedge z_1=x+1)\\
  y = x - 1 &\equiv successor(y,x)\\
  last(x) &\equiv \lnot \exists y(y>x)
\end{flalign*}

Le altre costanti (1, 2, 3, ...) possono essere ricavate tramite l'utilizzo del predicato successor (di arietà due) applicata alla costante 0 tante volte quanto il numero che si vuole ottenere.

Una volta introdotti tutti questi concetti della logica monadica del primo ordine, si può ora analizzare l'interpretazione di queste formule logiche rispetto a stringhe di un determinato alfabeto di riferimento.

Dato un alfabeto \(I\), una stringa \(w\in I^+\) ed un simbolo \(a\in I\), \(a(x)\) è vero se e solo se l'\(x\)-esimo simbolo di \(w\) è \(a\), considerando un'indicizzazione che parte da 0.

\paragraph*{Semantica} La semantica, nella logica monadica del primo ordine, assegna un valore di verità o falsità alle formule, tramite una funzione di assegnamento \(v_1 : V_1\to [0,...,|w|-1]\), in cui \(w\in I^+\) è una stringa composta a partire dall'alfabeto \(I\), mentre \(V_1\) è l'insieme delle variabili. Tale funzione si comporta nel seguente modo:
\begin{itemize}
  \item \(w, v_1 \vDash a(x)\) se e solo se \(w = uav\) e \(|u| = v_1(x)\);
  \item \(w, v_1 \vDash x<y\) se e solo se \(v_1(x) < v_1(y)\);
  \item \(w, v_1 \vDash \lnot F\) se e solo se \(w, v_1 \nvDash F\);
  \item \(w, v_1 \vDash F_1 \wedge F_2\) se e solo se \(w,v_1 \vDash F_1\) e \(w,v_1\vDash F_2\);
  \item \(w, v_1 \vDash \forall x(F)\) se e solo se \(w, v_1' \vDash F\,\, \forall v_1'\) con \(v_1'(y) = v_1(y),\, y\neq x\).
\end{itemize}

\paragraph*{Proprietà} La logica monadica del primo ordine ha delle importanti proprietà che consentono di identificare quali linguaggi è possibile rappresentare con tale logica. Nello specifico, tutti i linguaggi esprimibili mediante questa logica sono chiusti rispetto all'unione, all'intersezione e al complemento, Inoltre, si può dimostrare che non è possibile esprimere il linguaggio \(L_p\) composto da tutte e sole le stringhe di lunghezza pari con \(I = \{a\}\): quindi, si può osservare come la logica monadica del primo ordine è strettamente meno espressiva rispetto agli automi a stati finiti, il che significa che data una formula di tale logica, si può sempre costruire un FSA equivalente, ma non il contrario. Infine, i linguaggi definiti da questa logica non sono chiusi rispetto alla stella di Kleene, ed è per questo motivo che tali linguaggi sono anche detti star-free, definibili tramite l'unione, l'intersezione, il complemento e la concatenazione di linguaggi finiti. 

\section{Logica Monadica del Secondo Ordine}
Per riuscire ad ottenere lo stesso potere espressivo degli FSA è necessario permettere alla logica del primo ordine di quantificare sui predicati monadici, ovvero di poter quantificare anche su insiemi di posizioni. Si ammettono quindi formule del tipo \(\exists X(F)\) oppure \(\exists X(x)\), in cui \(X\) è detta essere una variabile del secondo ordine, il cui dominio non è più l'insieme dei numeri naturali \(\mathbb N\), ma l'insieme dei predicati monadici. Per convenzione si utilizzano le lettere maiuscole per indicare le variabili del secondo ordine e lettere minuscole per le variabili del primo ordine.

\paragraph*{Semantica}
L'assegnamento delle variabili del secondo ordine, che fanno parte dell'insieme\(V_2\), avviene attraverso la funzione \(v_2: V_2 \to \wp ([0,...,|w|-1])\) tale che:
\begin{itemize}
  \item \(w,v_1,v_2 \vDash X(x)\) se e solo se \(v_1(x) \in v_2(X)\);
  \item \(w,v_1,v_2 \vDash \exists X(F)\) se e solo se \(w,v_1, v_2'\vDash F\) per qualche \(v_2'(Y) = v_2(Y),\, Y\neq X\)
\end{itemize}

Tramite questa logica, si possono scrivere formule che risolvono il problema della rappresentazione del linguaggio \(L_p\), composto da tutte e sole le stringhe di lunghezza pari con \(I=\{a\}\), che si può indicare nel seguente modo: \(\exists D(\forall x(\lnot D(0) \wedge (\lnot D(x) \Leftrightarrow D(x+1)) \wedge a(x) \wedge (last(x) \Rightarrow D(x))))\), in cui \(D\) è una variabile del secondo ordine che indica le posizioni dispari.

\paragraph{}
Si può dimostrare che, data una qualsiasi formula \(F\) appartenente alla logica monadica del secondo ordine, è possibile costruire un automa a stati finiti che accetta lo stesso linguaggio \(L\) definito da \(F\) e, viceversa, dato un qualsiasi automa a stati finiti, è possibile enunciare una formula che riconosce lo stesso linguaggio. Di conseguenza, si può affermare che la classe dei linguaggi definibili dalle formule della logica monadica del secondo ordine coincide con i linguaggi regolari. 

\begin{figure}[!h]
  \centering
  \includegraphics[width=17cm]{gerarchia linguaggi.png} 
  \caption{Gerarchia dei linguaggi}
\end{figure}

\break

\section{Logica per la descrizione di Proprietà}
Quando si programma una funzione è importante definire con precisione quale sia il suo funzionamento, senza necessariamente descrivere come funzioni. A tal proposito si è soliti scrivere le cosiddette prcondizioni e postcondizioni:
\begin{itemize}
  \item Precondizione: indica le condizioni che devono valere prima che la funzione venga invocata;
  \item Postcondizione: indica le condizioni che devono valere al termine dell'esecuzione del programma.  
\end{itemize}

Dunque, il programma \(P\) deve essere tale per cui se la precondizione \(Pre\) vale prima dell'esecuzione di tale programma, allora la post condizione \(Post\) deve valere dopo l'esecuzione. Tali condizioni possono essere espresse in linguaggi diversi, ma il più comune è la logica del primo ordine.
  \chapter{Computabilità}
  In questa sezione si cercherà di rispondere alla domanda: quali problemi possono essere affrontati e risolti mediante gli automi analizzati?
  Si ricordi che automi e grammatiche, pur essendo modelli matematici, si possono considerare dispositivi meccanici per la risoluzione di problemi e che esistono formalismi che hanno un potere espressivo maggiore di altri, ossia sono in grado di riconoscere una classe di linguaggi che altri formalismi non riescono a riconoscere. Inoltre, si ricordi che, nessun formalismo è più potente delle macchine di Turing, sia dal punto di vista del riconoscimento che della traduzione di linguaggi e, per tanto, sono detti formalismi massimi.

  \section{Formalizzazione dei Problemi}
  Molti problemi possono essere opportunamente descritti come il riconoscimento di un determinato linguaggio o come la sua traduzione in un altro linguaggio. Ogni problema matematico è descrivibile mediante una di queste forme, alla sola condizione che il dominio di tale problema sia un insieme numerabile, in maniera tale che i suoi elementi si possono porre in corrispondenza biunivoca con gli elementi di \(\mathbb N\) o, se si preferisce, di \(V^*\), in cui \(V\) rappresenta un alfabeto. Dunque, il problema di origine si può riformulare come il problema di calcolo di una funzione \(f:\mathbb N\to\mathbb N\). Quanto detto è in perfetto accordo con tutti i formalismi matematici esaminati fino ad ora: questi, infatti, sono discreti e hanno un dominio matematico numerabile.

  Il riconoscimento di linguaggi e la loro traduzione sono due formulazioni differenti di un problema, che sono facilmente riducibili l'uno all'altro. Infatti, il problema di stabilire se una determinata stringa \(x\) appartenga o meno al linguaggio \(L\) può anche essere impostato come la traduzione \(\tau_L(x)\), per cui \(\tau_L(x) = 1\) se \(x\in L\), \(\tau_L(x)=0\) altrimenti. Viceversa, data la traduzione \(\tau:V_1^*\to V_2^*\), si può definire il linguaggio seguente:
  
  \(L_{\tau}=\{z\;|\;z=x\$y,\; x\in V_1^*,\; y=\tau(x)\in V_2^*,\; \$ \notin (V_1\cup V_2)\} \)
  
  \noindent ovvero il linguaggio formato da una stringa e la sua traduzione, separati dal simbolo \(\$\). Un dispositivo che riconosce il linguaggio \(L_{\tau}\) può essere utilizzato come trasduttore che calcola \(\tau\): per ogni \(x\), infatti, è possibile enumerare tutte le \(y\in V_2^*\) e verificare se \(x\$y\in L_{\tau}\) oppure no. Prima o poi, se la funzione \(\tau(x)\neq \bot \), verrà trovata una stringa per cui la macchina risponderà positivamente.

  \section{Tesi di Church}
  Le macchine di Turing, come visto in precedenza, sono il formalismo più potente che si ha a disposizione per il calcolo computazionale: ogni programma eseguibile da un calcolatore moderno può essere eseguito anche da una macchina di Turing. Dunque, le macchine di Turing hanno la stessa espressività dei linguaggi di programmazione ad alto livello, detti anche Turing completi. 

  Più formalmente, data una TM M, è possibile costruire un programma, scritto in un determinato linguaggio di programmazione (come C, Java ecc...), che simuli il comportamento di M, purchè il calcolatore disponga di una quantità di memoria sufficiente durante l'esecuzione. Inoltre, dato un programma scritto in un determinato linguaggio di programmazione, è possibile costruire una TM M che calcoli la stessa funzione calcolata dal programma.

  \begin{thesis}[Tesi di Church - Prima Parte]
    Non esiste alcun formalismo, per modellare una determinata computazione meccanica, che sia più potente delle TM e dei formalismi ad essi equivalenti.
  \end{thesis}

  La tesi di Church non è un teorema perchè per sua natura non è dimostrabile, in quanto andrebbe verificato ogni qual volta si introduce un nuovo formalismo computazionale.

  In base a questo risultato si può affermare che, se si riesce a dimostrare che un determinato problema è risolvibile da una TM, allora è sicuramente possibile risolverlo mediante un modello matematico di calcolo, che abbia la stessa potenza delle macchine di Turing. Viceversa, se si dimostra che un problema non può essere risolto da una TM, allora è verificato che tale problema è irrisolvibile da qualunque modello matematico.

  \paragraph{Algoritmi}
  Si introduce ora il concetto di algoritmo, centrale nell'informatica. Per algoritmo si intende la procedura di risoluzione di un problema mediante un dispositivo automatico di calcolo. Gli algoritmi si possono anche intendere come un metodo astratto di descrizione dei programmi eseguibili, ovvero una sequenza di comandi che, una volta eseguiti, portano alla risoluzione del problema.
  
  Ogni algoritmo ha le seguenti proprietà:
  \begin{enumerate}
    \item Un algoritmo deve contenere una sequenza finita di istruzioni;
    \item Ogni istruzione deve essere immediatamente eseguibile da qualche procedimento meccanico di calcolo, ossia deve esistere un processore che sia in grado di comprendere univocamente le istruzioni e di eseguirle producendo risultati precisi ed inequivocabili;
    \item Il processore è dotato di celle di memoria in cui possono essere immagazzinati i riultati intermedi;
    \item La computazione è discreta, ossia l'informazione è codificata in forma digitale e la computazione procede attraverso passi discreti;
    \item Gli algoritmi vengono eseguiti deterministicamente;
    \item Non esiste un limite finito sui dati di ingresso e di uscita: ogni calcolatore può ricevere in ingresso o emettere in uscita stringhe di lunghezza arbitraria;
    \item Non esiste un limite alla quantità di memoria richiesta per effettuare i calcoli;
    \item Non esiste un limite al numero di passi discreti richiesti per effettuare un calcolo ed è dunque possibile avere computazioni infinite.
  \end{enumerate}

  La tesi di Church non si ferma solo nell'affermazione che nessun formalismo sia più espressivo delle TM, ma afferma anche che nessun algoritmo è in grado di risolvere un problema che non è risolvibile da una TM. Formalmente:

  \begin{thesis}[Tesi di Church - Seconda Parte]
    Ogni algoritmo per la soluzione automatica di un problema può essere codificato in termini di una TM (o di un formalismo a potenza equivalente).
  \end{thesis}

  \begin{theorem}
    Ogni funzione (o problema), per cui esiste una TM che la calcoli (o risolva), si dice computabile o calcolabile (o risolvibile). Un problema risolvibile la cui risposta sia booleana ed esistente per ogni valore del dominio di definizione (ossia è formalizzato da una funzione calcolabile e totale) si dice decidibile.
  \end{theorem}

  Grazie alla seconda parte della tesi di Church si può affermare che è possibile studiare i limiti del calcolo automatico indipendentemente dalla formalizzazione del problema e del particolare modello computazionale.

  \section{Enumerazione delle TM}
  Le macchine di Turing possono essere viste come dei calcolatori astratti, specializzati nella risoluzione di un solo problema e non programmabili. Ci si pone quindi la domanda: 'le TM sono in grado di simulare i calcolatori programmabili e di risolvere i problemi da \(\mathbb{N}\) a \(\mathbb{N}\)?'

  Per poter rispondere a tale domanda, si noti innanzitutto che dato un qualsiasi insieme S, questo può essere enumerato algoritmicamente se è possibile stabilire una biiezione fra l'insieme \(S\) e l'insieme dei numeri naturali \(\mathbb{N}\), calcolabile attraverso un algoritmo o da una TM.
  Alla stessa maniera è possibile enumerare algoritmicamente l'insieme delle TM tramite una biiezione \(E:\{TM\}\leftrightarrow\mathbb{N}\). Tale biiezione è implementabile da un algoritmo che riceve in ingresso un numero \(n\) e ritorna la \(n\)-esima macchina di Turing. Un'enumerazione calcolabile da una TM viene chiamata Gödelizzazione, mentre il numero naturale biiettivamente associato da tale enumerazione ad una TM è detto numero di Gödel della TM. 

  Inoltre, è noto che una TM M può risolvere una funzione \(f_M:D\to R\), con \(D\) ed \(R\) opportunamente codificati nell'alfabeto di \(M\), dunque si indicherà con \(f_y\) la funzione calcolata dalla \(y\)-esima macchina di Turing, indicata con \(M_y=E(y)\).

  \section{Macchine di Turing Universali}
  Le UTM (Universal Turing Machines) sono TM in grado di modellare dispositivi generali di risoluzione dei problemi, in cui il problema da risolvere non viene codificato nella struttura del dispositivo (come avviene per le TM), ma gli viene fornito come input, assieme ai dati con cui operare (esattamente come gli odierni calcolatori). Le UTM sono quindi MT che calcolano la funzione \(g(y,x)=f_y(x)\), in cui \(y\) rappresenta la funzione \(f_y\), calcolata dalla TM \(M_y\), ed \(x\) rappresenta l'ingresso su cui \(M_y\) opera; calcolano, dunque, il valore della funzione \(f_y\) applicata ad \(x\).

  Come si può osservare, la UTM così definita non sembra appartenere all'insieme delle macchine di Turing, in quanto la funzione \(g(y,x)\) è opera da \(\mathbb{N}\times\mathbb{N}\) ad \(\mathbb{N}\), anzichè da \(\mathbb{N}\) ad \(\mathbb{N}\) come tutte le altre TM. È però possibile, come già dimostrato in precedenza, definire una biiezione calcolabile algoritmicamente, tramite la funzione:
  \begin{equation*}
    \displaystyle d(x,y) = x+\frac{(x+y)(x+y-1)}{2}
  \end{equation*}
  che mette in corrispondenza l'insieme \(\mathbb{N}\times\mathbb{N}\), composto dalle coppie di numeri naturali, all'insieme \(\mathbb{N}\), composto da numeri naturali.

  Graficamente, è come visitare le coppie di punti nel piano in un ordine prefissato, dove la posizione di un punto nella visita rappresenta il numero naturale associato alla coppia che identifica le coordinate del punto.
  
  \begin{figure}[!h]
    \centering
    \includegraphics[width=8cm]{biiezione.png} 
    \caption{Grafico di biiezione}
  \end{figure}

  Si osservi che la funzione \(g(y,x)\) è computabile da una macchina di Turing, ossia \(\exists i\in \mathbb N : f_i=g\), ed è possibile calcolarla tramite i seguenti passaggi:
  \begin{enumerate}
    \item Si sceglie un alfabeto finito \(A\) per codificare i numeri naturali ed ogni altra informazione richiesta per la computazione;
    \item Si traduce la rappresentazione di \(n\) in un'opportuna rappresentazione della coppia \(<x,y>\) corrispondente ad \(n\). La rappresentazione decimale di \(n\) può essere tradotta nelle due rappresentazioni decimali di \(x\) ed \(y\), separate dal simbolo \(\$\);
    \item Si traduce \(y\) in un'opportuna codifica della TM \(y\)-esima \(M_y\) nella enumerazione di Gödel;
    \item Si simula la computazione di \(M_y\) su \(x\).
  \end{enumerate}

  \begin{theorem}
    Per ogni x ed ogni y, esiste e si può costruire una macchina di Turing universale in grado di calcolare \(g(y,x)=f_y(x)\)
  \end{theorem}  
  Tramite questo teorema si può affermare che è possibile creare una macchina di Turing che simuli il comportamento degli odierni calcolatori "general purpose".

  \section{Problemi Algoritmicamente Irrisolvibili}
  Come si è visto in precedenza, tutte le funzioni computabili \(f_y:\mathbb{N}\to\mathbb{N}\) si possono enumerare: questo significa che la cardinalità dell'insieme delle funzioni computabili è pari ad \(\aleph_0\), ovvero alla cardinalità dei numeri naturali \(\mathbb{N}\). L'insieme delle funzioni \(\{f:\mathbb{N}\to\mathbb{N}\}\) contiene la classe delle funzioni \(\{f:\mathbb{N}\to\{0,1\}\), in quanto \(\{0,1\}\subseteq \mathbb{N}\). Quindi, poichè \(\;|\;\{f:\mathbb{N}\to\mathbb{N}\}\;|\; \ge \;|\;\{f:\mathbb{N}\to\{0,1\}\}\;|\; = \wp (\mathbb{N}) = 2^{\aleph_0}\), si può dedurre che la cardinalità della classe delle funzioni da \(\mathbb{N}\) ad \(\mathbb{N}\) è strettamente maggiore della cardinalità della classe delle funzioni computabili: dunque, gran parte delle funzioni di \(\mathbb{N}\) non può essere calcolata. 

  Ora, quando si vuole definire una funzione si usa un linguaggio che la esprima, ovvero un sottoinsieme del monoide libero su di un determinato alfabeto finito: dunque, il linguaggio è un insieme numerabile. Si ricava quindi che la classe delle funzioni denotabili è a sua volta numerabile.

  Quando si scrive un programma, ci sono diverse proprietà che si vorrebbero garantire. Una di queste è la terminazione del programma, ovvero la garanzia che, dato un qualsiasi ingresso conforme al programma stesso, esso termini la propria computazione e non vada, dunque, in un ciclo infinito. Nella realtà, però, non è possibile garantire a priori la terminazione del programma per un generico valore in ingresso, nè decidere attraverso un algoritmo se ciò possa avvenire in corrispondenza di uno specifico valore in ingresso. Più in generale, il problema della terminazione del calcolo automatico è in generale non decidibile, nonostante tale problema sia definibile.
  Si è quindi constatato che esistono problemi definibili, ma che non possono essere risolti algoritmicamente: dunque, l'insieme dei problemi definibili contiene strettamente l'insieme dei problemi risolvibili, nonostante entrambi siano numerabili e con la stessa cardinalità.

  \begin{figure}[!h]
    \begin{center}    
      \begin{tikzpicture}
        \foreach \X [count=\Y starting from 0.75] 
        in {\(f_y\) risolvibili, \(f\) definibili, \(f:\mathbb{N}\to\mathbb{N}\) (problemi totali)} {
          \draw (-\Y,-\Y/2) circle ({2*\Y} and \Y);
          \node at (1-2*\Y,-1.1*\Y) {\X}; 
        }
      \end{tikzpicture}
    \end{center}
    \caption{Gerarchia dei problemi}    
  \end{figure}

  \begin{theorem} [Halting Problem]
    Nessuna TM può calcolare la funzione \(g:\mathbb N\times \mathbb N \to \{0,1\}\) definita nel seguente modo:
    
    \(g(x,y) = if\; f_y(x) = \bot \; then \; 1 \; else \; 0\)
  \end{theorem}

  La dimostrazione di tale teorema si ottiene tramite la tecnica della diagonale, detta anche metodo di Cantor: l'obiettivo è quello di mostrare che un'enumerazione di oggetti di cardinalità almeno 2, non è completa, ossia che un oggetto che si vorrebbe trovare all'interno di tale enumerazione in realtà non è presente. L'enumerazione di una successione può essere rappresentata come una tabella con un numero infinito di righe. L'elemento che non compare in tale tabella viene individuato per assurdo considerando inizialmente la diagonale \(d\) (dunque \(d_i\) è l'elemento che si trova all'\(i\)-esima riga e all'\(i\)-esima colonna) e poi componendo una diagonale \(d'\) tale che, per ogni \(i\), \(d'_i\) sia diverso da \(d_i\).
  
  \begin{theorem}
    Nessuna TM è in grado di calcolare la funzione totale k definita nel seguente modo:

    \(k(x)= if\;f_x(x)\neq\bot\;then\;1\;else\;0\)
  \end{theorem}

  Questo problema rappresenta un caso speciale della funzione \(g(y,x)\) in quanto \(k(x)=g(x,x)\), dunque la calcolabilità della funzione \(k\) è direttamente correlata alla calcolabilità della funzione \(g\).
  Si noti che, in generale, se un problema è irrisolvibile, può accadere che un suo caso particolare sia risolvibile, mentre una sua generalizzazione è necessariamente irrisolvibile. Al contrario, se un problema è risolvibile, può accadere che una sua generalizzazione diventi irrisolvibile, mentre un suo caso particolare rimane sicuramente risolvibile.

  Un altro teorema certamente importante è il seguente:
  \begin{theorem}
    Nessuna TM è in grado di calcolare la funzione k definita nel seguente modo:

    \(k(y)=if\; f_y(x)\neq\bot\;then\;1\;else\;0\)
  \end{theorem}
  
  Da un punto di vista pratico questo problema è interessante perchè qualifica tutti i possibili dati in ingresso. Afferma, infatti, l'irrisolvibilità del problema di decidere se un certo programma termini la propria esecuzione per qualsiasi dato in ingresso o se, al contrario, per qualche dato il programma andrebbe in loop. Nel caso precedente, invece, si era interessati al problema di sapere se un certo programma con certi dati avrebbe terminato o meno la propria esecuzione.
  
  In definitiva, si è constatato che esistono problemi non risolvibili algoritmicamente. Ciò non esclude comunque la possibilità di trovare una soluzione per tali problemi, in quanto non tutti i problemi sono risolvibili tramite un procedimento algoritmico. 

  \section{Problemi di Decisione}
  Un problema di decisione è una domanda che ha come uniche risposte si o no (0, 1). Questo può anche essere espresso come un problema di appartenenza di un determinato elemento ad un certo insieme. Più in generale si noti che una qualsiasi proprietà di un determinato elemento di un insieme può essere formalizzata come un suo sottoinsieme (ad esempio, la proprietà di terminazione del calcolo per ogni valore dei dati in ingresso individua un sottoinsieme dell'insieme di tutti i programmi). 
  
  In questa sezione si prendono in particolare considerazione i sottoinsiemi di \(\mathbb N\), indicati convenzionalmente con \(S\subseteq\mathbb{N}\). Quindi, formalmente, dato un determinato elemento \(x\in\mathbb{N}\) e un insieme \(S\), si cerca di capire se \(x\) appartenga ad \(S\).

  \begin{definition}
    La funzione caratteristica \(c_S:\mathbb{N}\to\{0,1\}\) di un insieme S è definita come segue:

    \(c_S(x)=if\;x\in S\;then\;1\;else\;0\)
  \end{definition}

  La risolvibilità del problema di appartenenza ad un insieme (detta anche ricorsività dell'insieme) dipende dalla computabilità della funzione caratteristica \(c_S\), come definito di seguito:

  \begin{definition}
    Un insieme S è ricorsivo (o decidibile) se e solo se la sua funzione caratteristica è computabile.
  \end{definition}

  Si noti, inoltre, che per ogni insieme \(S\), la sua funzione caratteristica \(c_S\) è totale: infatti, dato un qualsiasi elemento \(x\in\mathbb{N}\), questo necessariamente appartiene o non appartiene all'insieme.
  
  \begin{definition}
    Un insieme S è ricorsivamente enumerabile (o semidecidibile) se e solo se è l'insieme vuoto oppure è l'immagine di una funzione totale e computabile \(g_s\), ovvero:

    \(S=I_{g_s} = \{x\;|\;\exists y, y\in\mathbb{N} \land x=g_s(y)\}\)
  \end{definition}

  Gli insiemi decidibili devono il loro nome al fatto che il problema di appartenenza può essere risolto tramite un algoritmo meccanico e che, quindi, una TM che implementi la loro funzione caratteristica fornisce necessariamente una risposta al quesito se \(x\in S, \forall x \in \mathbb{N} \). Inoltre, per ogni insieme ricorsivamente numerabile \(S\) è possibile costruire una sequenza \(x_0=g_S(0), x_1=g_S(1), x_2=g_S(2),...\) tale per cui, se \(x\in S\), allora esiste \(i\) tale che \(x=g_S(i)\). In questo caso, esaminando la sequenza di elementi \(\{x_i\}\) si riuscirà a trovare l'elemento \(x\), concludendo che questo appartiene all'insieme \(S\). Perciò, se per un qualsiasi \(\bar{i}\) risultasse che \(x\notin\{g_S(i)\;|\;0\le i \le \bar{i}\}\) non si potrebbe concludere nè che \(x\in S\), nè che \(x\notin S\): per questo motivo, l'insieme \(S\) viene anche detto semidecidibile.

  \begin{theorem}
    Se S è ricorsivo, è anche ricorsivamente enumerabile. \\
    S è ricorsivo se e solo se sia S che \(\bar{S}=\mathbb{N}-S\) sono ricorsivamente enumerabili.
  \end{theorem}

  \break

  Quindi, riepilogando, si dice che un insieme \(S\) è:
  \begin{itemize}
    \item Ricorsivo (o Decidibile), se e solo se la sua funzione caratteristica \(c_S\) è computabile;
    \item Ricorsivamente enumerabile (o Semidecidibile), se e solo se:
    \begin{itemize}
      \item S è l'insieme vuoto;
      \item S è l'immagine di una funzione \(g_S\) totale e computabile (detta generatrice); \\ quindi, \(S=I_{g_S}=\{g_S(0), g_s(1), g_S(2),...\}\)
    \end{itemize}
  \end{itemize}

  Si consideri ora il seguente teorema:
  \begin{theorem}
    Per ogni insieme S, se \(i \in S\) implica che \(f_i\) sia totale e se per ogni funzione \(f\) totale e computabile, esiste \(i\in S\;|\;f=f_i\), allora S non è ricorsivamente enumerabile. 
  \end{theorem}

  Informalmente, questo teorema stabilisce che tutte le funzioni totali computabili non sono ricorsivamente enumerabili (mentre le funzioni parziali computabili lo sono). Dunque, tale teorema afferma implicitamente che non esiste nessun formalismo ricorsivamente enumerabile in grado di definire tutte e sole le funzioni totali e computabili: infatti, gli FSA sono in grado di definire le funzioni totali, ma non tutte, le TM definiscono tutte le funzioni computabili, ma anche quelle non totali, e un linguaggio di programmazione (come il C) è in grado di definire tutti gli algoritmi, ma anche quelli che non terminano mai. 
  
  Si cerca quindi di comprendere se sia possibile eliminare le funzioni non totali: per far ciò, si prenda in considerazione una generica funzione parziale, ad esempio, arricchendo \(\mathbb{N}\) con il valore \(\{\bot\}\) o con qualsiasi altro simbolo che indichi che la funzione non è definita per certi valori. Tale trasformazione da funzione parziale a totale, però, non può essere applicata perchè nel passaggio è possibile perdere la computabilità della funzione. Questo risultato è enunciato nel seguente teorema:
  
  \begin{theorem}
    Non esiste una funzione totale e computabile h che sia un'estensione della seguente funzione:
    \(g(x)=if\;f_x(x)\neq\bot\;then\;f_x(x)+1\;else\;\bot\)
  \end{theorem}

  Tale teorema, afferma quindi che non è possibile estendere una funzione parziale ad una totale, in quanto si potrebbe perdere la sua computabilità.

  Vale anche il seguente risultato:
  \begin{theorem}
    Un insieme S è ricorsivamente enumerabile se e solo se \(S=D_h\), in cui h è una funzione parziale e computabile (\(S=\{x\;|\;h(x)\neq \bot\}\)), oppure se e solo se \(S=I_g\), in cui g è una funzione parziale e computabile (\(S=\{x\;|\;x=g(y), y\in \mathbb{N}\}\)).
  \end{theorem}

  Quindi, dato l'insieme \(K=\{x\;|\;f(x) \neq \bot\}\) questo è semidecidibile perchè \(K=D_h\), con \(h(x)=f_x(x)\), ma è anche indecidibile in quanto la funzione caratteristica dell'insieme \(K\), definita come \(c_K(x)=if\;f_x(x)\neq \bot\;then\;1\;else\;0\), non è computabile. Si è appena dimostrato che esistono insiemi che sono semidecidibili, ma allo stesso tempo indecidibili.

  \begin{figure}[!h]
    \begin{center}    
      \begin{tikzpicture}
        \foreach \X [count=\Y starting from 0.75] 
        in {insiemi ricorsivi, insiemi RE*, \(\wp(\mathbb{N})\)} {
          \draw (-\Y,-\Y/2) circle ({2*\Y} and \Y);
          \node at (1-2*\Y,-1.1*\Y) {\X}; 
        }
      \end{tikzpicture}
    \end{center}
    \caption{Gerarchia degli insiemi}    
  \end{figure}

  *Con RE si indicano gli insiemi Ricorsivamente Enumerabili.

  Si noti che tutte le inclusioni sono strette.

  \section{Teoremi di Kleene e Rice}

  \begin{theorem} [Teorema di Kleene del punto fisso]
    Sia t una qualunque funzione totale e computabile. Allora è sempre possibile trovare un intero p, tale per cui:
    
    \(f_p=f_{t(p)}\)\\
    La funzione \(f_p\) è detta punto fisso di t, perchè t trasforma \(f_p\) in \(f_p\) stessa.
  \end{theorem}

  \begin{theorem}[Teorema di Rice]
    Sia F un insieme generico di funzioni computabili. L'insieme \(S=\{x\;|\;f_x\in F\}\) degli indici delle TM che calcolano le funzioni di F, è ricorsivo se e solo se \(F=\emptyset\) oppure F è l'insieme di tutte le funzioni computabili.
  \end{theorem}

  Il teorema di Rice ha un forte impatto pratico negativo, in quanto afferma che, in tutti i casi non banali, \(S\) non è decidibile. Non è quindi possibile stabilire algoritmicamente se un dato algoritmo sia in grado di risolvere un determinato problema, nè se due programmi siano equivalenti (ossia se calcolino la stessa funzione). Il grande impatto pratico del teorema di Rice deriva dal fatto che il concetto di sottoinsieme \(F\) di funzioni computabili è un'espressione formale del concetto generale di proprietà di problemi risolvibili: una proprietà degli elementi di un insieme è un sottoinsieme dell'insieme dato e una funzione computabile è una formalizzazione del concetto di problema risolvibile; quindi, il teorema di Rice afferma che non è possibile stabilire se un determinato algoritmo risolve un problema pur risolvibile che godi di una qualsiasi proprietà non banale.

  \part{Algoritmi e Strutture Dati}
  \chapter{Linguaggi Formali}

  \section{Alfabeti e Stringhe}
  Di seguito sono riportate alcune definizioni importanti riguardanti i linguaggi formali:
  \begin{definition}[Alfabeto]
    Un alfabeto (o vocabolario) V è un insieme finito di oggetti elementari detti simboli (o caratteri)
  \end{definition} 

  \begin{definition} [Stringa]
    Una stringa x appartenente ad un alfabeto V è una sequenza finita e ordinata di caratteri appartenenti a tale alfabeto.
  \end{definition}

  Data una stringa \(x\), la notazione \(|\,x\,|\) indica la lunghezza di tale stringa, ovvero il numero di caratteri di cui è composta (la cardinalità del suo dominio \(V\)). Per convenzione la stringa vuota, che non contiene nessun carattere, è indicata con la lettera \(\varepsilon\). La stringa vuota \(\varepsilon\) ha lunghezza zero. 

  Per poter confrontare due stringhe è necessario verificare che:
  \begin{itemize}
    \item \(|x| = |y|\)
    \item \(x_i = y_i\; \forall i=0,1,2,...,|x|\)
  \end{itemize}

  La concatenazione (o prodotto) \(x.y\) di due stringhe \(x\) e \(y\) è una stringa composta da tutti i caratteri di \(x\) seguiti da tutti i caratteri di \(y\). Inoltre, la concatenazione \(x.\varepsilon\) produce come risultato la stringa \(x\).

  Data una stringa \(s\), una stringa \(x\) è sottostringa (o fattore) di \(s\) se esistono due stringhe \(y\) e \(z\), tali che \(s=yxz\). Inoltre:
  \begin{itemize}
    \item se \(y=\varepsilon\), \(x\) è detta prefisso.
    \item se \(z=\varepsilon\), \(x\) è detta suffisso.
    \item se \(z=y=\varepsilon\), \(x=s\)
  \end{itemize}

  Data una stringa \(x\), l'espressione \(x^i\) indica la concatenazione della stringa \(x\) con sè stessa per \(i-1\) volte, ovvero \(x^0 = \varepsilon\) e \(x^{i+1} = x.x^{i}\).

  \section{Operatori di Kleene}
  Per poter proseguire con il trattato, è prima necessario introdurre alcuni concetti matematici fondamentali. Si danno quindi le seguenti definizioni:
  \begin{definition}[Semigruppo] \label{Semigruppo}
    Un semigruppo è una coppia \(<S, \circ >\), dove:
    \begin{itemize}
      \item \(S\) è un insieme chiuso rispetto a \(\circ\); per cui, se si prendono due qualsiasi elementi \(A\) e \(B\) di tale insieme, l'operazione \(A\circ B\) produce come risultato un elemento appartenente ad \(S\);
      \item \(\circ\) è un'operazione associativa su \(S\).
    \end{itemize}
  \end{definition}

  Nel contesto dei linguaggi, l'operatore \(\circ\) rappresenta la concatenazione di stringhe.
  
  \begin{definition}[Monoide] \label{Monoide}
    Un monoide è un semigruppo in cui è definito un elemento unitario \(u\in S\), tale che \(\forall x, \exists u(x\circ u = x)\).
  \end{definition}

  Nel contesto dei linguaggi, l'elemento unitario \(u\) rappresenta la stringa vuota \(\varepsilon\): infatti, la concatenazione di una generica stringa \(x\) con la stringa vuota \(\varepsilon\) produce come risultato nuovamente la stringa originaria \(x\). 

  \begin{definition}[Gruppo] \label{Gruppo}
    Un gruppo è un monoide in cui è definito un elemento inverso \(x^{-1}\) unico per ogni elemento \(x\) dell'insieme \(S\), tale che \(\forall x(x\circ x^{-1} = u)\).
  \end{definition}

  Date le definizioni \ref{Semigruppo}, \ref{Monoide} e \ref{Gruppo}, si possono definire gli operatori di Kleene:

  \begin{definition} [Più di Kleene]
    Sia \(<S, \circ>\) un semigruppo. Per ogni \(X\subseteq S\), \(S^+\) indica il sottoinsieme di S generato da X, ovvero l'insieme di tutti gli elementi \(s\in S\) tale per cui \(s=x_1\circ x_2\circ...\circ x_n\) per qualche \(n\geq 1\), con \(x_i\in X\) (stringhe non vuote).
  \end{definition}

  Nel contesto dei linguaggi, l'operatore + di Kleene rappresenta l'insieme infinito di stringhe non vuote, che si possono generare a partire dall'insieme \(S\) di simboli, a cui si applica tale operatore.

  \begin{definition} [Stella di Kleene]
    Se \(<S, \circ, u>\) è un monoide, allora \(X^*=X^+\cup\{u\}\) è un monoide (più precisamente un sottomonoide) di S ed è detto monoide libero generato da X.
  \end{definition} 

  Nel contesto dei linguaggi, la stella di Kleene rappresenta l'insieme infinito di stringhe, inclusa la stringa vuota \(\varepsilon\), che si possono generare a partire dall'insieme \(S\) di simboli, a cui si applica tale operatore. Quindi, \(X^*=X^+\cup \{\varepsilon\}\).

  Ad esempio, dato l'insieme di simboli \(S=\{a,b,c\}\), \(S^+=\{a,b,c,aa,ab,ac,ba,bb,bc,ca,cb,cc,aaa...\}\) e \(S^*=\{\varepsilon,a,b,c,aa,ab,ac,ba,bb,bc,ca,cb,cc,aaa...\}\). 

  \section{Linguaggi} 
  Data la definizione della stella di Kleene, si può ora dare la definizione di linguaggio:
  \begin{definition} [Linguaggio]
    Un linguaggio L su un alfabeto V è un sottoinsieme di \(V^*\).
  \end{definition}
  Dato un insieme di simboli, si possono generare infiniti linguaggi.
  
  \vspace{10pt}

  \noindent 
  Poichè i linguaggi sono un inisieme di stringhe, valgono tutte le operazioni insiemistiche come:
  \begin{itemize}
    \item Unione (\(L_1\cup L_2\)): l'inisieme di tutte le stringhe che appartengono o ad \(L_1\) o ad \(L_2\) o ad entrambi i linguaggi;
    \item Intersezione (\(L_1 \cap L_2\)): l'insieme di tutte le stringhe che appartengono sia ad \(L_1\) che ad \(L_2\);
    \item Differenza (\(L_1\backslash L_2\)): l'insieme di tutte le stringhe che appartengono ad \(L_1\) ma non ad \(L_2\);
    \item Complementare (\(L^C = A^* \backslash L\)): l'insieme di tutte le stringhe che non appartengono al linguaggio \(L\);
    \item Concatenazione (\(L_1.L_2\)): l'insieme di tutte le stringhe che si ottengono concatenando ad ogni stringa di \(L_1\) ogni stringa di \(L_2\); formalmente \(L_1.L_2=\{xy:x\in L_1, \, y\in L_2\}\);
    \item Potenza \(n\)-esima (\(L^n\)): l'insieme di tutte le stringhe che si ottengono concatenando \(L\) con sè stesso \(n\) volte, utilizzando le regole della concatenazione precedentemente definite;
    \item Chiusura di Kleene \(\biggl(\displaystyle L^*=\bigcup_{n=0}^\infty L^n\) e \(\displaystyle L^+=\bigcup_{n=1}^\infty L^n\biggr)\)
  \end{itemize}
  Le operazioni su di un determinato linguaggio crea nuove classi di linguaggi con caratteristiche proprie, talvolta interessanti. Un linguaggio diventa di interesse nel momento in cui le stringhe di cui è composto possono essere utilizzate per veicolare informazioni, problemi, soluzioni o per rappresentare programmi, documenti, elementi multimediali o, nel caso più rilevante, per rappresentare computazioni.
  \chapter{Algoritmi}
In questo capitolo si analizzano a fondo i principali algoritmi di ordinamento e i relativi tempi di esecuzione. Nello specifico, si utilizzerà come modello di riferimento la macchina RAM con un criterio di costo costante, come analizzato nei capitoli precedenti. Prima di proseguire nella trattazione è necessario dare una definizione generale di algoritmo:

\begin{definition}
  Un algoritmo è una procedura di calcolo ben definita che prende un certo valore, o un insieme di valori, in input e genera un valore, o un insieme di valori, in output. Dunque, un algoritmo è una serie di passi computazionali che trasformano l'input in output.
\end{definition}

Un algoritmo può anche essere visto come uno strumento per la risoluzione di un problema computazionale ben definito: sotto questo sguardo, un algoritmo si definisce corretto se, per ogni istanza di input, termina con l'output corretto. Se un algoritmo è corretto, allora risolve quel determinato problema computazionale. Esistono molti modi per poter specificare un determinato algoritmo: si può utilizzare la lingua italiana o inglese, ma anche un linguaggio di programmazione come C, C++, JAVA e Pascal, o ancora tramite uno pseudocodice.

\section{Pseudocodifica}
La pseudocodifica può avvenire in molti modi, ma nel seguito si utilizzeranno le convenzioni qui riportate:
\begin{itemize}
  \item L'indentazione serve ad indicare la struttura a blocchi dello pesudocodice, in modo da comprendere quali istruzioni appartengono, per esempio, ad un ciclo \code{for}, a un ciclo \code{while} o ad un \code{if}-\code{else} statement. Non sono utilizzate le parentesi graffe o parole chiave come begin ed end in quanto appesantiscono la sintassi;
  \item I costrutti iterativi \code{while}, \code{for}, \code{repeat-until} e il costrutto condizionale \code{if-else} hanno interpretazioni simili a quelle dei comuni linguaggi di programmazione. Il contatore del ciclo mantiene il suo valore dopo la fine del ciclo, quindi il valore che ha provocato la terminazione del ciclo stesso. Inoltre, si utilizza la parola chiave \code{to} quando il ciclo \code{for} incrementa il valore del suo contatore ad ogni iterazione, mentre si utilizza la parola chiave \code{down to} nel caso la variabile venga decrementata;
  \item Le assegnazioni di un valore ad una certa variabile avviene con il simbolo \code{:=}, differente dall'operatore \code{=}, che invece indica l'eguaglianza di due valori all'interno di un costrutto \code{if};
  \item Per identificare un elemento appartenente ad un array, si utilizza la notazione con le parentesi quadre, al cui interno si indica l'indice dell'elemento a cui si vuole accedere: \code{array[i]}; Per indicare un intervallo di valori all'interno dell'array si utilizza la seguente sintassi: \code{array[i..j]}, con cui si indica la sottomatrice composta dagli elementi compresi fra \(i\) e \(j\); 
  \item I dati utilizzati sono tipicamente organizzati in oggetti, formati da attributi, a cui si accede tramite la notazione punto: \code{oggetto.prop}. Le variabili che rappresentano un determinato oggetto sono trattate come puntatori a tale oggetto. Un puntatore che non fa riferimento ad alcun oggetto è inizializzato con il valore \code{NIL};
  \item I parametri vengono passati ad una procedura per valore: la procedura chiamata riceve una sua copia dei parametri e, quindi, se a una di queste variabili è assegnato un nuovo valore, la modifica non è visibile dalla procedura chiamante. Nel caso venga passato come argomento un oggetto, viene copiato il puntatore a tale oggetto e quindi le modifiche sono visibili anche dalla procedura chiamante;
  \item L'istruzione \code{return} restituisce immediatamente il controllo al punto in cui la procedura chiamante ha effettuato la chiamata. Le istruzioni \code{return} possono anche ritornare un valore al chiamante;
  \item Gli operatori booleani \code{and} e \code{or} sono cortocircuitati. Ciò significa che nella valutazione dell'espressione \code{x and y}, si valuta prima se il valore di \code{x} sia falso, in quanto, se lo fosse, l'intera espressione sarebbe falsa e non avrebbe quindi alcun senso valutare il valore della variabile \code{y}. Al contrario, nella valutazione dell'espressione \code{x or y}, si verifica innanzitutto se il valore di \code{x} sia vero, in quanto, se lo fosse, l'intera espressione sarebbe vera e non avrebbe quindi alcun senso valutare il valore della variabile \code{y}.
\end{itemize}

Tramite queste regole è possibile definire un generico algoritmo.

\section{Insertion Sort}
Una classe di algoritmi molto studiati è quella riguardante l'ordinamento di un vettore, che consiste nella disposizione dei suoi elementi in ordine crescente.

\vspace{10pt}

Il primo algoritmo analizzato è l'\textbf{insertion sort}, che prende in input una sequenza di \(n\) numeri \([a_1, a_2, ...,a_n]\) e restituisce in output una permutazione \([a_1', a_2',...,a_n']\) tale che \(a_1'\le a_2' \le ... \le a_n'\). Questo algoritmo ordina sul posto \footnote{L'algoritmo risistema gli elementi della sequenza all'interno dell'array avendo, in ogni istante, al più un numero finito di elementi memorizzati all'esterno dell'array: ciò permette di risparmiare memoria nel calcolatore.} gli elementi assumendo che la sequenza da ordinare sia inizialmente partizionata in una sottosequenza già ordinata, all'inizio composta da un unico elemento (il primo dell'array), e una sottosequenza ancora da ordinare. Ad ogni iterazione viene rimosso un elemento dalla sottosequenza non ordinata e inserita nella posizione corretta all'interno della sottosequenza già ordinata. 

In pseudocodice:

\lstinputlisting{../docs/algorithms/insertion_sort.txt}

All'inizio di ogni iterazione del ciclo \code{for}, il cui indice è \(j\), la sottosequenza di elementi \code{A[1..j-1]} è la parte ordinata dell'array, mentre la sottosequenza \code{A[j+1..n]} è costituita da elementi ancora da ordinare.

\vspace{10pt}

Si analizza ora il tempo di esecuzione della procedura \code{insertion sort}: per ogni \(j=2,3,...,n\) in cui \(n\) = \code{A.length}, si indica con \(t_j\) il numero di volte che il test del ciclo \code{while} nella riga 5 viene eseguito per quel determinato valore di \(j\).

\begin{table}[!h]
  \centering
  \begin{tabular}{l l l}
    Codice & Costo & Numero di volte \\
    \hline
    \code{for j:= 2 to A.length} & \(c_1\) & \(n\) \\
    \code{key := A[j]} & \(c_2\) & \(n-1\) \\
    \code{i := j - 1} & \(c_3\) & \(n-1\) \\
    \code{while i > 0 and A[i] > key} & \(c_4\) & \(\sum_{j=2}^n{t_j}\) \\
    \code{A[i + 1] := A[i]} & \(c_5\) & \(\sum_{j=2}^n{(t_j-1)}\) \\
    \code{i := i - 1} & \(c_6\) & \(\sum_{j=2}^n{(t_j-1)}\) \\
    \code{A[i + 1] := key} & \(c_7\) & \(n-1\) \\
  \end{tabular}
  
\end{table}


Ad ogni riga di codice viene associato un costo \(c_i\) che va moltiplicato per il numero di volte che tale riga viene eseguita. Il tempo totale di esecuzione si calcola, dunque, sommando i vari contributi di tempo di ogni riga, ottenendo così l'espressione di \(T(n)\):

\begin{equation*}
  \displaystyle T(n) = c_1n+c_2(n-1)+c_3(n-1)+c_4\sum_{j=2}^n{t_j}+c_5\sum_{j=2}^n(t_j-1)+c_6\sum_{j=2}^n(t_j-1)+c_7(n-1)
\end{equation*}

Ovviamente, il caso migliore si verifica quando l'array in input è già ordinato. In questo caso, \(t_j = 1 \;\; \forall j=2,3...,n\) e l'espressione di \(T(n)\) assume la forma:

\begin{equation*}
  T(n) = (c_1+c_2+c_3+c_4+c_7)n - (c_2+c_3+c_4+c_7)
\end{equation*}

che è funzione lineare di \(n\). Dunque, \(T(n)=\Theta(n)\).

Al contrario, il caso pessimo si verifica quando l'array in input è ordinato, ma in ordine decrescente. In questo caso \(t_j = j \;\; \forall j=2,3...,n\) e l'espressione di \(T(n)\) assume la forma:

\begin{equation*}
  T(n) = \frac{1}{2}(c_4+c_5+c_6)n^2+(c_1+c_2+c_3)n+\frac{1}{2}(c_4-c_5-c_6+c_8)n-(c_2+c_3+c_4+c_7)
\end{equation*}

che è funzione quadratica di \(n\). Dunque, \(T(n)=\Theta(n^2)\).

\section{Merge Sort}
L'algoritmo appena analizzato utilizza un approccio di tipo incrementale: dopo aver ordinato il sottoarray \code{A[1..j-1]} inserisce l'elemento \code{A[j]} nella posizione corretta, ottenendo il sottoarray ordinato \code{A[1..j]}. Nel seguito, invece, si analizza un secondo approccio, più efficiente del primo, soprattutto per array di molti elementi: Divide et Impera. Questo criterio si basa sulla suddivisione ricorsiva del problema in sottoproblemi più piccoli, simili a quello originario, ma di dimensione ridotta, per poi risolvere i sottoproblemi di dimensione minima e fondere i risultati ottenuti, per costruire una soluzione generale del problema originario. 

Il paradigma Divide et Impera, si basa in realtà su tre passaggi:
\begin{enumerate}
  \item Divide: il problema viene suddiviso in un certo numero di sottoproblemi, che sono istanze più piccole del problema originario, fino ad ottenere sottoproblemi minimi, non più divisibili;
  \item Impera: i sottoproblemi di dimensione minima vengono risolti in maniera ricorsiva; se i problemi hanno dimensione sufficientemente piccola vengono risolti direttamente;
  \item Combina: le soluzioni dei sottoproblemi vengono combinate per generare la soluzione del problema generale.
\end{enumerate}

Un tipico algoritmo che segue questo metodo di risoluzione è il \textbf{merge sort}, che suddivide l'array originario a metà e ordina ricorsivamente i due sottoarray ottenuti, chiamando sè stesso fino ad ottenere sequenze di dimensione uno, di per sè già ordinate. A questo punto, le sottosequenze vengono fuse in modo da ottenere un array ordinato. 

Quest'ultimo passaggio viene effettuato tramite una procedura ausiliaria \code{merge(A,p,q,r)}, dove \(A\) è un array, e \(p,q,r\) sono tre indici dell'array tali che \(p\le q < r\).
La procedura assume che le sottosequenze \code{A[p..q]} e \code{A[q+1..r]} siano ordinate e, quindi, le fonde per formare un unico sottoarray ordinato che sostituisce il sottoarray corrente \code{A[p..r]}. La procedura \code{merge(A,p,q,r)} impiega un tempo \(\Theta(n)\) con \(n=r-p+1\) il numero di elementi da fondere. Ad ogni iterazione, la procedura \code{merge} confronta gli elementi più piccoli dei due sottoarray, inserendoli nel sottoarray "successivo" fino a quando uno dei due sottoarray è vuoto: a quel punto, i restanti elementi del sottoarray rimanente vengono copiati per completare l'array "successivo". Da un punto di vista computazionale, ogni iterazione della procedura impiega un tempo costante, in quanto deve semplicemente confrontare i due elementi dei due sottoarray. Poichè tale procedura viene effettuata per un massimo di \(n\) volte, la fusione impiega un tempo \(\Theta(n)\).

In pseudocodice:

\lstinputlisting[mathescape=true]{../docs/algorithms/merge.txt}

\noindent
In altri termini, le righe 2 e 3 inizializzano i valori di \(n_1\) ed \(n_2\), che rappresentano la lunghezza dei due sottoarray \code{A[p..q]} e \code{A[q+1..r]}. Nelle due righe successive vengono creati i due sottoarray ausiliari \code{L} (per Left) ed \code{R} (per Right), che contano \(n+1\) elementi (per motivi che verranno chiariti a breve). Le righe dalla 6 alla 9, inizializzano gli array appena creati con i valori contenuti rispettivamente nella prima e nella seconda metà dell'array \code{A}. Le righe 10 e 11 inizializzano l'ultimo (\(n+1\) -esimo) elemento dei due sottoarray \code{L} ed \code{R}, con un valore sentinella. Impostando tale valore ad \(\infty\), si è certi che non possa essere il valore più piccolo fra i due confrontati: in questo modo, una volta arrivati alla fine di uno dei due sottoarray, gli elementi dell'altro vengono ricopiati nell'array "successivo" in quanto necessariamente più piccoli di \(\infty\). Le ultime righe (dalla 12 alla 20) implementano la logica del confronto e dell'inserimento dell'elemento correntemente più piccolo nell'array \code{A}.

\vspace{10pt}

Una volta analizzata la procedura \code{merge}, si può introdurre l'algoritmo di ordinamento \code{mergeSort}. In pseudocodice:

\lstinputlisting[mathescape=true]{../docs/algorithms/merge_sort.txt}

L'algoritmo calcola, in riga 2, un indice \code{q}, che serve a suddividere l'array \code{A} in due sottoarray che contengono rispettivamente \(\lceil n/2 \rceil\) elementi ed \(\lfloor n/2 \rfloor\) elementi, su cui richiama ricorsivamente sè stessa. Una volta suddiviso l'array \code{A} in sottoarray di dimensione minima, viene chiamata la procedura \code{merge}, precedentemente analizzata. 

Come si può facilmente osservare, la procedura \code{mergeSort} è definita in maniera ricorsiva, quindi l'analisi delle prestazioni temporali diventa leggermente più complessa: infatti, si deve necessariamente far uso di un'equazione di ricorrenza, che esprime il tempo di esecuzione totale di un problema di dimensione \(n\), in funzione del tempo di esecuzione per input più piccoli. Se la dimensione del problema diventa sufficientemente piccola, per esempio \(n\le c\) per qualche costante \(c\), la soluzione del problema è diretta e richiede un tempo di esecuzione costante, indicata con \(\Theta(1)\). Si suppone, inoltre, che il problema originario venga suddiviso in \(a\) sottoproblemi, tutti di dimensione \(1/b\) volte la dimensione del problema originario. Dunque, è necessario un tempo \(T(n/b)\) per risolvere un sottoproblema di dimensione \(n/b\) e un tempo \(aT(n/b)\) per risolverli tutti. Infine, se si impiega un tempo \(D(n)\) per suddividere il problema in \(a\) sottoproblemi e un tempo \(C(n)\) per fonderne le soluzioni, si ottiene la ricorrenza:

\begin{equation*}
  T(n) = \begin{cases}
    \Theta(1) & if\;n\le c\\
    aT(n/b)+D(n)+C(n) & else
  \end{cases}
\end{equation*}

Per trovare ora il tempo di esecuzione \(T(n)\) nel caso peggiore si può ragionare come segue. Nel caso in cui i sottoarray abbiano cardinalità uno, la soluzione è diretta, quindi viene impiegato un tempo costante per risolvere il problema, mentre se i sottoarray hanno \(n > 1\) elementi, si suddivide il tempo di esecuzione impostando \(D(n) = \Theta(1)\), in quanto si impiega un tempo costante per calcolare il centro di un array, \(C(n)=\Theta(n)\), in quanto si è già precedentemente dimostrato che la procedura \code{merge} impieghi un tempo lineare per la fusione delle soluzioni, e infine si pone \(a=b=2\), in quanto si suddivide ricorsivamente il problema in due sottoproblemi di uguale dimensione \footnote{In realtà, sarebbe più accurato scrivere \(T(\lfloor n/2 \rfloor) + T(\lceil n/2 \rceil)\) in quanto non sempre la dimensione dell'array \code{A} è potenza di 2 e, dunque, divisibile ricorsivamente in due metà. Tale approssimazione, comunque, non influisce sulla complessità finale del calcolo.}. Con questo ragionamento, la ricorrenza assume l'espressione:

\begin{equation*}
  T(n)=\begin{cases}
    \Theta(1) & if\; n=1\\
    2T(n/2)+\Theta(n)+\Theta(1) & if\; n>1
  \end{cases}
\end{equation*}

Si può facilmente dimostrare (analiticamente oppure tramite il teorema dell'espreto, di cui si discuterà successivamente) che tale equazione ha soluzione \(T(n)=\Theta(n\,log_2\,n)\), che rappresenta il tempo di esecuzione dell'algoritmo \code{mergeSort} nel caso pessimo. Si può osservare come tale algoritmo sia decisamente migliore rispetto all'\code{insertionSort}, il cui tempo di esecuzione nel caso pessimo è \(\Theta(n^2)\).

Un modo per comprendere meglio come mai la complessità temporale del \code{mergeSort} sia proprio \(\Theta(n\,log_2\,n)\), si riscrive la ricorrenza nel seguente modo:

\begin{equation*}
  T(n)=\begin{cases}
    c & if\; n=1\\
    2T(n/2)+cn+c & if\; n>1
  \end{cases}
\end{equation*}

in cui la costante \(c\) rappresenta sia il tempo richiesto per risolvere i problemi di dimensione 1, sia il tempo per elemento dell'array dei passi divide e combina. Si può costruire un albero di ricorsione, in cui ogni ramo rappresenta una metà dell'array precedente e ogni foglia sia un array di dimensione unitaria. Il primo livello (in alto) ha un costo totale di \(cn\), il secondo livello ha un costo totale di \(cn/2 + cn/2 = cn\) e così via fino all'ultimo livello, con costo totale di \(n + n +...+ n\) (\(c\) volte), quindi di \(cn\). In generale, il livello \(i\) ha \(2^i\) nodi, ciascuno dei quali ha un costo di \(c(n/2^i)\), quindi, il numero totale di livelli dell'albero di ricorsione è \(log_2\,n+1\), con \(n\) la dimensione dell'input. Dunque, per calcolare il costo totale, basta sommare i costi di tutti i livelli, ottenendo \(cn(log_2\,n+1) = cn(log_2\,n)+cn\), ovvero \(\Theta(n\,log_2\,n)\).

\section{Risoluzione Ricorrenze}
Come già detto in precedenza, quando i problemi sono abbastanza grandi da essere risolti ricorsivamente, si ha il cosiddetto caso ricorsivo, tramite cui si divide il problema in problemi più piccoli di uguale natura. Una volta che i sottoproblemi diventano sufficientemente piccoli da non richiedere più il passo ricorsivo, si è raggiunto il cosiddetto caso base, da cui inizia la soluzione del problema. 

Questo modello di risoluzione del problema viene anche detto Divide et Impera e richiede l'utilizzo di equazioni di ricorrenza, tramite cui si caratterizzano i tempi di esecuzione degli algoritmi in termini dei loro valori con input più piccoli. Per risolvere tali equazioni, ovvero per trovare i limiti asintotici \(\Theta\) oppure \(O\), esistono tre metodi:
\begin{enumerate}
  \item Metodo di Sostituzione: si fa un'ipotesi di soluzione e si utilizza l'induzione matematica per dimostrare che l'ipotesi sia corretta;
  \item Metodo dell'Albero di ricorsione: si converte la ricorrenza in una struttura ad albero, i cui nodi rappresentano i costi ai vari livelli della ricorsione;
  \item Metodo dell'Esperto (Master theorem): fornisce i limiti per ricorrenze nella forma \\ \(T(n)=aT(n/b)+f(n)\) con \(a\ge 1, b>1\) e \(f(n)\) data. Una ricorrenza in questa forma caratterizza un algoritmo divide et impera che crea \(a\) sottoproblemi di dimensione \(1/b\), i cui passi divide e combina richiedono un tempo \(f(n)\).
\end{enumerate}

A volte, le ricorrenze non saranno delle uguaglianze, ma delle disuguaglianze nella forma \(T(n) \le ...\), che stabilisce un limite superiore su \(T(n)\) (quindi si utilizza la notazione \(O\) anzichè \(\Theta\)), oppure nella forma \(T(n) \ge ...\), che stabilisce invece un limite inferiore su \(T(n)\) (quindi si utilizza la notazione \(\Omega\) anzichè \(\Theta\)). Inoltre, ci sono casi in cui si trascurano dei dettagli tecnici di poca importanza, come le condizioni al contorno: infatti, poichè il tempo di esecuzione di un algoritmo con un input di dimensione costante è costante, le ricorrenze che ne derivano hanno \(T(n)=\Theta(1)\), per valori sufficientemente piccoli di \(n\). Questa decisione risiede nel fatto che, sebbene le condizioni al contorno cambino la soluzione esatta della ricorrenza, tuttavia la soluzione non cambia per più di un fattore costante e quindi asintoticamente rimane immutata.

\subsection{Metodo di Sostituzione}
Uno dei metodi per la risoluzione delle occorrenze e, quindi, per il calcolo del tempo di esecuzione degli algoritmi, è il metodo della sostituzione, che richiede due passaggi:
\begin{enumerate}
  \item Ipotizzare la forma della soluzione;
  \item Utilizzare l'induzione matematica per dimostrare che la soluzione ipotizzata sia corretta.
\end{enumerate}
Questo metodo può essere applicato solamente se si ha un'idea della forma generale della soluzione e si vuole calcolare il limite superiore o inferiore della ricorrenza che si analizza.

\textit{ESEMPIO:} Si determini il limite superiore della ricorrenza \(T(n)=2T(\lfloor n/2 \rfloor)+n\).

Si suppone che la soluzione sia \(O(n\,log_2\,n)\). Il metodo di sostituzione consiste nel dimostrare che \(T(n)\le c\,n\,log_2\,n\) per un generico \(c>0\). Si verifica, innanzitutto, che questo limite sia valido anche per \(\lfloor n/2 \rfloor\), ovvero che \(T(\lfloor n/2 \rfloor)\le c\,\lfloor n/2 \rfloor\,log_2(\lfloor n/2 \rfloor)\). Facendo le opportune sostituzioni si ha:  
\begin{flalign*}
  T(n)\;\;\; &\le \;\;\; 2(c\lfloor n/2 \rfloor log_2(\lfloor n/2 \rfloor)) + n &&\\
  &\le \;\;\; c\,n\,log_2(n/2)+n &&\\
  &= \;\;\; c\,n\,log_2\,n-c\,n\,log_2\,2+n &&\\
  &= \;\;\; c\,n\,log_2\,n-c\,n+n &&\\
  &\le \;\;\; c\,n\,log_2\,n &&
\end{flalign*}
L'ultimo passaggio è vero solo per \(c \ge 1\).
A questo punto, l'induzione matematica richiede di dimostrare che la soluzione vale per le condizioni al contorno. Si suppone, per esempio, che l'unica condizione al contorno sia \(T(1)=1\): si deve dimostrare che è possibile scegliere una costante \(c\) sufficientemente grande in modo che il limite \(T(n)\le c\,n\,log_2\,n\) sia valido anche per le condizioni al contorno. Quindi per \(n=1\) (condizione al contorno), il limite \(T(n)\le c\,n\,log_2\,n\) diventa \(T(1)\le c\, log_2\;1 = 0\), che però è in contrasto con \(T(1)=1\): il caso base della dimostrazione induttiva non è valido!

Questo ostacolo nella dimostrazione può essere facilmente superato sfruttando la notazione asintotica, che richiede di provare che \(T(n)\le c\,n\,log_2\,n\) sia valida solamente dopo un certo \(n_0\) in poi, scelto arbitrariamente: l'idea è quella di escludere la condizione al contorno dalla dimostrazione induttiva. Si osservi che, per \(n\ge 3\), la ricorrenza non dipende direttamente da \(T(1)\), quindi si può sostituire con \(T(2)\) e \(T(3)\), impiegati come casi base della dimostrazione induttiva. Inoltre, ponendo \(n_0=2\), se \(T(1)=1\) allora \(T(2)=4\) e \(T(3)=5\). Basta quindi determinare una costante \(c\) tale per cui \(T(2)= 4 \le 2c\,log_2(2)\) e \(T(3)= 5 \le 3c\,log_2(3)\): le precedenti condizioni sono soddisfatte solo per \(c\ge 2\).

\vspace*{10pt}

Non esiste un metodo unico e generale per indovinare la soluzione corretta di una ricorrenza, ma è possibile formulare delle buone ipotesi tramite il metodo dell'albero di ricorsione. Inoltre, se una ricorrenza è simile ad una   già risolta in precedenza, allora è possibile che anche la soluzione sia analoga. Un altro metodo per formulare un'ipotesi di soluzione consiste nel dimostrare dei limiti superiori e inferiori molto generali e larghi, per poi ridurre gradualmente il grado di incertezza, aumentando il limite inferiore e diminuendo il limite superiore.

Ci sono poi casi in cui la soluzione ipotizzata sembra essere corretta, ma i calcoli matematici non soddisfano il passo induttivo: solitamente, il problema risiede nel fatto che l'ipotesi induttiva non è abbastanza forte per dimostrare il limite esatto. In un caso del genere, spesso è necessario semplicemente correggere l'ipotesi sottraendo un termine di ordine inferiore per fare in modo che i calcoli soddisfino i requisiti. 

\textit{ESEMPIO:} Si calcoli la ricorrenza \(T(n)=T(\lfloor n/2 \rfloor)+T(\lceil n/2 \rceil)+1\) supponendo che la soluzione sia \(T(n)=O(n)\). Si deve quindi dimostrare che \(T(n)\le cn\) per qualche \(c\) arbitraria. Sostituendo l'ipotesi all'interno della ricorrenza si ottiene:
\begin{flalign*}
  T(n) \;\;\; &\le \;\;\; c\lfloor n/2\rfloor + c\,\lceil n/2 \rceil + 1 &&\\
  &=\;\;\; c\,n+1 &&
\end{flalign*}
che non implica che \(T(n)\le c\,n\) per qualunque valore di \(c\). Sembrerebbe quindi che l'ipotesi fatta sia sbagliata, ma al contrario si può dimostrare che è corretta, formulando un'ipotesi induttiva più forte. Per affrontare tale problema, si sottrae un termine di ordine inferiore dalla precedente ipotesi, ad esempio, un termine costante \(d\ge 0\), ottenendo come nuova ipotesi \(T(n)\le c\,n-d\), che sostituita alla ricorrenza:
\begin{flalign*}
  T(n)\;\;\; &\le \;\;\; (c\lfloor n/2 \rfloor -d)+ (c\lceil n/2 \rceil-d)+1 &&\\
  &= \;\;\; cn -2d +1 &&\\
  &\le\;\;\; cn-d
\end{flalign*}
che diventa valida per ogni \(d\ge 1\). Come prima, la costante \(c\) deve essere scelta arbitrariamente grande affinchè siano soddisfatte le condizioni al contorno.

\vspace*{10pt}

Infine, ci sono casi in cui tramite una piccola manipolazione algebrica è possibile rendere una ricorrenza ignota simile ad una più familiare.

\textit{ESEMPIO:} Si calcoli la ricorrenza \(T(n)=2T(\lfloor \sqrt{n} \rfloor)+ log_2(n)\). Tale ricorrenza sembra molto complessa da risolvere, ma è possibile semplificarla ponendo \(m=log_2n\), ottenendo così \\ \(T(2^m)=2T(2^{m/2})+m \). Chiamando \(S(m)\) la ricorrenza appena ottenuta, è possibile scrivere \(S(m)=2S(m/2)+m\), simile alla precedente ricorrenza analizzata \(T(n)=2T(\lfloor n/2 \rfloor)+n\); in effetti, la soluzione della ricorrenza \(S(m)\) è la stessa ottenuta in precedenza. Dunque, la soluzione è \(S(m)=m\,log_2\,m\) e, ripristinando i termini con la sostituzione \(m=log_2n\), si ottiene che \(T(n)=O(log_2n \cdot log_2(log_2n))\).

\subsection{Metodo dell'Albero di Ricorsione}
Dato che spesso è complesso formulare un'ipotesi di soluzione per una data ricorrenza, è possibile utilizzare il metodo dell'albero di ricorsione, in cui ogni nodo rappresenta il costo di un singolo sottoproblema. Sommando i costi dei nodi di ogni livello, si ottengono i costi relativi a quel livello e, sommando tali costi, si ottiene il costo generale della ricorrenza, che rappresenta l'ipotesi da verificare con il metodo della sostituzione. Utilizzando questo metodo, si tollera un certo livello di approssimazione, in quanto è interessante analizzare solamente il comportamento asintotico della ricorrenza: si possono quindi eliminare gli operatori 'ceil' e 'floor' e fare delle ipotesi blande per semplificare i calcoli.

\textit{ESEMPIO:} Si calcoli la ricorrenza \(T(n)=3T(\lfloor n/4 \rfloor)+\Theta(n^2)\). Come detto, si può approssimare la ricorrenza eliminando l'operatore floor, ottenendo \(T(n)=3T(n/4)+cn^2\), per una data costante \(c>0\). Per comodità, si suppone anche che \(n\) sia una potenza di 4, in modo tale che ogni livello dell'albero abbia dimensione intera. Si ottiene così il seguente albero delle ricorrenze:

\begin{figure}[!h]
  \centering
  \includegraphics[width=10cm]{albero ricorrenze.jpg}
  \caption{Albero della ricorrenza \(T(n)=3T(n/4)+cn^2\)}
\end{figure}

La parte \((a)\) della figura mostra \(T(n)\), che viene espanso nella parte \((b)\) in un albero equivalente che rappresenta la ricorrenza. Il termine \(cn^2\) nella radice di quest'albero rappresenta il costo al livello più alto della ricorsione, mentre i tre sottoalberi rappresentano i costi richiesti dai tre sottoproblemi di dimensione \(n/4\). La parte \((c)\) mostra l'espansione dei nodi di costo \(T(n/4)\) dalla parte \((b)\), in cui ogni nodo figlio ha costo \(c(n/4)^2\). Tale processo viene ripetuto più e più volte fino ad ottenere i casi base, rappresentati nella parte \((d)\) con \(T(1)\). 

La dimensione dei sottoproblemi per i nodi alla profondità \(i\) è di \(n/4^i\), quindi la dimensione del sottoproblema diventa 1 (dimensione delle foglie) quando \((n/4)^i=1\), ovvero quando \(i=log_4(n)\): dunque, l'albero della ricorrenza ha esattamente \(log_4n+1\) livelli. Ora, per determinare il costo di ogni livello, basti pensare che ogni nodo dell'albero genera tre sottonodi e che, dunque, il numero di nodi alla profondità \(i\) è \(3^i\). Moltiplicando il risultato appena ottenuto con il costo di un singolo nodo, si ottiene che ogni livello ha un costo di \(3^ic(n/4^i)^2 = (3/16)^icn^2\). L'ultimo livello dell'albero conta \(n^{log_43}\) nodi, ognuno di costo \(T(1)\), per un costo totale di \(n^{log_43}T(1)\), ovvero \(\Theta(n^{log_43})\).

A questo punto, si sommano i contributi di ogni livello, ottenendo:
\begin{flalign*}
  T(n)\;\;\; &= \;\;\; cn^2 + \frac{3}{16}cn^2 + \bigg(\frac{3}{16} \bigg)^2cn^2+...+\bigg(\frac{3}{16} \bigg)^{log_4n-1}cn^2+\Theta(n^{log_43}) &&\\
  &=\;\;\; \sum_{i=0}^{log_4n-1}\bigg(\frac{3}{16}\bigg)^icn^2+\Theta(n^{log_43}) &&\\
  &\overset{*}{<} \;\;\; \sum_{i=0}^{\infty}\bigg(\frac{3}{16}\bigg)^icn^2+\Theta(n^{log_43}) &&\\
  &= \;\;\; \frac{1}{1-(3/16)}cn^2 + \Theta(n^{log_43}) &&\\
  &= \;\;\; \frac{16}{13}cn^2 + \Theta(n^{log_43}) &&\\
  &= \;\;\; O(n^2).
\end{flalign*}

Il passaggio segnato con * rappresenta una piccola approssimazione: la \(\sum_{i=0}^{log_4n-1}\big(\frac{3}{16}\big)^icn^2\) ammette come limite superiore \(\sum_{i=0}^{\infty}\big(\frac{3}{16}\big)^icn^2\), che rappresenta una serie geometrica decrescente infinita. In questo modo è possibile proseguire con agilità i calcoli, ottenendo come ipotesi \(T(n)=O(n^2)\), che dovrà essere verificata con il metodo della sostituzione. 

\subsection{Metodo dell'Esperto}
Il metodo dell'esperto è impiegato per la risoluzione di ricorrenze del tipo \(T(n)=aT(n/b)+f(n)\), con \(a\ge 1, b>1\) costanti ed \(f(n)\) una funzione asintoticamente positiva. Una ricorrenza di questo tipo rappresenta il tempo di esecuzione di un algoritmo che divide il problema di dimensione \(n\) in \(a\) sottoproblemi di dimensione \(n/b\), mentre la funzione \(f(n)\) rappresenta il costo di divisione del problema e di combinazione delle soluzioni.
Il metodo dell'esperto dipende dal seguente teorema:

\begin{theorem}[Master Theorem]
  Date le costanti \(a\ge 1\), \(b>1\) e la funzione \(f(n)\), se la ricorsione \(T(n)\) si presenta nella forma \(T(n)=aT(n/b)+f(n)\), allora può essere limitata asintoticamente nei seguenti modi:
  \begin{enumerate}
    \item Se \(f(n)=O(n^{log_b a-\varepsilon})\) per qualche \(\varepsilon>0\), allora \(T(n)=\Theta(n^{log_b a})\);
    \item Se \(f(n)=\Theta(n^{log_b a})\), allora \(T(n)=\Theta(n^{log_b a}log_2(n))\);
    \item Se \(f(n)=\Omega(n^{log_b a +\varepsilon})\) per qualche \(\varepsilon>0\) e se \(af(n/b)\le cf(n)\) per qualche \(c<1\) e per ogni \(n\) sufficientemente grande, allora \(T(n)=\Theta(f(n))\).
  \end{enumerate}
\end{theorem}

Si osservi che in ciascuno dei tre casi, si confronta la funzione \(f(n)\) con la funzione \(n^{log_b a}\): intuitivamente, la soluzione della ricorrenza è determinata dalla funzione polinomialmente \footnote{Una funzione è polinomialmente più grande rispetto ad un'altra funzione se la prima è asintoticamente più grande della seconda di un fattore \(n^\varepsilon\) per qualche \(\varepsilon >0\).} più grande. Se la funzione \(n^{log_b a}\) è più grande polinomialmente, come nel caso uno, allora sarà soluzione della ricorrenza, altrimenti la soluzione sarà \(f(n)\), come enunciato nel caso tre, in cui si deve anche verificare la condizione di regolarità della funzione. Nel caso due, in cui le due funzioni sono asintoticamente uguali, si moltiplicano entrambi i membri per un fattore logaritmico e la soluzione sarà \(T(n)=\Theta(n^{log_b a}log_2(n)) = \Theta(f(n)log_2(n))\).

I tre casi, sfortunatamente, non coprono tutte le funzioni \(f(n)\) possibili, in quanto ci sarà un intervallo fra i casi 1 e 2, in cui la funzione \(f(n)\) è minore di \(n^{log_b a}\), ma non polinomialmente, mentre ci sarà anche un intervallo fra i casi 2 e 3, in cui la funzione \(f(n)\) è maggiore di \(n^{log_b a}\), ma non polinomialmente. In questi casi, il teorema dell'esperto non può essere applicato. 

Per utilizzare il teorema enunciato, bisogna semplicemente determinare in quali dei tre casi rientra la funzione \(f(n)\) e confrontarla con la funzione \(n^{log_b a}\).

\textit{ESEMPIO:} Si determini la soluzione della ricorrenza \(T(n)=9T(n/3)+n\). In questo caso, si ha che \(a=9, b=3\) ed \(f(n)=n\) e quindi \(n^{log_b a} = n^{log_3(9)}=\Theta(n^2)\). Dato che \(f(n)=O(n^{log_3(9)-\varepsilon})\), con \(\varepsilon = 1\) (in quanto \(f(n)=n\)), si può applicare il caso 1 del teorema dell'esperto e concludere immediatamente che la soluzione della ricorrenza è \(T(n)=\Theta(n^2)\), in quanto \(n^2\) è polinomialmente più grande di \(n\). 

\vspace*{10pt}

\textit{ESEMPIO:} Si determini la soluzione della ricorrenza \(T(n)=2T(n/2)+n\,log_2\,n\). In questo caso, si ha che \(a=2,b=2\) ed \(f(n)=n\,log_2\,n\) e quindi \(n^{log_b a}=n^{log_2(2)} = n\). Si potrebbe erroneamente pensare di essere nel terzo caso del teorema dell'esperto, ma le due funzioni non sono polinomialmente comparabili quindi non si può applicare il teorema. La ricorrenza, dunque, deve necessariamente essere risolta con l'utilizzo dei metodi precedentemente analizzati.

\section{Heap Sort}
Analizzando l'algoritmo Merge Sort si è constatato che è efficiente dal punto di vista temporale, ma non dal punto di vista spaziale, in quanto occupa una grande quantità di memoria. A questo proposito, si analizza ora l'algoritmo \textbf{Heap Sort}, che effettua un ordinamento sul posto degli elementi utilizzando una struttura dati detta Heap ('mucchio'), per la gestione delle informazioni.

Un Heap (binario) è una struttura dati ad albero binario quasi completo \footnote{Un albero binario quasi completo è una struttura dati ad albero in cui ogni livello è completo, eccetto per al più l'ultimo livello, che potrebbe essere completo solo fino ad un certo punto da sinistra}, in cui ogni nodo rappresenta un elemento dell'array da ordinare. Nello specifico, \code{A[1]} è la radice dell'albero, e per ogni elemento \code{A[i]}, \code{A[2i]} e \code{A[2i+1]} rappresentano i figli del nodo, mentre \lstinline[mathescape]{A[$\lfloor n/2 \rfloor$]} rappresenta il nodo padre. Si possono quindi definire le seguenti procedure:

\begin{lstlisting}[mathescape=true]
parent(i):
  return $\lfloor i/2 \rfloor$
\end{lstlisting}
\begin{lstlisting}[mathescape=true]
left(i):
  return $2i$
\end{lstlisting}
\begin{lstlisting}[mathescape]
right(i):
  return $2i+1$
\end{lstlisting}

Oltre all'attributo \code{A.length}, che ne ritorna la lunghezza, l'array \code{A} possiede in questo caso anche l'attributo \code{A.heapSize}, che indica il numero degli elementi dell'heap che sono registrati nell'array \code{A}. In altre parole, anche se l'array contiene \(n\) elementi, con \(n\)=\code{A.length}, soltanto gli elementi in \code{A[1..A.heapSize]}, con \(0\le\) \code{A.heapSize} \(\le\) \code{A.length}, sono elementi validi dell'heap.

Esistono, inoltre, due tipologie di heap binari: max-heap e min-heap. Il primo, più importante, è costruito in modo tale che ogni nodo rispetti la condizione per cui \lstinline[mathescape]{A[parent(i)] $\ge$ A[i]}; dunque, il valore di un nodo è al massimo il valore del nodo padre e, di conseguenza, l'elemento più grande di un max-heap è memorizzato alla sua radice. Il secondo, meno utilizzato, è costruito in modo tale che ogni nodo rispetti la condizione per cui \lstinline[mathescape]{A[parent(i)] $\le$ A[i]}. 

Per implementare l'algoritmo heapsort si fa utilizzo del max-heap; per poterne mantenere le proprietà si utilizza la procedura di supporto \code{maxHeapify}, un algoritmo che prende in input un array \code{A} e un suo indice \code{i}, e restituisce l'array ordinato in modo tale da rappresentare il max-heap. Quando tale procedura viene invocata, essa assume che gli alberi binari di radici \code{left(i)} e \code{right(i)} siano dei max-heap, ma assume anche che \code{A[i]} possa essere più piccolo dei suoi figli, violando la proprietà fondamentale. La procedura, quindi, ha il compito di far 'scendere' il valore \code{A[i]} in modo tale che il sottoalbero con radice di indice \(i\) diventi un max-heap.

\vspace{1in}

In pseudocodice:
\lstinputlisting[mathescape]{../docs/algorithms/max_heapify.txt}
 
A ogni passo viene determinato il più grande degli elementi \code{A[i]}, \code{A[left(i)]} e \code{A[right(i)]} e il suo indice viene memorizzato nella variabile \code{max}. Se \code{A[i]} è l'elemento più grande, allora il sottoalbero è già un max-heap e la procedura termina la propria esecuzione, altrimenti uno dei due figli contiene l'elemento più grande e \code{A[i]} viene scambiato con \code{A[max]} (\code{swap with}); in questo modo, il nodo di indice \(i\) e i suoi figli soddisfano la proprietà di max-heap. Il nodo con indice massimo, però, presenta il valore originale di \code{A[i]} e, quindi, il sottoalbero di radice \code{max} potrebbe violare la proprietà fondamentale: quindi, la procedura viene chiamata ricorsivamente sul sottoalbero, fino a raggiungere le foglie.

Questa procedura viene eseguita in un tempo \(O(h)\), con \(h\) l'altezza dell'albero. Essendo l'albero quasi completo, \(h=O(log\,n)\), quindi \(T(n)=O(log\,n)\).

\vspace*{10pt}

Ora, tramite la procedura \code{maxHeapify} è possibile convertire un array \code{A[1..n]} (con \(n\)=\code{A.length}) in un max-heap. Prima di procedere, è importante osservare come tutti gli elementi \lstinline[mathescape]{A[$\lfloor n/2 \rfloor + 1$ .. n]} siano foglie dell'albero e, quindi, ciascuno di essi è un heap di un solo elemento, che si può utilizzare come punto di partenza per la costruzione dell'heap. Si introduce, dunque, la procedura \code{buildMaxHeap}, che attraversa i nodi restanti dell'albero ed esegue la procedura \code{maxHeapify} in ciascuno di essi. 

In pseudocodice:
\lstinputlisting[mathescape]{../docs/algorithms/build_max_heap.txt}

Si può dimostrare che tale procedura impiega un tempo di esecuzione \(T(n)=O(n)\).

\vspace*{10pt}

A questo punto, è possibile scrivere l'algoritmo \code{heapSort}. In pseudocodice:

\lstinputlisting[mathescape]{../docs/algorithms/heap_sort.txt}

Questo algoritmo si basa sul fatto che, una volta riordinato l'array in maniera che rappresenti un max-heap, l'elemento più grande dell'array si trova in \code{A[1]}: questo elemento può quindi essere inserito nella posizione finale corretta scambiandolo con \code{A[n]}. Se ora si toglie il nodo \(n\) dall'heap, diminuendo \code{A.heapSize}, si nota che i figli della radice restano max-heap, ma la nuova radice potrebbe violare la proprietà del max-heap. Questo problema può essere rimosso chiamando la procedura \code{maxHeapify(A, 1)}, che lascia un max-heap in \code{A[1..n-1]}. Questa operazione viene ripetuta fino ad un heap di dimensione 2, già ordinato per definizione. 

Come detto in precedenza, la procedura \code{buildMaxHeap} impiega un tempo di esecuzione lineare (\(O(n)\)), mentre le \(n-1\) chiamate alla procedura \code{maxHeapify} impiegano ciascuna un tempo \(O(log\,n)\). Pertanto il tempo di esecuzione dell'\code{heapSort} impiega un tempo di esecuzione \(T(n)=O(n\,log\,n)\).

\section{Quick Sort}
\textbf{Quick sort} è un algoritmo di ordinamento divide et impera, il cui tempo di esecuzione nel caso peggiore è \(O(n^2)\). Nonostante un tempo di esecuzione molto lento nel caso peggiore, quick sort è uno degli algoritmi più utilizzati perchè ha un tempo medio atteso \(\Theta(n\,log\,)\) e i fattori costanti nascosti dalla notazione asintotica sono pressochè nulli. Inoltre, è un algoritmo di ordinamento sul posto, che lo rende utilizzabile anche in calcolatori con memoria limitata. 

L'idea generale di questo algoritmo si basa sui tre tipici passi del metodo divide et impera, per un sottoarray \code{A[p..r]}:
\begin{enumerate}
  \item Divide: partiziona l'array \code{A[p..r]} in due sottoarray \code{A[p..q-1]} e \code{A[q+1..r]} (eventualmente vuoti) tali che ogni elemento di \code{A[p..q-1]} sia minore o uguale di \code{A[q]} che, a sua volta, è minore o uguale di ogni elemento di \code{A[q+1..r]}.
  \item Impera: ordina i due sottoarray \code{A[p..q-1]} e \code{A[q+1..r]} chiamando ricorsivamente sè stesso.
  \item Combina: nessuna operazione di ricombinazione necessaria in quanto l'array \code{A[p..r]} è già ordinato.
\end{enumerate}

La procedura \code{quickSort} è implementata tramite il seguente pseudocodice:

\lstinputlisting{../docs/algorithms/quick_sort.txt}

Per poter ordinare un intero array \code{A}, la chiamata iniziale a tale algoritmo è \code{quickSort(A, 1, A.length)}. Si noti che all'interno di tale algoritmo viene chiamata la sottoprocedura \code{partition}, definita come segue in pseudocodifica:

\lstinputlisting{../docs/algorithms/partition.txt}

Questo algoritmo riarrangia il sottoarray \code{A[p..r]} sul posto selezionando un elemento \code{x = A[r]} come pivot, intorno a cui partizionare l'array. All'inizio di ogni iterazione del ciclo \code{for} di riga 4, per qualsiasi indice \(k\) si individuano quattro regioni:
\begin{enumerate}
  \item Se \(p\le k\le i\), allora \lstinline[mathescape]{A[k] $\le$ x};
  \item Se \(i+1\le k \le j-1\), allora \lstinline[mathescape]{A[k] $\ge$ x};
  \item Se \(k=r\), allora \code{A[k]=x};
  \item Gli indici \(k\) tali che \(j \le k \le r-1\) non hanno una particolare relazione con il pivot \(x\).
\end{enumerate}

Le ultime due righe della procedura, invece, inseriscono il pivot al suo posto nel mezzo dell'array, scambiandolo con l'elemento più a sinistra, che è maggiore di \(x\), e restituisce un nuovo indice di pivot. Il tempo di esecuzione dell'algoritmo \code{partition} con input il sottoarray \code{A[p..r]} è \(\Theta(n)\), con \(n=r-p+1\).

Il tempo di esecuzione dell'algoritmo \code{quickSort} dipende solamente da come viene partizionato l'array (in maniera bilanciata o meno) che, a sua volta, dipende da quali elementi vengono utilizzati per il partizionamento. Se il partizionamento è bilanciato, l'algoritmo ha un tempo di esecuzione \(\Theta(n\,log\,n)\), mentre nel caso peggiore, quando il partizionamento è sbilanciato, l'algoritmo converge ad una solzuione in un tempo \(\Theta(n^2)\).

\vspace*{10pt}

Il comportamento nel caso peggiore si verifica quando la subroutine \code{partition} produce un sottoproblema con \(n-1\) elementi e uno vuoto. Per calcolare il tempo di esecuzione, si suppone che questo sbilanciamento si verifichi per ogni chiamata ricorsiva. Il partizionamento costa un tempo di esecuzione \(\Theta(n)\) e, dato che uno dei due array è vuoto e l'altro conta \(n-1\) elementi, si ha un tempo totale:
\begin{equation*}
  T(n) = T(n-1) + T(0) + \Theta(n) = T(n-1) + \Theta(1) + \Theta(n)
\end{equation*}
Intuitivamente, e si sommano i costi ad ogni livello della ricorsione si ottiene una serie aritmetica, il cui valore è \(\Theta(n^2)\). Questa situazione si verifica quando l'array di partenza è già completamente ordinato. 

\vspace*{10pt}

Il comportamento nel caso ottimo si verifica quando la subroutine \code{partition} produce due sottoproblemi di dimensione non maggiore di \(n/2\): in questo caso il tempo di esecuzione dell'algoritmo è molto più rapido e avviene in un tempo totale \(T(n) \le 2T(n/2) + \Theta(n)\), che per il secondo caso del teorema dell'esperto, ha soluzione \(T(n)=\Theta(n\,log\,n)\). Si noti, inoltre, che nel caso in cui la partizione non fosse perfettamente bilanciata, l'algoritmo riuscirebbe comunque a riordinare l'array in un tempo \(T(n)=\Theta(n\,log\,n)\): questo dimostra che il \code{quickSort} è un algoritmo molto più vicino al caso ottimo che al caso pessimo, caso che si verifica in una sola istanza del problema (quando, appunto, è ordinato). 

\vspace*{10pt}

Il comportamento nel caso medio si verifica quando la subroutine \code{partition} produce una combinazione di partizioni 'buone' e 'cattive'. Si suppone, per semplicità, che le partizioni buone e cattive si alternino all'interno dell'albero di ricorsione e che quelle buone siano tutte nel caso migliore, mentre le quelle cattive siano nel caso pessimo. Si ipotizzi che nella radice dell'albero il costo di ripartizione è \(n\) e i sottoarray prodotti hanno dimensione \(n-1\) e 0 (caso pessimo), mentre nel livello successivo il partizionamento del sottoarray \(n-1\) produca due array di dimensione \((n-1)/2\) e \((n-1)/2 - 1\) (caso migliore). Il costo di una divisione cattiva, seguito da una divisione buona è comunque \(\Theta(n)\), ovvero lo stesso costo di una divisione buona: intuitivamente, quindi, la coppia divisione buona/cattiva impiega lo stesso tempo totale di esecuzione \(\Theta(n\,log\,n)\), con l'unica differenza che cambiano le costanti moltiplicative, eclissante nella notazione asintotica.

\section{Counting Sort}
Gli algoritmi analizzati fino ad ora, seppur differenti fra loro, condividono un'importante proprietà: l'ordinamento che effettuano è basato soltanto su confronti fra gli elementi di input. Questi algoritmi sono detti di ordinamento per confronti e, dati due elementi \(a_i\) e \(a_j\), eseguono uno dei test \(a_i < a_j, a_i\le a_j, a_i=a_j, a_i\ge a_j\) o \(a_i > a_j\) per determinare il loro ordine relativo. 

Gli algoritmi di ordinamento per confronti possono essere visti in termini di alberi di decisione, ovvero alberi binari completi che rappresentano i confronti fra gli elementi effettuati da un determinato algoritmo di ordinamento. Ora, l'esecuzione di un algoritmo di ordinamento corrisponde ad indicare su tale albero un cammino semplice che collega la radice dell'albero con una foglia (nello specifico, una delle foglie che rappresentano le permutazioni ordinate dell'array di ingresso). Ogni nodo interno di tale albero rappresenta un confronto: il sottoalbero sinistro corrisponde a confronti del tipo \(a_i \le a_j\), mentre quello destro corrisponde a confronti del tipo \(a_i > a_j\). Si noti che sulle foglie sono presenti tutte le possibili permutazioni della sequenza di input: dunque un albero di decisione può presentare più di \(n!\) foglie, in quanto alcune permutazioni potrebbero comparire più volte, ma meno di \(2^h\) (con h, l'altezza dell'albero). La lunghezza del cammino semplice più lungo dalla radice di un albero di decisione ad una delle sue foglie rappresenta il numero di confronti che un determinato algoritmo di ordinamento deve svolgere nel caso peggiore: questo numero è equivalente all'altezza dell'albero stesso. Si introduce quindi il seguente teorema, che determina un limite inferiore sul tempo di esecuzione degli algoritmi di ordinamento per confronti:

\begin{theorem}
  Qualsiasi algoritmo di ordinamento per confronti richiede \(\Omega(n\,log\,n)\) confronti nel caso peggiore.
\end{theorem}

\noindent
da cui deriva anche:

\begin{theorem}
  Ogni albero di decisione di un algoritmo di ordinamento di n elementi ha altezza \(\Omega(n\,log\,n)\).
\end{theorem}

\vspace*{10pt}

Una volta determinato il limite inferiore del tempo di esecuzione degli algoritmi di ordinamento, si introduce qui l'algoritmo \textbf{counting sort}, che riordina un array di dimensione \(n\) in un tempo lineare \(\Theta(n)\). 
Tale algoritmo suppone che ciascuno degli elementi di input sia un numero intero compreso nell'intervallo da 0 a \(k\in \mathbb{N}\) e determina per ciascun di essi il numero di elementi minori; utilizza poi questa informazione per inserire l'elemento corrente direttamente nella giusta posizione nell'array di output. Ad esempio, se l'elemento \(x\) è più grande di 5 altri elementi, allora verrà inserito nella posizione 6 dell'array di output.

Nel codice di counting sort, si suppone che l'input sia un array \code{A[1..n]}, con \(n\)=\code{A.length}. Occorrono altri due array: l'array \code{B[1..n]}, che contiene l'output ordinato, e l'array \code{C[0..k]} fornisce la memoria temporanea di lavoro. In pseudocodice:

\lstinputlisting{../docs/algorithms/counting_sort.txt}

Dopo che il ciclo \code{for} in riga 3 inizializza a zero tutti gli elementi dell'array \code{C}, ogni elemento dell'input viene esaminato con il secondo ciclo \code{for} (in riga 5): se il valore di un elemento è \(i\), viene incrementato il valore di \code{C[i]}. Dunque, dopo la riga 6, l'array \code{C} contiene il numero di elementi uguagli ad \(i\), per ogni \(i=0,1,...,k\). Le righe 7 e 8 determinano, per ogni \(i=0,1,...,k\), quanti elementi di input sono minori o uguali a \(i\). Infine, il ciclo \code{for} di riga 9 inserisce l'elemento corrente nella posizione corretta all'interno dell'array di output \code{B}. 

Il primo ciclo \code{for} impiega un tempo di esecuzione \(\Theta(k)\), il secondo un tempo \(\Theta(n)\), il terzo un tempo \(\Theta(k)\) e, infine, il ciclo \code{for} di riga 9 impiega un tempo di esecuzione lineare \(\Theta(n)\). Dunque, il tempo di esecuzione totale dell'algoritmo \code{countingSort} è \(T(n)=\Theta(n+k)\). Di solito, questa procedura viene utilizzata quando \(h=\Theta(n)\), facendo si che il tempo totale di esecuzione sia un \(\Theta(n)\). 

L'algoritmo appena analizzato batte il limite inferiore di tempo \(\Omega(n\,log\,n)\) per gli algoritmi di ordinamento per confronti perchè non confronta nessun elemento dell'input, ma utilizza il valore di ogni elemento come indice di un array \code{C} di appoggio. 





  
\end{document}
