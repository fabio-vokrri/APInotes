\chapter{Algoritmi}
In questo capitolo si analizzano a fondo i principali algoritmi di ordinamento e i relativi tempi di esecuzione. Prima di tutto, però, è necessario dare una definizione generale di algoritmo:

\begin{definition}
  Un algoritmo è una procedura di calcolo ben definita che prende un certo valore, o un insieme di valori, in input e genera un valore, o un insieme di valori, in output. Dunque, un algoritmo è una serie di passi computazionali che trasformano l'input in output.
\end{definition}

Un algoritmo può anche essere visto come uno strumento per la risoluzione di un problema computazionale ben definito: sotto questo sguardo, un algoritmo si definisce corretto se, per ogni istanza di input, termina con l'output corretto. Se un algoritmo è corretto, allora risolve quel determinato problema computazionale. Esistono molti modi per poter specificare un determinato algoritmo: si può utilizzare la lingua italiana o inglese, ma anche un linguaggio di programmazione come C, C++, JAVA e Pascal, o ancora tramite uno pseudocodice.

\section{Pseudocodifica}
La pseudocodifica può avvenire in molti modi, ma nel seguito si utilizzerano le convenzioni qui riportate:
\begin{itemize}
  \item L'indentazione serve ad indicare la struttura a blocchi dello pesudocodice, in modo da comprendere quali istruzioni appartengono, per esempio, ad un ciclo for, a un ciclo while o ad un if-else statement. Non sono utilizzate le parentesi graffe o parole chiave come begin ed end in quanto appesantiscono la sintassi;
  \item I costrutti iterativi \code{while}, \code{for}, \code{repeat-until} e il costrutto condizionale \code{if-else} hanno interpretazioni simili a quelle dei comuni linguaggi di programmazione. Il contatore del ciclo mantiene il suo valore dopo la fine del ciclo, quindi il valore che ha provocato la terminazione del ciclo stesso. Inoltre, si utilizza la parola chiave \code{to} quando il ciclo \code{for} incrementa il valore del suo contatore ad ogni iterazione, mentre si utilizza la parola chiave \code{down to} nel caso la variabile venga decrementata;
  \item Le assegnazioni di un valore ad una certa variabile avviene con il simbolo \code{:=}, differente dall'operatore \code{=}, che invece indica l'eguaglianza di due valori all'interno di un costrutto \code{if};
  \item Per identificare un elemento appartenente ad un array, si utilizza la notazione con le parentesi quadre, al cui interno si indica l'indice dell'elemento a cui si vuole accedere: \code{array[i]}; Per indicare un intervallo di valori all'interno dell'array si utilizza la seguente sintassi: \code{array[i..j]}, con cui si indica la sottomatrice composta dagli elementi compresi fra \(i\) e \(j\); 
  \item I dati utilizzati sono tipicamente organizzati in oggetti, formati da attributi, a cui si accede tramite la notazione punto: \code{oggetto.proprietà}. Le variabili che rappresentano un determinato oggetto sono trattate come puntatori a tale oggetto. Un puntatore che non fa riferimento ad alcun oggetto è inizializzato con il valore \code{NIL};
  \item I parametri vengono passati ad una procedura per valore: la procedura chiamata riceve una sua copia dei parametri e, quindi, se a una di queste variabili è assegnato un nuovo valore, la modifica non è visibile dalla procedura chiamante. Nel caso venga passato come argomento un oggetto, viene copiato il puntatore a tale oggetto e quindi le modifiche sono visibili anche dalla procedura chiamante;
  \item L'istruzione \code{return} restituisce immediatamente il controllo al punto in cui la procedura chiamante ha effettuato la chiamata. Le istruzioni \code{return} possono anche ritornare un valore al chiamante;
  \item Gli operatori booleani \code{and} e \code{or} sono cortocircuitati. Ciò significa che nella valutazione dell'espressione \code{x and y}, si valuta prima se il valore di \code{x} sia falso, in quanto, se lo fosse, l'intera espressione sarebbe falsa e non avrebbe quindi alcun senso valutare il valore della variabile \code{y}. Al contrario, nella valutazione dell'espressione \code{x or y}, si verifica innanzitutto se il valore di \code{x} sia vero, in quanto, se lo fosse, l'intera espressione sarebbe vera e non avrebbe quindi alcun senso valutare il valore della variabile \code{y}.
\end{itemize}

Tramite queste regole è possibile definire un qualsiasi algoritmo.
